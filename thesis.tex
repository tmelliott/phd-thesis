\documentclass[english]{MastersDoctoralThesis}\usepackage[]{graphicx}\usepackage[]{color}
%% maxwidth is the original width if it is less than linewidth
%% otherwise use linewidth (to make sure the graphics do not exceed the margin)
\makeatletter
\def\maxwidth{ %
  \ifdim\Gin@nat@width>\linewidth
    \linewidth
  \else
    \Gin@nat@width
  \fi
}
\makeatother

\definecolor{fgcolor}{rgb}{0.345, 0.345, 0.345}
\newcommand{\hlnum}[1]{\textcolor[rgb]{0.686,0.059,0.569}{#1}}%
\newcommand{\hlstr}[1]{\textcolor[rgb]{0.192,0.494,0.8}{#1}}%
\newcommand{\hlcom}[1]{\textcolor[rgb]{0.678,0.584,0.686}{\textit{#1}}}%
\newcommand{\hlopt}[1]{\textcolor[rgb]{0,0,0}{#1}}%
\newcommand{\hlstd}[1]{\textcolor[rgb]{0.345,0.345,0.345}{#1}}%
\newcommand{\hlkwa}[1]{\textcolor[rgb]{0.161,0.373,0.58}{\textbf{#1}}}%
\newcommand{\hlkwb}[1]{\textcolor[rgb]{0.69,0.353,0.396}{#1}}%
\newcommand{\hlkwc}[1]{\textcolor[rgb]{0.333,0.667,0.333}{#1}}%
\newcommand{\hlkwd}[1]{\textcolor[rgb]{0.737,0.353,0.396}{\textbf{#1}}}%
\let\hlipl\hlkwb

\usepackage{framed}
\makeatletter
\newenvironment{kframe}{%
 \def\at@end@of@kframe{}%
 \ifinner\ifhmode%
  \def\at@end@of@kframe{\end{minipage}}%
  \begin{minipage}{\columnwidth}%
 \fi\fi%
 \def\FrameCommand##1{\hskip\@totalleftmargin \hskip-\fboxsep
 \colorbox{shadecolor}{##1}\hskip-\fboxsep
     % There is no \\@totalrightmargin, so:
     \hskip-\linewidth \hskip-\@totalleftmargin \hskip\columnwidth}%
 \MakeFramed {\advance\hsize-\width
   \@totalleftmargin\z@ \linewidth\hsize
   \@setminipage}}%
 {\par\unskip\endMakeFramed%
 \at@end@of@kframe}
\makeatother

\definecolor{shadecolor}{rgb}{.97, .97, .97}
\definecolor{messagecolor}{rgb}{0, 0, 0}
\definecolor{warningcolor}{rgb}{1, 0, 1}
\definecolor{errorcolor}{rgb}{1, 0, 0}
\newenvironment{knitrout}{}{} % an empty environment to be redefined in TeX

\usepackage{alltt}

\usepackage[utf8]{inputenc}
\usepackage[T1]{fontenc}

\usepackage{mathpazo} % Use the Palatino font by default

\usepackage[backend=bibtex,style=authoryear,natbib=true]{biblatex}



\title{Tom's awesome thesis on buses}
\author{Tom Elliott}
\date{2019}
\IfFileExists{upquote.sty}{\usepackage{upquote}}{}
\begin{document}

\maketitle

\tableofcontents
\tabularnewline

\frontmatter

\chapter*{Abstract}

This is the abstract.

\mainmatter
% Main chapters



\chapter{Introduction}

This is the intro. Literature review, the problem,
and the goals of this work (i.e., to make better predictions
that don't rely on the timetable).

\part{Real-time bus models}



\chapter{Literature review}

This is a ``review'' of what's been happening in the field of
bus arrival-time prediction and modeling.
From its routes in Kalman filtering,
through fancy ANN and SVM models,
to computer intensive particle filter models
(not for real-time applications).

What's missing: focus on improved arrival-time prediction,
rather than an OR approach.



\chapter{GTFS data and route segmentation}

Talk about the data itself: where it comes from,
the important aspects we care about
(trips/routes, shapes, stops, and stop times).


\section{Real-time data}

What's involved, frequency, and some of the
quirks (at least in our Auckland Transport example).


\section{Segmentation of routes}

This is cool --- by segmenting routes
(at intersections) we remove dependency of
speed/travel time on \emph{route} and instead
relate it to the physical road the vehicle is traveling along.



\chapter{Transit vehicle model}

This chapter should be about the model itself,
including traffic lights, dwell times, speed, etc.
It's a recursive Bayesian model,
which means the state at time $k$ is a function
of the state at time $k-1$.

\section{Particle filter}

About how it's actually implemented using a particle filter,
and the reasons why we chose that.


\part{Transit network}



\chapter{Introduction}

Here we give a detailed overview of what exactly the
transit network is
(in the previous chapter, we only alluded to it,
and assumed segments where known).

The basic idea of splitting shapes at intersections,
which we get manually, or could potentially
obtain using shape-processing methods.



\chapter{Real-time network model}

The modified Kalman filter approach to modeling
travel times as vehicles travel along each
road segment.



\chapter{Historical data based priors}

How we use Bayesian heirarchical models to
estimate the typical speed along roads in the
network, based on historical data.


\appendix
% knit_child('endmatter/appendix/app1.Rnw')



\end{document}
