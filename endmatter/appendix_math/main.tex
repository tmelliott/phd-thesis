\chapter{Numerical assessment criteria}
\label{app:error-functions}

\glsreset{rmse}
\glsreset{mae}
\glsreset{mape}

We use several formulae to assess the performance of methods. The first three are the \gls{rmse}, \gls{mae}, and \gls{mape}. In each of these, we obtain $N$ estimates of a value $\hat X_n$ of which the actual value is $X_a$ and compare the difference. The formulae are:
\begin{align}
\label{eq:app_rmse}
\text{RMSE} &= \sqrt{\frac{1}{N}\sum_{n=1}^N \left(X_a - \hat X_n\right)^2}, \\
\label{eq:app_mae}
\text{MAE} &= \frac{1}{N}\sum_{n=1}^N \left|X_a - \hat X_n\right|, \\
\label{eq:app_mape}
\text{MAPE} &= \frac{100}{N}\sum_{n=1}^N \left|\frac{X_a - \hat X_n}{X_a}\right|.
\end{align}
The units for \gls{rmse} and \gls{mae} are the same as the units of $X$, while \gls{mape} is a percentage. In all cases, smaller values indicate better estimation or prediction.


\Gls{rmse} and \gls{mae} give similar results, though the former is more sensitive to extreme values. \gls{mape} is relative to the \emph{size} of the true value, so, in the context of forecasting, it is more sensitive to errors in short-term predictions.

Another criteria is \gls{picp}, which is the proportion of true values that lie within the prediction interval. Given $N$ prediction intervals $(X_{n,\ell}, X_{n,u})$ for which the actual value is $X_a$,
\begin{equation}
\label{eq:picp}
\text{PICP} = \frac{1}{N}\sum_{n=1}^N I_n, \quad
I_n =
\begin{cases}
1 & X_a \in (X_{n,\ell}, X_{n,u}), \\
0 & \text{otherwise.}
\end{cases}
\end{equation}
\Gls{picp} allows us to assess the performance of the arrival time prediction distribution as a whole: if \gls{picp} is smaller than the nominal coverage of the intervals, this implies that not enough uncertainty is being incorporated into the predictions.



\phantom{\gls{rmse},\gls{mae},\gls{mape},\gls{picp}}
