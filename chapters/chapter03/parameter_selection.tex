\subsection{Parameter selection}
\label{sec:pf_params}

Now that we understand the issues with the data and can deal with them, we can begin determining the values of the model parameters. Some of these are fixed and constant across all vehicles, routes, and stops, for example, \gls{gps} error, $\GPSerr$, system noise, $\Vnoise$, and minimum dwell time, $\mindwell$. Others inevitably vary between routes, stops, and time of day, such as stopping probability, $\Prstop$, and dwell time, $\dwell$.


\subsubsection{GPS error}
\label{sec:pf_params_gps}

The \gls{gps} or \emph{measurement} error used in the model has a strong effect on performance, as we saw in \cref{sec:pf_issues}. We can get a simple estimate of \gls{gps} error by examining the distribution of observations around the route path; that is, by computing the shortest distance between the route and each observation, which is graphed in \cref{fig:pf_param_gps}. We see two modes at about 0.5 and 2.5~meters, which could be due to a multitude of reasons. One likely one, however, is \emph{road width}, since the route shapes typically run in the middle of the road and the buses drive either side of the centre line.\footnote{New Zealand roads are 2.5--3.5~m wide, \url{https://www.nzta.govt.nz/assets/resources/road-traffic-standards/docs/rts-15.pdf}.} Single-lane roads will have a smaller average distance between the bus and the centre line, while for roads with two or more lanes, this will be larger. Additionally, some roads have a median (either painted or raised) which further increases the distance between the bus and the ``centre line''. Some combination of these could result in the distribution shown in \cref{fig:pf_param_gps}.



\begin{knitrout}\small
\definecolor{shadecolor}{rgb}{0.969, 0.969, 0.969}\color{fgcolor}\begin{figure}
\includegraphics[width=\maxwidth]{figure/pf_param_gps-1} \caption[Distribution of distance from observation to nearest point on the route, truncated to 10~m]{Distribution of distance from observation to nearest point on the route, truncated to 10~m.}\label{fig:pf_param_gps}
\end{figure}


\end{knitrout}

The other issue is the heavy tail in the distribution of distance to the path, which we truncated to 10~meters to more easily see the modes in \cref{fig:pf_param_gps}. \gls{gps} devices usually have good accuracy, but occasionally they may be quite far off of the true location, possibly due to physical interference. 6\% of bus observations were more than 10~meters from the shape, excluding any observations greater than 50~meters since these were most likely attributed to the wrong trip (and therefore not anywhere near the route path). In the current version of our application, we skip observations that are more than 50~meters from the shape path.





\subsubsection{System noise}
\label{sec:pf_params_noise}

The definition of system noise is model-dependent; for transition models $f_{A1}$ and $f_{A2}$ it is \emph{the average change in speed per second}, while for model $f_{A3}$ it is \emph{the average change in acceleration per second}. From the simulations in \cref{sec:pf_issues}, we demonstrated that system noise affected the performance of the particle filter (how often resampling is required) but neither the degeneration rate nor parameter estimation.

Unlike \gls{gps} error, it is not possible to estimate system noise directly from the data. Indeed, most of the time a vehicle's speed is constant but may change suddenly at specific locations (which are unknown), so the system noise must allow for this. We found that a smaller value of system noise under the second transition model $f_{A2}$ gave the best results in terms of sampling possible trajectories, and as such this was the model used during the simulation.



\subsubsection{Dwell times}
\label{sec:pf_params_dwell}

We can observe a large proportion of dwell times at stops by compiling all those for which we have observed both arrival times $\Varr_{srm}$ and departure times $\Vdep_{srm}$ at stop $m$ of trip $r$ on day $s$ giving us a set of dwell times
\begin{equation}
\label{eq:dwell_time_obs}
\Vdwell_{srm} = \Vdep_{srm} - \Varr_{srm}.
\end{equation}

\begin{knitrout}\small
\definecolor{shadecolor}{rgb}{0.969, 0.969, 0.969}\color{fgcolor}\begin{figure}
\includegraphics[width=\linewidth]{figure/observed_dwell-1} \caption[Distribution of dwell times observed over the course of five days, truncated at one minute]{Distribution of dwell times observed over the course of five days, truncated at one minute.}\label{fig:observed_dwell}
\end{figure}


\end{knitrout}

There is no precise way to measure the minimum dwell time parameter $\mindwell$. \citet{Hans_2015} used 14.25~seconds as the time lost due to deceleration and acceleration, and 4.1~seconds as the time needed to open and close the doors. Other studies have found values of 3--42~seconds for acceleration and deceleration \citep{Robinson_2013}, and 2--5~seconds for doors to open and close \citep{Meng_2013}.


% which is marked by a dashed vertical line in \cref{fig:observed_dwell}. There are a few dwell times less than this, but given the spike at 6~seconds, it seems reasonable to continue with this value.

The raw data from five days of observations are shown in \cref{fig:observed_dwell}. Here, we see an interesting pattern with apparent peaks every nine~seconds. While we could not determine the precise cause, we assume it to be due to a systematic problem in the arrival time recording system used to collect the data. However, this makes the first peak at 9~seconds difficult to evaluate: are there dwell times this short (contrary to the literature) or is it an artefact? For example, if a bus passes a stop without stopping and reports an arrival time followed a few seconds later by a departure time, does the 9-second anomaly affect this?

From the historical dwell time data, we were able to estimate the mean and variance of dwell time for each stop $j$, $\bar\dwell_j$ and $\dwellvar_j$, respectively, which could then be used in the dwell time model. The dwell times for each stop calculated above \emph{include} the minimum dwell time phase. To avoid having to recompute each stop's dwell time parameter whenever $\mindwell$ is changed, we adjusted the mean of each stop's dwell time at run time to account for the minimum dwell time. That is, we use $\dwell_j = \bar\dwell_j - \mindwell$ in \cref{eq:stop_dwell_time}.

As for the probability of stopping parameter, $\Prstop$, this cannot be estimated from these data. Using a vague value of $\Prstop=0.5$ for the particle filter is sufficient; however, this has a more dramatic effect on arrival time prediction as we will see in \cref{cha:prediction}.
