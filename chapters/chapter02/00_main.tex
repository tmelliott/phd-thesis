\chapter{\Rt{} transit data}
\label{cha:data}



\AT{} operates a fleet of vehicles consisting of buses, trains, and ferries to transport passengers around Auckland, New Zealand's most populous urban region \citep{StatsNZ_2019}, from Wellsford in the north to Pukekohe in the south. Serving the region are 1040~routes\footnote{As of 12 August 2019, sourced from \url{https://at.govt.nz/}.}, each providing a service from an \emph{origin} to a \emph{destination} via a specified sequence of \emph{stops} at which passengers may board or alight.


For each route, one or more \emph{trips} are scheduled to operate, typically departing from the origin stop at a fixed time, with approximate arrival times for the intermediate stops along the way. In some cases, services wait at specific stops until the scheduled departure time, known as a \emph{layover}; in others, drivers may adjust their speed to adhere to the schedule. In most cases, however, drivers do not heed the schedule and drive along the route at whatever speed is reasonable\footnote{This is at least what I have observed over 10~years of catching public transport.}. Not all trips run every day; typically trips are scheduled as `weekday', `Saturday', or `Sunday and public holidays'. Altogether there are (at the time of writing) 24,919~trips provided by \AT{}, 14,022 of which run every weekday.


The assignment of routes, trips, and stops is a common occurrence in transport systems, to the extent that it was formalised by Google's \emph{\acrfull{gtfs}} \citep {GoogleDevelopers_2006}. \GTFS{} provides a detailed specification of how transit data should be organised, making it significantly easier for external systems to access, with the primary example being Google Maps. However, plenty of other applications exist that make use of it\footnote{Including the one we are presenting.}. \GTFS{} has now been adopted by 1,234~providers across 672~locations globally which are shown in \cref{fig:gtfs_feeds}. \Cref{sec:gtfs} provides further information on \gls{gtfs}.

\begin{figure}
\centering
\includegraphics[width=0.8\textwidth]{figure/gtfs_feeds.png}
\caption[Locations of GTFS feed providers]{Locations of the 672~\GTFS{} feed providers,\\
sourced from \url{http://transitfeeds.com/} on 12 Feb 2020.}
\label{fig:gtfs_feeds}
\end{figure}


The vehicles servicing the trips are observed as they travel along the route, either as intermittent \GPS{} locations (\emph{vehicle positions}), or on arrival at or departure from bus stops along the way (\emph{trip updates}). The structure of this data is specified by `GTFS-realtime',\footnote{\url{https://developers.google.com/transit/gtfs-realtime}} and is described in detail in \cref{sec:realtime-data}, along with a discussion of the issues encountered using the \AT{} data.


One of the main ideas we discussed in \cref{sec:literature} was the importance of combining data from multiple routes when predicting arrival times. As it stands, there is no direct method for determining if two routes overlap---that is, share a common path---from \GTFS{} data. The most straightforward approach is to compare subsequences of stops, with routes servicing the same stops most likely following the same path between them. There are, however, several situations where this is not the case. We formalise this idea, as well as expand on it, in \cref{sec:route-segments}.


The real-time nature of the application demands that predictions are made as soon after observing the data as possible. For this reason, \glspl{rbm} are a logical choice for estimating states in real-time and have been used extensively in the literature. This family of models are commonly used in vehicle tracking applications \citep{Zhao_1997,Mutambara_2000,Carpenter_1999,Wall_1999}, allowing the vehicle's state (such as speed) to be estimated from a real-time sequence of noisy measurements of its location. The final section of this chapter provides an introduction to the \glspl{rbm} used throughout the remainder of this thesis.


\section{GTFS}
\label{sec:gtfs}

The adoption of GTFS \citep{GoogleDevelopers_2006} to public transport agencies around the world has made it possible for apps such as Google Maps to access and display public transport data to users, regardless of their location. The main goal of GTFS is to specify, in detail, how transit data should be organised so that it is consistent across agencies around the world. Whether you are in Auckland, Paris, or C\'ordoba, Google Maps is able to show you public transport directions because their transit data are defined using the same standard.


\GTFS{} consists of two components, \emph{static} and \emph{\rt{}}. The static component specifies how information portaining to the schedule, fares, and route geography are organised, while the \emph{\rt{}} component specifies the format for \rt{} data: vehicle locations, stop updates and \glspl{eta}, and service advisories. Each of the static and \rt{} components are implemented by a transit provider; for example \AT{}'s \GTFS{} service is hosted at \url{https://dev-portal.at.govt.nz}.


\subsection{Static GTFS}
\label{sec:gtfs_static}


\begin{table}[t]
\centering
\begin{tabular}{ll}
\toprule
Term & Definition \\
\midrule
route & a collection of \emph{trips} that are displayed to comuters
as a single service \\
trip & a journey servicing two or more stops at a specific time \\
stop & a location where passengers are picked up or dropped off \\
stop time & the (scheduled) times at which vehicles
will arrive at stops for each trip \\
shape & the GPS track a vehicle will take for a specific route \\
\bottomrule
\end{tabular}
\caption{Definitions of relevant GTFS terms, taken from
\url{https://developers.google.com/transit/gtfs/reference/}}
\label{tab:gtfs_terms}
\end{table}


There are several components of GTFS that are of particular interest to us: routes, trips, stops, stop times, and shapes. The definitions of these terms are given in \cref{tab:gtfs_terms}. Extensive documentation can be found on these and the other components on the GTFS website\footnote{\url{https://developers.google.com/transit/gtfs/}} \citep{GoogleDevelopers_2006}.


\Cref{fig:gtfs_nw} demonstrates a single \emph{route}, along which there are two active \emph{trips} (A and B). The route's \emph{shape} is represented by the line connecting the six \emph{stops} numbered 1--6. The \rt{} arrivals board is shown for stop~5, displaying the scheduled \emph{stop time} (arrival time) for each trip at that stop. The additional information displayed is described in the next section.



\begin{knitrout}\small
\definecolor{shadecolor}{rgb}{0.969, 0.969, 0.969}\color{fgcolor}\begin{figure}

{\centering \includegraphics[width=\textwidth]{figure/gtfs_nw-1} 

}

\caption{A \gls{gtfs} diagram of a single \emph{route}, showing two \emph{trips} (A and B) travelling along the route's \emph{shape} path. The numbered squares represents \emph{stops}, and the scheduled arrival times of each trip at stop 5 are displayed in the \emph{stop times} box.}\label{fig:gtfs_nw}
\end{figure}


\end{knitrout}

\begin{knitrout}\small
\definecolor{shadecolor}{rgb}{0.969, 0.969, 0.969}\color{fgcolor}\begin{kframe}


{\ttfamily\noindent\itshape\color{messagecolor}{\#\# Loading required namespace: transitr}}\end{kframe}
\end{knitrout}

Static \GTFS{} data is typically distributed by transport providers as plain text files, one for each of the components (such as \verb+routes.csv+ and \verb+trips.csv+). Often these are available to download as a single ZIP archive, so the data can easily be loaded into a \emph{relational database}. \Cref{app:gtfs} describes the relational structure of the \GTFS{} database. The R package developed as part of this work provides the \verb+create_gtfs()+ function to do this automatically:
\begin{knitrout}\small
\definecolor{shadecolor}{rgb}{1, 1, 1}\color{fgcolor}\begin{kframe}
\begin{alltt}
\hlkwd{library}\hlstd{(transitr)}
\hlstd{nw} \hlkwb{<-} \hlkwd{create_gtfs}\hlstd{(}\hlstr{"at_gtfs.zip"}\hlstd{,} \hlkwc{db} \hlstd{=} \hlstr{"at_gtfs.sqlite"}\hlstd{)}
\end{alltt}
\end{kframe}
\end{knitrout}




Within these separate files or tables, the data is stored as per \GTFS{}. Importantly, shapes are stored as sequences of coordinates that draw a path on a map, while stops are represented as a single coordinate marking the location of the bus stop. Stop times are stored in trip-stop pairs\footnote{It is a pivot table, if you know what that is.}, with one row for every stop of each trip, along with the scheduled arrival and departure times\footnote{There are 732,184 rows in the stop times table for Auckland Transport!}.



On its own, the static \GTFS{} information is as useful as a printed timetable, allowing simple journey planning to take place. It therefore provides the necessary ``fallback'' state in cases where no \rt{} information is available for a given trip. In later sections, scheduled stop times are used to obtain \emph{prior information}, a core component of the Bayesian paradigm.



\subsection{\Rt{} GTFS}
\label{sec:gtfs_rt}

The \rt{} component of \GTFS{} is responsible for handling vehicle positions, trip updates (arrivals and departures from stops), and service alerts (cancellations and stop closures, for example). Data is processed by a central server, and then stored in an appropriate fashion to enable quick access via \glspl{api}. There is therefore the additional need of a server that can handle vast numbers of \gls{api} requests, meaning only a subset of the transport providers using \GTFS{} have also implemented the \rt{} component. Below we give a brief summary of these components, but further information can be read on the \GTFS{} website\footnote{see \url{https://developers.google.com/transit/gtfs-realtime/}.}.


\subsubsection{Vehicle positions}
\label{sec:gtfs_rt_vehicle}

The key components of a vehicle position (in the context of \GTFS{}) are a \emph{vehicle descriptor} which includes information about the physical vehicle, a \emph{trip descriptor} which holds information about the trip being serviced, a \emph{timestamp} specifying exactly when the observation was made, and a \emph{position} containing the actual data, such as the \gls{gps} observation.

The specification also allows for additional measurements, such as \emph{speed} or an \emph{odometer} reading. However, these were not available from \AT{} at the time this work was carried out, so we have not included them in our framework. It is well worth noting, however, that they could be integrated with minimal effort if they become available.


\subsubsection{Trip updates}
\label{sec:gtfs_rt_trip}

As vehicles equipped with \gls{avl} technology arrive at and depart from stops, information about their time of arrival, and most importantly \emph{schedule adherence}, is stored in trip updates. These also contain a \emph{trip descriptor}, as well as one or more \emph{stop time updates}. \AT{} provides a single stop time update for the most recently visited stop; however, it is possible to retain all previous stop time updates, as well as provide predictions for upcoming stops.


Each stop time update reports either the arrival or departure time and, where schedule information is available, the schedule adherance by way of an \emph{arrival} or \emph{departure delay} (in seconds). The onboard \gls{avl} device is responsible for detecting these events, and they are not necessarily linked to the \gls{gps} position of the vehicle. In Auckland, each trip update also triggers a vehicle location update, but the coordinates are those of the \emph{stop} and not the vehicle, which can cause problems we address in \cref{sec:realtime-data}.



\subsubsection{Service alerts}
\label{sec:gtfs_rt_alerts}

Less important for the current work, but essential for reliable \gls{rti}, \emph{service alerts} enable transit operators to modify the static \GTFS{} in \rt{}. So, when a trip is cancelled, they have the ability to send out a service alert announcing the cancellation\footnote{Although they often don't.}. This is often displayed to passengers as a ``C'' on the \rt{} board. It is also possible to add trips, for example during special events, or to reroute trips around stop closures, but this is beyond the scope of our work.



\subsection{Accessing \rt{} data (API)}
\label{sec:gtfs_rt_api}

In order to distribute vehicle locations, trip updates, and service alerts to passengers quickly and usefully requires more than just a \rt{} board at bus stops. Personal mobile devices have revolutionised the way we live our lives, and developers are constantly creating new apps to assist with everyday activities. Included in these are transit apps, which are capable of passing on \rt{} \GTFS{} data to passengers.

The most common method of distributing \rt{} data is via an \gls{api}. These are, in simple terms, a fixed web address from which developers can request a data file---either JSON or, in the case of some \GTFS{} systems, protobuf \citep{cn}---which can then be parsed and displayed to users. Usually developers need to register for an \emph{\gls{api} key} which helps to control server demand by limiting the number of requests a user can make, or controlled access to specific data. Our R package includes the ability to connect to a \GTFS{}-based \gls{api} easily using the following command:
\begin{knitrout}\small
\definecolor{shadecolor}{rgb}{1, 1, 1}\color{fgcolor}\begin{kframe}
\begin{alltt}
\hlstd{url} \hlkwb{<-} \hlstr{"https://api.at.govt.nz/v2/public/realtime/vehiclelocations"}
\hlstd{nw} \hlkwb{<-} \hlkwd{load_gtfs}\hlstd{(}\hlstr{"at_gtfs.sqlite"}\hlstd{)} \hlopt
    \hlkwd{realtime_feed}\hlstd{(url)} \hlopt
    \hlkwd{with_headers}\hlstd{(}
        \hlstr{"Ocp-Apim-Subscription-Key"} \hlstd{=} \hlkwd{Sys.getenv}\hlstd{(}\hlstr{'APIKEY'}\hlstd{)}
    \hlstd{)}
\end{alltt}
\end{kframe}
\end{knitrout}

\section{Charactersitics of real-time transit data}
\label{sec:realtime-data}

There are a range of \gls{avl} technologies which allow transit vehicles to report their locations in real-time. The most common of these is now the \gls{gps}, which provides the longitude and latitude of the vehicle. In the simplest of deployments, each vehicle reports its location to a central server at a fixed time interval. The server then collates the reports from all buses in the fleet and makes them available via an \gls{api}.


Another type of real-time data available to us is arrival and departure information at bus stops. When a vehicle arrives at or departs from a stop,\footnote{Or thinks it does, see \cref{sec:vp_data}.} it reports to the server which stop it is at, and its time of arrival or departure. The server then computes the delay (the time between actual and scheduled arrival or departure), collates the data from multiple vehicles, and also makes these available through an \gls{api}.


Real-time vehicle information can be displayed in one of two ways. \Gls{gps} positions are displayed on a map accessed through a mobile app, allowing commuters to see where their bus was when it last reported its location. For trip updates, the \emph{current delay} is added to the scheduled arrival time and displayed to commuters, usually as a \emph{time until arrival}, in minutes.\footnote{That is ``scheduled arrival time + delay - current time''.} The \gls{eta} can be displayed either on a mobile app or, more commonly, on a \gls{dms} at the stop.


What we have just described is, in fact, the entirety of \gls{rti} in Auckland and some other transit locations around the world. While at first it seems an adequate solution, discussion with just about any regular public transport user suggests otherwise. The reasons for this become apparent with a little scrutiny, which we will now uncover.


\subsection{Vehicle positions}
\label{sec:vp_data}

Every measurement of a data point comes with some associated error. In the case of \gls{gps} devices, this error is usually small with precision depending on the quality of the device. However, any device can succumb to several factors which may place the bus far from its expected path, the primary reason being buildings or other obstacles resulting in a poor signal \citep{Xinghao_2013,Mutambara_2000,Zhao_1997}. Surprisingly, however, this is not the main issue with vehicle position data.


Object tracking has been well studied, and many algorithms exist for tracking an object through space using \gls{gps} observations. However, these usually take high-frequency observations (\citet{Gustafsson_2002} updated the car's location twice per second) which can generate an almost exact real-time estimate of the object's actual position.


Many examples of real-time object tracking exist, but the most relatable to most readers will be in their pocket. When getting directions from your phone, the maps application requests the user's phone's location continuously, providing the exact location with a second or less of delay. However, have you ever been driving along, following directions, and accidentally missed the turn-off? Often, the maps application will show you as \emph{on course} for several seconds until it realises that you have well and truly gone off track and reroutes you. This is an example of a real-time position tracking algorithm that is attempting to follow the device's location while simultaneously accounting for inherent noise in the measurements. When the driver first goes off track, the algorithm assumes this is a measurement error and assumes the vehicle is on course. Eventually, the error becomes large and consistent enough that the model stops assuming the driver is following the planned route.


With real-time transit data, the frequency of observations is vastly reduced, with observations obtained with anything from ten seconds to a minute (or more) between them. This makes it very difficult to estimate the vehicle's exact location. On receiving a vehicle position that seems incorrect, we must wait until we receive the next observation to determine if it was the result of a bad measurement or a real event.


Another major complication with the Auckland Transport vehicle data is that vehicles often report their location when arriving at or departing from a bus stop. However, instead of reporting their \gls{gps} location as measured from the \gls{gps} device, they report the location of the bus stop itself, which is known exactly. So, what happens when the bus approaches a bus stop at speed, only to come to a halt at an intersection 100~m before hand? The vehicle's \gls{gps} continues predicting the trajectory and places the vehicle at the bus stop before it actually arrives. This triggers a trip update (section~\ref{sec:tu_data}), which itself produces a vehicle position update \emph{at the bus stop}. However, a passenger standing at the stop will see the bus sitting at the lights even though the \gls{dms} displays the bus as having arrived. For the passenger waiting at this stop, this is of no concern, as they can see the bus and have no need for real-time data. For passengers farther down the line, however, this can have some frustrating implications.

Another related phenomenon exists in which the bus may appear to go backwards according to the sequence of \gps{} coordinates, discussed in detail in \cref{sec:data_issues}. The main consequence of this problem is that within the data processing component of our application, we check for vehicle position updates associated with trip updates and remove them (that is, we use the trip update and not the vehicle position).\footnote{This seems straightforward, but was frustrating until identified.}




\subsection{Trip updates}
\label{sec:tu_data}

As alluded to earlier, trip updates are prone to measurement error. Without human intervention, it is challenging for the \gls{gps} tracking system on the bus to determine exactly when the bus arrives at or leaves a bus stop. In situations like the one described above, the arrival time may be reported before the bus arrives, resulting in a premature arrival time and, more importantly, \emph{a reduced delay}. Traffic lights may hold up a bus for a minute (or more), so the bus may, for argument sake, appear to arrive precisely on time with a delay of zero minutes. The result of this is the propagation of the current delay to all future stops, which then display an \gls{eta} that matches the scheduled arrival time. However, two minutes later, after the bus has finally arrived at the stop, dropped off and picked up passengers, it departs, triggering another trip update. The delay, now two minutes, is propagated to upcoming stops. Passengers waiting at these stops see the \gls{eta} jump suddenly by two minutes, as demonstrated in \cref{fig:tu_eta_jump}, leading inevitably to much frustration for passengers.

\begin{knitrout}\small
\definecolor{shadecolor}{rgb}{0.969, 0.969, 0.969}\color{fgcolor}\begin{figure}

{\centering \includegraphics[width=.8\textwidth]{figure/tu_eta_jump-1} 

}

\caption[Demonstration of how ETAs are percieved by passengers under the current system]{The current system displays ETAs to passengers as integer minutes which decrease over time. When new data is recieved (marked by an arrow) the ETA is updated, which may result in a sudden ``jump'', as demonstrated.}\label{fig:tu_eta_jump}
\end{figure}


\end{knitrout}


The reverse can also occur, for example, if for some reason the bus reports its arrival or departure late. However, more prevalent is that the update is skipped altogether, so a bus that was five minutes behind schedule has made up several minutes and arrives at the bus stop while the \gls{dms} still shows it as being two minutes away. At first, this may seem like a good thing: passengers at the stop have a shorter wait. Other passengers making their timely way to the bus, however, may have to sprint or miss the bus altogether, which is not ideal.


The main repercussion of the described issues is that arrival and departure time data become very noisy and difficult to trust. They do, however, provide a lot of information without which it would be challenging to infer the trajectory of a bus between stops---which is the primary aim of this part of our work. We discuss this further in \cref{sec:pf-likelihood} when we present the likelihood function.

\section{Constructing a network from GTFS data}
\label{sec:route-segments}

In \cref{sec:literature}, we reported how arrival time prediction is greatly improved when data from multiple routes is combined to estimate traffic conditions \citep{cn}. However, many of these applications were specific to a certain set of routes, or used external information (such as automatic toll readers and taxis) to get \rt{} traffic information. We wanted to develop a \emph{simpler}, \emph{more generalised} approach that uses solely \GTFS{} data (stops and shapes) to construct a network of non-overlapping\footnote{Except of course reverse directions.} road segments. The network should consist of \emph{nodes} (which could be intersections or bus stops) connected by \emph{edges}, or road segments.


Essentially, we want to detect where two routes overlap, even partially, so tspeed information from one can be used to improve \glspl{eta} for the other. An obvious place to start is by looking at common subsequences of bus stops. Using stops as nodes is not ideal, however, as there are many situations where there will still be overlapping road segments. For example, an express route might skip a series of stops on a busy road, likely services by non-express versions of the same route, as well as other routes. Additionally, routes do not diverge at bus stops\footnote{Who would put a bus stop on an intersection?}, and instead overlap until an \emph{intersection} at which point they diverge.


% (Note to self: maybe this can be dropped, since it hasn't been done yet ...?)
% Getting intersection data is not easy, however, and initial attempts at doing so using Open Street Maps \citep{OpenStreetMap_2017} were unsuccessful due to the difficulty in identifying an intersection, and replication of traffic lights (one for each physical light). One possible method to reduce the instances of the second situation is to find road segments that end at the same stop, but have a different origin (or vice versa). Then we travel along the paths until they meet, at which point we place an intersection.


\begin{knitrout}\small
\definecolor{shadecolor}{rgb}{0.969, 0.969, 0.969}\color{fgcolor}\begin{kframe}


{\ttfamily\noindent\itshape\color{messagecolor}{\#\# Loading required namespace: transitr}}\end{kframe}
\end{knitrout}
One other issue specific to \AT{}'s \GTFS{} feed is that all of the object IDs are \emph{versioned}. That is, instead of a fixed route having a route ID of \verb+02702+, the date and version number are appended to it: \verb+02702-20190806160740_v82.21+. This is the same for the IDs of trips, shapes, and stops. This makes it difficult to transfer existing segments based on stops from previous versions of the \gls{api}, requiring an additional step to remove the version information from the raw data.


The algorithm we implemented uses only bus stops, but has the capability to extend to intersections, if available. To avoid the versioning issue described above, each stop is processed as though it is an intersection point, and checked to see if a node in that location already exists. If it does, the route is assigned the existing node; otherwise, a new node is created and assigned instead. As for the road segments, these are defined as unique connections between any two stops; that is, there can only be one segment going from node $A$ to node $B$ (the reverse is considered a different segment). The network can be constructed by the \verb+transitr+ package using the \verb+construct()+ function:
\begin{knitrout}\small
\definecolor{shadecolor}{rgb}{0.969, 0.969, 0.969}\color{fgcolor}\begin{kframe}
\begin{alltt}
\hlkwd{library}\hlstd{(transitr)}
\hlstd{nw} \hlkwb{<-} \hlkwd{load_gtfs}\hlstd{(}\hlstr{"at_gtfs.sqlite"}\hlstd{)} \hlopt \hlkwd{construct}\hlstd{()}
\hlstd{transitr}\hlopt{:::}\hlkwd{load_road_segments}\hlstd{(nw)} \hlopt \hlstd{head}
\end{alltt}
\begin{verbatim}
##   road_segment_id node_from node_to    length
## 1               1      5414     537  485.6548
## 2               2       537    3987  196.5567
## 3               3      3987    7265  567.0883
## 4               4      7265    1255 1049.0181
## 5               5      1255    4099  682.5452
## 6               6      4099    3436  535.0515
\end{verbatim}
\end{kframe}
\end{knitrout}

\input{chapters/chapter02/recursive_bayes.tex}

\section{\Rt{} implementation in Rcpp}
\label{sec:rt-implementation}

In \cref{sec:gtfs} we introduced the R package \verb+transitr+, which loads a \GTFS{} database and connects to a \rt{} \gls{api}. The advantage of R \citep{rcore} was its superior ability for processing data of various forms through additional packages, and providing an easy-to-use interface for users. However, when it comes to computational efficiency, C++ is the better choice. Fortunately, R offers an interface to C++ through the 'Rcpp' package developed by \cite{Rcpp}. This gives us the speed and memory management capabilities of C++ from within R.

The general structure of our program comes in two parts. The first handles data collection from a transit provider and creation of the transit network (\cref{sec:route-segments}), which are all stored in an SQLite database.  Users are able to connect a GTFS-realtime \gls{api}, as well as pass additional arguments, to the generated \verb+gtfs+ object, which are then forwarded onto the second part which runs purely within C++.

Within the C++ component there are two main phases: the setup phase and the modelling phase. During the setup, the GTFS database is loaded and parameter values establised. A vector of \verb+Vehicle+ objects is initialized which will contain the \rt{} vehicle states. The modelling phase consists of a single \verb+while+ loop which runs until the program is sent a kill signal by the operating system. It is inside this loop that all of the \rt{} modelling discussed throughout the remainder of this thesis occurs.

Within the program, we use \gls{oop} to represent objects---both static (routes and trips) and \rt{} (vehicles, road segments). These objects contain a lot of \emph{interdependence}; for example, a trip belongs to a route, and a vehicle services trips. Pointers are used in C++ to allow easy access to these relationships without any duplication of information. For example, the route number for a vehicle can be obtained using the following command:
\begin{knitrout}\small
\definecolor{shadecolor}{rgb}{0.969, 0.969, 0.969}\color{fgcolor}\begin{kframe}
\noindent
\ttfamily
\hlstd{vehicle}\hlopt{.}\hlstd{}\hlkwd{trip\ }\hlstd{}\hlopt{(){-}$>$}\hlstd{}\hlkwd{route\ }\hlstd{}\hlopt{(){-}$>$}\hlstd{}\hlkwd{route\textunderscore short\textunderscore name\ }\hlstd{}\hlopt{();}\hlstd{}\hspace*{\fill}
\mbox{}
\normalfont
\end{kframe}
\end{knitrout}
Pointers are not fixes, so the above example will work even after the vehicle changes to another trip. In addition to retrieving information, pointers can be used to pass information between objects. Later in this thesis, we use vehicle to estimate the average speed of vehicles along individual roads, and then forward these observations on to the appropriate road segment. First, we note that a route follows a sequence of road segments, which we store in C++ using a \verb+std::vector+ containing an ordered sequence of \verb+RouteSegment+ objects, each pointing to the appropriate \verb+Segment+ object. Thus, once a vehicle completes travel along the segment with index \verb+index+, the observed average speed can be passed to directly to the \verb+Segment+ object:
\begin{knitrout}\small
\definecolor{shadecolor}{rgb}{0.969, 0.969, 0.969}\color{fgcolor}\begin{kframe}
\noindent
\ttfamily
\hlstd{vehicle}\hlopt{.}\hlstd{}\hlkwd{trip\ }\hlstd{}\hlopt{(){-}$>$}\hlstd{}\hlkwd{route\ }\hlstd{}\hlopt{()}\hspace*{\fill}\\
\hlstd{}\hlstd{\ \ \ \ }\hlstd{}\hlopt{{-}$>$}\hlstd{}\hlkwd{segments\ }\hlstd{}\hlopt{().}\hlstd{}\hlkwd{at\ }\hlstd{}\hlopt{(}\hlstd{index}\hlopt{){-}$>$}\hlstd{}\hlkwd{segment\ }\hlstd{}\hlopt{()}\hspace*{\fill}\\
\hlstd{}\hlstd{\ \ \ \ }\hlstd{}\hlopt{{-}$>$}\hlstd{}\hlkwd{push\textunderscore data\ }\hlstd{}\hlopt{(}\hlstd{speed}\hlopt{,\ }\hlstd{uncertainty}\hlopt{);}\hlstd{}\hspace*{\fill}
\mbox{}
\normalfont
\end{kframe}
\end{knitrout}
This data is then handled by the segment, as we describe later in \cref{cha:network_model}.

The main issue with pointers is that, if an object is deleted or moved, a pointer may no longer point to the appropriate object, resuling in a \emph{segmentation fault} and crashing the program at runtime. However, with care, and by checking a pointer is valid before using it, we can avoid such problems. For example, the above would be better written as follows.
\begin{knitrout}\small
\definecolor{shadecolor}{rgb}{0.969, 0.969, 0.969}\color{fgcolor}\begin{kframe}
\noindent
\ttfamily
\hlstd{Trip}\hlopt{{*}\ }\hlstd{trip\ }\hlopt{=\ }\hlstd{vehicle}\hlopt{.}\hlstd{}\hlkwd{trip\ }\hlstd{}\hlopt{();}\hspace*{\fill}\\
\hlstd{}\hlkwa{if\ }\hlstd{}\hlopt{(}\hlstd{trip\ }\hlopt{==\ }\hlstd{}\hlkwc{nullptr}\hlstd{}\hlopt{)\ }\hlstd{}\hlkwa{return}\hlstd{}\hlopt{();}\hspace*{\fill}\\
\hlstd{Route}\hlopt{{*}\ }\hlstd{route\ }\hlopt{=\ }\hlstd{trip}\hlopt{{-}$>$}\hlstd{}\hlkwd{route\ }\hlstd{}\hlopt{();}\hspace*{\fill}\\
\hlstd{}\hlkwa{if\ }\hlstd{}\hlopt{(}\hlstd{route\ }\hlopt{==\ }\hlstd{}\hlkwc{nullptr}\hlstd{}\hlopt{)\ }\hlstd{}\hlkwa{return}\hlstd{}\hlopt{();}\hspace*{\fill}\\
\hlstd{}\hlkwa{if\ }\hlstd{}\hlopt{(}\hlstd{route}\hlopt{{-}$>$}\hlstd{segments}\hlopt{.}\hlstd{}\hlkwd{size\ }\hlstd{}\hlopt{()\ $<$=\ }\hlstd{index}\hlopt{)\ }\hlstd{}\hlkwa{return\ }\hlstd{}\hlopt{();}\hspace*{\fill}\\
\hlstd{Segment}\hlopt{{*}\ }\hlstd{segment\ }\hlopt{=\ }\hlstd{route}\hlopt{{-}$>$}\hlstd{segments}\hlopt{.}\hlstd{}\hlkwd{at\ }\hlstd{}\hlopt{(}\hlstd{index}\hlopt{){-}$>$}\hlstd{}\hlkwd{segment\ }\hlstd{}\hlopt{();}\hspace*{\fill}\\
\hlstd{}\hlkwa{if\ }\hlstd{}\hlopt{(}\hlstd{segment\ }\hlopt{==\ }\hlstd{}\hlkwc{nullptr}\hlstd{}\hlopt{)\ }\hlstd{}\hlkwa{return\ }\hlstd{}\hlopt{();}\hspace*{\fill}\\
\hlstd{segment}\hlopt{{-}$>$}\hlstd{}\hlkwd{push\textunderscore data\ }\hlstd{}\hlopt{(}\hlstd{speed}\hlopt{,\ }\hlstd{uncertainty}\hlopt{);}\hlstd{}\hspace*{\fill}
\mbox{}
\normalfont
\end{kframe}
\end{knitrout}

The last topic for this section is \emph{multithreading}, which is the process of using more than one CPU core to run the program with the assistance of OpenMP. This requires our program to be \emph{threadsafe}, such that two independent cores wil not adversely interact (for example by both trying to modify the same object). The simplest way around this is to perform read-only operations on common resources.

Sometimes, however, write operations are unavoidable, such as the example before with passing data to road segments. \emph{Mutex locking} provides a simple way of ensuring that only one process can perform a specified task at a time. Continuing with the same example, if two vehicles traverse a road at the same time (which happens frequently), they will both want to push their observed speed observations to this same road segment object at the same time. To prevent errors, we create a lock at the beginning of the \verb+push_data ()+ method. Now when the first vehicle calls \verb+push_data ()+, the segment is ``locked'' until the method is complete (the data is processed and stored within the segment object). If a second vehicle calls the method before the first has finished, it will find the segment locked and the process will wait until the lock has been released before it can continue. This lets us parallelise the processing of vehicles to multiple cores, greatly speeding up iteration timings.

