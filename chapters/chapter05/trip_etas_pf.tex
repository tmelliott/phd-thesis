\subsection{Particle filter \glspl{eta}}
\label{sec:trip_etas_pf}

An alternative to using a \kf{} estimate of travel times and dwell times,
which assumes Gaussian parameters,
we could instead use the current state of the vehicle,
as approximated by the \pf{},
to obtain a sample of arrival times for each stop.
This would involve the same distributions as shown above,
however needn't summarise them using the expectation and variance.


It then becomes a trival case of simulating
a travel time $(\NWstate_c)\vi_\ell$ for each particle
along each segment $\ell\in\RouteSegs$ using \cref{eq:seg_tt_dist},
and similarly a dwell time $\Tdwell_j\vi$
at each stop $j$ using \cref{eq:stop_dwell_times}.
Then the arrival time for each particle at stop $j$ is given by
\begin{equation}
\label{eq:pf_eta_estimate}
\Tarr\vi_j =
t_k +
\SegProg b\vi_\TripSeg +
\sum_{\ell=\TripSeg+1}^{\TripSeg(j)-1}
    b\vi_\ell +
\pdwell\vi_s +
\sum_{m=s+1}^{j-1} \Istop\vi_m(\mindwell + \pdwell\vi_m)
\end{equation}
where
\begin{equation}
\label{eq:pf_seg_tt}
b\vi_\ell \sim
\Normal{(\hat\NWstate_c)_\ell}{(\NWvar_c)_{\ell,\ell}}
\end{equation}
Then, using the particles' posterior weights after the last update,
the arrival time becomes
\begin{equation}
\label{eq:pf_eta_dist}
p(\Tarr_j | \Vobs_{1:k}) \approx
    \sum_{i=1}^\Np \Pwt_k \dirac (\Tarr_j - \Tarr\vi_j)
\end{equation}


However, \cref{eq:pf_seg_tt} does not take into account
correlations between road segments,
which are planned to be included in future work.
Therefore, we can replace it with the marginal distribution,
which for bivariate normal variables $X_1,X_2$
with means $\mu_1,\mu_2$,
variances $\sigma_1^2,\sigma_2^2$
and covariance $\rho$, then
\begin{equation}
\label{eq:pf_seg_tt_marginal}
% b\vi_\ell | b\vi_{\ell-1} \sim
% \Normal{
%     (\hat\NWstate_c)_\ell +
%     \frac{(\NWvar_c)_{\ell,\ell}}{(\NWvar_c)_{\ell-1,\ell-1}}
%     (\NWvar_c)_{\ell-1,\ell}
%     (b\vi_{\ell-1} - (\hat\NWstate_c)_{\ell-1})
% }{
%     (1 - (\NWvar_c)_{\ell-1,\ell}^2)
%     (\NWvar_c)_{\ell,\ell}^2
% }
X_2 | X_1 = x_1 \sim
\Normal{\mu_2 + \frac{\sigma_2}{\sigma_1}\rho(x_1 - \mu_1)}
{(1-\rho^2)\sigma_2^2}
\end{equation}
Thus the \pf{} estimates will account for any known
correlation structure in the road network.


The obvious downside of the \pf{} is that
it is computationally intensive.
So, while it makes it a lot easier to estimate
the arrival time distribution
accounting for all the uncertainty accurately
(i.e., without making normality assumptions),
we need to check that it is faesible in the \rt{} context.

