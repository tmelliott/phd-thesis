\chapter{Predicting arrival time}
\label{cha:prediction}


So far, we have introduced the structure of transit data, the construction of a ``transit network'', and presented a real-time transit vehicle model to estimate road speeds. We next described how to summarise the real-time network state using the speed estimates from the vehicle model. The single goal of all of this is to predict the arrival time of transit vehicles at bus stops. Yet, while we may have all of the pieces, we still need to put them together, and the clock is still ticking---the \glspl{eta} still need to be estimated and distributed before our passengers can make use of them.

In \cref{cha:vehicle_model} we estimated the state of a vehicle; in \cref{cha:network_model}, the state of a road. In this chapter, we shift our focus to estimate \emph{trip state}, which combines all of the information we have gained thus far. The trip state allows estimation of arrival times at stops by combining it with the network state and dwell time information, if available. The first step, therefore, is to estimate trip state (\cref{sec:trip_state}) based on the \gls{gtfs} schedule and estimated vehicle states from \cref{cha:vehicle_model}.

\begin{itemize}
\item relate to \citet{Cathey_2003} and other approaches that use `\rt{} data + travel time information (+ dwell time information)'
\end{itemize}

In the remainder of this chapter, we present a selection of models---two which use the real-time vehicle and network states, and two methods for comparison, one of which is the currently used method in Auckland and some other locations using \gls{gtfs}-realtime. Having presented the four forecasting methods (\cref{sec:prediction_arrival_time}), we present the results of arrival time prediction for a full day of observations for which we know the actual arrival time and compare the methods (\cref{sec:prediction_model_comparison}). Finally, we look at the real-time implementation and performance results in \cref{sec:prediction_performance} to assess the practicality of our proposed method.
This chapter focuses on the statistical performance of our proposed method. We examine the practical results in \cref{cha:etas}.


\section{Trip state}
\label{sec:trip_state}

At any one time, there are numerous scheduled trips within the transit network we want to obtain the \emph{state} of, which can then be updated with any real-time vehicle or network information, if available. Occasionally there is no vehicle associated with a given trip, so having a trip state makes it possible to combine schedule data with real-time network information to obtain \glspl{eta}. That is, it enables real-time arrival time prediction in the absence of real-time vehicle data.


At any time $t$, we obtain a list of scheduled trips such that the trip's start time $T_\text{start}$ is less than 30~minutes before $t$, and the trip's end time (arrival at last stop) is less than 60~minutes after $t$. The state of each trip within this window is represented by: $\Tript_k$, the time of the previous vehicle observation associated with it (if there is one); $\TripStop_k$, the current stop index; $\TripDep_k$, an indicator of whether the vehicle has departed from that stop; $\TripSeg_k$, the current segment index; and $\SegProg_k$, the progress (as a proportion) along the segment. These are stored in the trip state vector
\begin{equation}
\label{eq:trip_state}
\Tripr_k = \tvec{\Tript_k, \TripStop_k, \TripDep_k, \TripSeg_k, \SegProg_k}.
\end{equation}
The initial state is $\Tripr_0 = \tvec{T_\text{start},0,0,0,0}$. Trips use the same time subscript $k$ as vehicle locations since they are initialised based on the schedule, and then only updated when vehicle data is received.

Estimation of the various state components depends on the type of the most recent observation associated with the trip. The three possible scenarios are:
\begin{enumerate}
\item the last observation was a \emph{trip update} for stop $\TripStop$,
\item the last observation was a \emph{GPS position}, or
\item no data is available for the trip.
\end{enumerate}


\paragraph{Scenario 1:}
On arrival at stop $\TripStop$ at time $\Tript_k$, the trip's state is
\begin{equation}
\label{eq:trip_state_tu}
\Tripr_k =
\begin{cases}
\tvec{\Vtime_k, \TripStop, 0, \TripSeg(s), 0} & \text{if arrival}, \\
\tvec{\Vtime_k, \TripStop, 1, \TripSeg(s), 0} & \text{if departure},
\end{cases}
\end{equation}
where $\TripSeg(s)$ is the segment index of stop $\TripStop$.


\paragraph{Scenario 2:}
Here, the vehicle has completed travel along part of the segment, so the \emph{remaining distance} is needed. From the \pf{} in \cref{cha:vehicle_model}, the segment along which the vehicle is travelling at time $\Tript_k$ is identified as $\TripSeg_k$, and the most recently visited stop is $\TripStop_k$. The proportion of the segment travelled at time $t_k$, $\SegProg_k$, is easily calculated using the vehicle's current distance $\Vdist_k$ and the segment's start distance $\Tsegd_{\TripSeg_k}$ and length $\Tseglen_{\TripSeg_k}$:
\begin{equation}
\label{eq:trip_seg_completed_prop}
\SegProg_k = \frac{\Vdist_k - \Tsegd_{\TripSeg_k}}{\Tseglen_{\TripSeg_k}}.
\end{equation}


\paragraph{Scenario 3:}
There are three main reasons for there being no observations associated with a scheduled trip:
\begin{enumerate}
\item the trip has been cancelled, so no vehicle is servicing the trip, but the cancellation has not been entered into the real-time system;
\item the vehicle is late and has not yet started the trip; or
\item a vehicle is servicing the trip but its \gls{avl} system is either not working or is registered with the wrong trip.\footnote{This happens more often than one might first imagine.}
\end{enumerate}
There is no way to differentiate these three situations without physical investigation, so we always assume there is a bus travelling the route. However, if no bus has been observed, it is desirable to display a warning to passengers so that they are aware and can choose to catch an alternative bus instead.


It is common for passengers waiting at the first stops along a route to have no available \gls{rti} until the bus is a few minutes away. This is because the bus does not register with the server until it has begun the trip. Until this happens, the \gls{dms} displays the default scheduled time, which can be problematic if the bus is late to start the route. In extreme cases, once the scheduled time has passed the service disappears from the \gls{dms}, leaving passengers wondering whether the bus will eventually come, which is why it is necessary to display to commuters if no \gls{rti} is available for a trip.


Additionally, historical data of arrival times can be used in conjunction with the real-time network state to provide a prior estimate of arrival time, which is particularly useful for trips prone to starting late. In these situations, a single prediction is made when first initialising the trip based on historical information. Once the trip is observed (or cancelled) the prediction can be updated. The goal here is not to predict arrival times for unobserved vehicles accurately, but instead to smooth the arrival time estimates at the beginning of a trip's schedule for those situations where the vehicle does finally start its route.

\section{Arrival time prediction methods}
\label{sec:prediction_arrival_time}




\begin{knitrout}\small
\definecolor{shadecolor}{rgb}{0.969, 0.969, 0.969}\color{fgcolor}\begin{figure}

{\centering \includegraphics[width=\linewidth]{figure/layover_observance-1} 

}

\caption[Vehicle delays at layover stops]{Vehicle delays at layover stops truncated to 10~minutes early and 20~minutes late. Negative values indicate non-adherance by drivers, which make up 40\% of cases.}\label{fig:layover_observance}
\end{figure}


\end{knitrout}

The estimated trip state (and any associated vehicle) together with the road network can now be combined to forecast how long it will take for the vehicle to reach all remaining stops along the route. Dwell times can also affect arrival time uncertainty \citep{Shen_2013,Wang_2016,Robinson_2013,Meng_2013,Shalaby_2004,Hans_2015}. Some trips have the added complexity of layovers at specific stops---these are common at major points of interest, such as at a mall, or stations where many routes connect so passengers can transfer between services (see \cref{sec:gtfs}). In theory, drivers wait at these stops until the scheduled departure time before continuing the trip; however, in 40\% of cases over two weeks, the driver left before the scheduled departure time, as is demonstrated in \cref{fig:layover_observance}, and 31\% departed more than one minute early.


There are various ways to forecast arrival times, each with associated drawbacks and advantages. This section presents four arrival time prediction methods:
\begin{itemize}
\item \Fpf{} uses the particle filter estimate of vehicle state and the network state to obtain arrival time distributions;
\item \Fnorm{} uses a normal approximation of the vehicle's state, along with trip state, network state, and dwell-time distributions;
\item \Fhist{} uses only historical delay data; and
\item \Fsched{} uses the schedule and current real-time delay, which is currently used by \AT{}.
\end{itemize}


Methods \Fpf{} and \Fnorm{} use the network state, which is simplified for each route by including only the segments (in order) used by route $r$ with mean and variance
\begin{equation}
\label{eq:route_nw_state}
\RouteNWstate^r =
\begin{bmatrix}
\RouteNWstateseg^r_1 \\
\RouteNWstateseg^r_2 \\
\vdots \\
\RouteNWstateseg^r_L
\end{bmatrix}\quad\text{and}\quad
\RouteNWstatevar^r =
\begin{bmatrix}
\RouteNWstatevarseg^r_{1} & \RouteNWstatecor^r_{1,2} & \cdots &\RouteNWstatecor^r_{1,L} \\
\RouteNWstatecor^r_{2,1} & \RouteNWstatevarseg^r_{2} & \ddots &\RouteNWstatecor^r_{2,L} \\
\vdots & \ddots & \ddots & \vdots \\
\RouteNWstatecor^r_{L,1} & \RouteNWstatecor^r_{L,2} & \cdots & \RouteNWstatevarseg^r_L
\end{bmatrix},
\end{equation}
respectively, where $\RouteNWstateseg^r_{i}$ is the average vehicle speed along the $i$th segment of route $r$, $\RouteNWstatevarseg^r_{i}$ is the variance for that segment, and $\RouteNWstatecor^r_{i,j}$ is the covariance between segments $i$ and $j$. Note that the current implementation does not include segment covariances, but the above set-up demonstrates that the model can include them, if available. To simplify notation, I have dropped the $r$ superscript for the remainder of this chapter since only one route is considered at any one time.

For bus stop dwell times, we collected two weeks of data (as in \cref{sec:nw_par_est}) and calculated the mean and variance of dwell time for all stops along each trip. It is possible to determine if a bus \emph{did} stop (there is both an arrival and a departure time), but not that a bus \emph{did not} stop. For example, if the bus reports only an arrival time, it is unclear whether this is because the bus truly did not stop, or if it is a deficiency with the data. The bus may not have reported both observations, or, more likely, the polling interval (30 seconds) did not see both of them (Auckland Transport's real-time feed only includes the most recent observation). Therefore, stopping probabilities cannot be estimated from the data, so the same value of $\Prstop_j=0.5$ is used as in \cref{cha:vehicle_model}.


\subsection{Particle filter (\Fpf{})}
\label{sec:prediction_arrival_time_pf}

Each active trip is associated with a single vehicle, itself associated with a set of  $\Np$~particles approximating the vehicle's state. Perhaps the most straightforward method---at least conceptually---of predicting arrival times is to let each particle progress to the end of the route and record its arrival times. Computationally, this is easy to implement but very intensive, significantly increasing iteration time for the application. However, as we no longer need to worry about updating (via importance resampling), we can use a smaller subset of particles, $\Np^\star \leq \Np$, which can be adjusted depending on how many stops there are along the remainder of the route.

\begin{itemize}
\item better argument for using smaller $N$
\item rewrite paragraph below more succinctly (perhaps as a list)
\item for the dwell time stuff, refer back to Ch 3
\end{itemize}


To implement the particle filter forecast method, which we denote \Fpf{}, we take a (weighted) subsample of $\Np^\star$ particles at time $\Vtime_k$, $\tilde\Vstate_k$ (the time of the most recent vehicle observation). For each particle, we calculate the arrival time at each upcoming stop. First, if the particle is at a stop,  the ``remaining travel time'' for the current segment is needed, which we calculate by taking the particle's current speed, adding some noise, and extrapolating to the end of the segment. Having completed the current (partial) segment $j$, the particle's travel time to stop $j$, $\Linkt\vi_j$, is used to estimate the arrival time,
\begin{equation}
\label{eq:particle_arrival_time}
\Tarr\vi_j = \Vtime + \Linkt\vi_j.
\end{equation}
Next, we iteratively add dwell time and travel time for stops and road segments, respectively, until the particle reaches the end of the route; alternatively, we could stop after some number of stops, as we discuss in \cref{sec:prediction_performance}. The dwell time is assumed Gaussian with mean and variance estimated from historical data, $\mu_{\dwell_j}$ and $\sigma_{\dwell_j}$, respectively:
\begin{equation}
\label{eq:prediction_dwell_time}
\begin{split}
\Istop\vi_j &\sim \Bern{\Prstop_j} \\
\pserve\vi_j &\sim \TNormal{\mu_{\dwell_j}}{\sigma_{\dwell_j}}{0}{\infty} \\
\pdwell\vi_j &= \Istop\vi_j \pserve\vi_j
\end{split}
\end{equation}


\begin{itemize}
\item better clarification of terms here
\end{itemize}

When estimating travel time, the particle's speed along each segment is drawn from the network state,
\begin{equation}
\label{sec:particle_travel_time_pred}
\Linkt\vi_j \sim
\Normal{\hat\RouteNWstateseg_j}{
  (P_j + \Delta_j\NWnoise)^2)\wedge\mu_{P_j}
},
\end{equation}
which allows uncertainty to increase for more temporally distant forecasts, with a maximum set by the historical (prior) uncertainty.


Having obtained arrival times for all $\Np^\star$ particles, the predictive distribution for stop $j$---as in \cref{cha:vehicle_model}---is simply
\begin{equation}
\label{eq:particle_predictive_dist}
p(\Tarr_j | \Tripr_k, \RouteNWstate_\Tripr) \approx
\sum_{i=1}^{\Np^\star} \Pwt \dirac_{\Tarr_j\vi}\left(\Tarr_j\right)
= \frac{1}{\Np^\star}\sum_{i=1}^{\Np^\star} \dirac_{\Tarr_j\vi}\left(\Tarr_j\right)
\end{equation}
since all weights are equal. From the particle approximation of the distribution of arrival times we obtain point estimates or quantiles---for example, the mean, median, or a 90\% credible region. As discussed in \cref{app:particle-summaries}, computing quantiles (including the median) is a computationally demanding task for large numbers of particles due to the necessity to sort the particles in order from earliest to latest arrival times---at each stop.

\subsection{Normal approximation (\Fnorm{})}
\label{sec:prediction_arrival_time_normal}

Due to the computational demand of the particle filter, significant speed improvements can be obtained if we were to use a normal approximation instead. The network state is a multivariate normal random variable already; hence, the issue lies with stop dwell times having a point mass at zero, which leads to a mixture predictive distribution. For each stop the vehicle passes, there are twice as many components, so after $m$ stops, there would be $2^m$ components. Fortunately, these mostly converge after a few stops as shown in \cref{fig:normal_approx}.

\begin{knitrout}\small
\definecolor{shadecolor}{rgb}{0.969, 0.969, 0.969}\color{fgcolor}\begin{figure}

{\centering \subfloat[One intermediate stop\label{fig:normal_approx1}]{\includegraphics[width=.8\textwidth]{figure/normal_approx-1} }\newline
\subfloat[Two intermediate stops\label{fig:normal_approx2}]{\includegraphics[width=.8\textwidth]{figure/normal_approx-2} }\newline
\subfloat[Three intermediate stops\label{fig:normal_approx3}]{\includegraphics[width=.8\textwidth]{figure/normal_approx-3} }\newline
\subfloat[Eight intermediate stops\label{fig:normal_approx4}]{\includegraphics[width=.8\textwidth]{figure/normal_approx-4} }

}

\caption[Normal approximation for a series of stops ahead]{Normal approximation for a series of stops ahead. The true distribution is shown by the histogram with the underlying components (dashed curves). The red vertical lines represent the sample quantiles, while the dashed blue lines represent the approximated quantiles using the optimisation algorithm. Finally, the green curve and corresponding vertical lines represent the single normal approximation.}\label{fig:normal_approx}
\end{figure}


\end{knitrout}

We can use a mixture of normal distributions to approximate the arrival time distribition \citep{Wang_2012} by expressing the mean and uncertainty as vectors $\tilde\mu$ and $\tilde\sigma^2$, respectively, along with a third vector $\tilde\pi$ denoting the $\tilde N$ mixture weights\footnote{We are using the tilde over parameters, e.g., $\tilde x$, to help distinguish them from others used throughout the thesis}, such that
\begin{equation}
\label{eq:ch5:mixture_weight_spec}
\tilde\pi_i > 0, i = 1, \ldots, \tilde N
\text{ and } \sum_{i=1}^{\tilde N} \tilde\pi_i = 1.
\end{equation}
The arrival time at stop $j + n$ is given by
\begin{equation}
\label{eq:arrival_time_normal_approx}
\Tarr_{j+n} | \tilde\mu, \tilde\sigma^2, \tilde\pi, \RouteNWstate =
\sum_{\ell=j}^{j+n-1} \RouteNWstateseg_\ell +
\sum_{i=1}^{\tilde N} \tilde\pi_i z_i,\quad
z_i \sim \Normal{\tilde\mu_i}{\tilde\sigma^2_i}.
\end{equation}


Each component $i$ has an indicator of whether or not it stopped at stop $m$, $I_{im} = \{0,1\}$, giving the total dwell time as
\begin{equation}
\label{eq:mixture_dwell_times}
\begin{split}
\tilde\mu_i &= \sum_{m=j}^{j+n} I_{im} \dwell_m \\
\tilde\sigma_i^2 &= \sum_{m=j}^{j+n} I_{im} \dwellvar_m
\end{split}
\end{equation}
which assumes dwell times at individual stops are independent of each other. The vector of indicators is a branching tree for each stop; all current components get duplicated, and one is assigned $I_{i,m+1} = 1$ and the other 0.

\begin{itemize}
\item this needs to be rewritten, preferably more succinctly
\end{itemize}


Finally, mixture weights are obtained through the stopping probability at each stop, $\pi_j$:
\begin{equation}
\label{eq:ch5:mixture_weights}
\begin{split}
\tilde\pi_i &= \prod_{m=j}^{j+n} \tilde p_{im} \\
\tilde p_{im} &=
\begin{cases}
\pi_m & \text{if } I_{im} = 1 \\
1 - \pi_m & \text{otherwise.}
\end{cases}
\end{split}
\end{equation}


The mixture approximation works well for a few stops ahead, but after some time the mixture weights become small and the components combine, as shown in \cref{fig:normal_approx}. To prevent $\tilde N$ from becoming too large, the full distribution is simplified into a single component with mean and variance
\begin{equation}
\label{eq:mixture_mean}
\begin{split}
\E{\Tarr_m | \tilde\pi, \tilde\mu, \tilde\sigma^2, \RouteNWstate} &=
\E{\sum_{\ell=j}^{j+n-1} \RouteNWstateseg_\ell +
  \sum_{i=1}^{\tilde N} \tilde\pi_i z_i} \\
&= \sum_{\ell=j}^{j+n-1} \E{\RouteNWstateseg_\ell} +
  \sum_{i=1}^{\tilde N} \tilde\pi_i \E{z_i} \\
&= \sum_{\ell=j}^{j+n-1} \hat\RouteNWstateseg_\ell +
  \sum_{i=1}^{\tilde N} \tilde\pi_i \tilde\mu_i
\end{split}
\end{equation}
and
\begin{equation}
\label{eq:mixture_variance}
\begin{split}
\Var{\Tarr_m | \tilde\pi, \tilde\mu, \tilde\sigma^2, \RouteNWstate} &=
\Var{\sum_{\ell=j}^{j+n-1} \RouteNWstateseg_\ell +
  \sum_{i=1}^{\tilde N} \tilde\pi_i z_i} \\
&= \sum_{\ell=j}^{j+n-1} \Var{\RouteNWstateseg_\ell} +
  \sum_{i=1}^{\tilde N} \tilde\pi_i^2 \Var{z_i} \\
&= \sum_{\ell=j}^{j+n-1} \hat\RouteNWstatesegvar_\ell +
  \sum_{i=1}^{\tilde N} \tilde\pi_i \tilde\sigma_i^2
\end{split}
\end{equation}
respectively, assuming segment travel time and dwell time are independent---assuming otherwise makes this model impossible to work with; indeed, this model versus the particle filter (which makes no such assumption) is effectively testing the viability of this assumption.


For quantiles $q_\alpha$ we use an optimisation algorithm to solve
\begin{equation}
\label{eq:mixture_quadratic}
\left[
  p\left(\Tarr_m | \tilde\pi, \tilde\mu, \tilde\sigma^2, \RouteNWstate\right) - q_\alpha
\right]^2 = 0
\end{equation}
which is straightforward using Brent's Algorithm \citep{Brent_1971}, implemented in the Boost \textsf{C++} library.


\begin{itemize}
\item reword this \ldots
\end{itemize}

However, after eight stops, the number of components is $2^8 = 256$, after which point it would be more efficient to use the particle filter. We use a simplification criteria such that, when passed, a single new component is close enough in approximation to the set of components that it can replace them. We have chosen to compare quantiles, and if the maximum absolute difference is less than one minute\footnote{When providing \glspl{eta} to commuters, they are rounded to the nearest minute, anyway.}, we replace the set of components with a single new one with mean and variance as specified in \cref{eq:mixture_mean,eq:mixture_variance}. At subsequent stops, the process repeats, and we iteratively check if a single distribution is an adequate approximation.


We also need to prevent the mixture from becoming too large: if $\tilde N > 8$, we combine components with $\tilde\pi_i < \frac{1}{2}\max_i(\pi_i)$, using appropriately modified versions of \cref{eq:mixture_mean,eq:mixture_variance}. In practice, after about five stops the distribution is approximately normal, and the error is minimal.


%%%%%%%%%%%%%%%%%%%%%%%%%%%%%%%%%%%%%%%%%%%%%%%%%%%%%%%%% Historical data
\subsection[Historical arrival delays]{Historical arrival delays (\Fhist{})}
\label{eq:prediction_arrival_historical}

Another way to make predictions is to use historical data instead of real-time. We collected two weeks of data and recorded arrival time delays by stop and route to obtain a distribution which can then be used to make predictions. The arrival time at stop $j$ along route $r$ has an average delay of $\bar d_{rj}$ seconds with variance $D_{rj}^2$, so the predicted arrival time at stop $j$ on route $r$ with scheduled arrival time $S_{rj}$ is
\begin{equation}
\label{eq:arrival_pred_historical}
\hat\Tarr_{j} \sim \Normal{S_{rj} + \bar d_{rj}}{D_{rj}^2}.
\end{equation}


%%%%%%%%%%%%%%%%%%%%%%%%%%%%%%%%%%%%%%%%%%%%%%%%%%%%%%%%% Schedule + delay
\subsection[Schedule delays]{Schedule delays (\Fsched{})}
\label{eq:prediction_arrival_sched_delay}

The currently deployed prediction method uses the scheduled arrival time at stop $j$ along route $r$, $S_{rj}$, along with the arrival or departure time at the most recently visited stop, $T_{rm}$, giving a current delay, in seconds, of
\begin{equation}
\label{eq:sched_cur_delay}
d_{r} = T_{rm} - S_{rm}.
\end{equation}
The arrival time is then predicted as
\begin{equation}
\label{eq:sched_pred_arr}
\hat\Tarr_{rj} = S_{rj} + d_r.
\end{equation}
Note, however, that if a trip has not yet been observed, the default delay is $d_r = 0$. This happens if there is not a vehicle servicing it, the vehicle is running late, or the trip has been cancelled altogether.

\section{Comparing models}
\label{sec:prediction_model_comparison}



We now have four models to evaluate, two of which use the road network state, one uses historical data, and one only the current delay, the last being the method currently used by \gls{at}\footnote{And many other public transport providers using \gls{gtfs}-realtime}. To compare them, we implemented the first two in the application and, using a single day of data, run the program, storing all arrival time estimates, the uncertainty, as well as the 5\% and 90\% quantiles\footnote{Initially we used a symmetric 95\% prediction interval, but this was often highly skewed by a few very late particles.}. We could then use the actual (reported) arrival times to compute the accuracy of each estimator. For the historical data, we compare the mean with the actual arrival time. For the schedule-delay method, the arrival time for all upcoming stops was predicted after each new observation, and the prediction error computed.

The comparison criteria used is the \emph{\gls{rmse}}, as is commonly used for model checking \citep{cn}. \Gls{rmse} is the mean squared difference (in seconds) between the predicted and actual arrival times. Of course, since predictions change over time, we also compute \gls{rmse} by \emph{time until (actual) arrival}, allowing us to compare the models at different time points, as well as by stop sequence and time of day.


Another criteria we are interested in is the \emph{coverage of the prediction interval}, which is only available for the three methods for which we can construct an 85\% \gls{ppi}: for the particle filter, this is achieved by sorting the particles in order of arrival time, and then taking the $\lfloor q_\text{lower} N^\star\rfloor^{\text{th}}$ particle as the lower bound, and the $\lceil q_{\text{upper}} N^\star\rceil^{\text{th}}$ particle as the upper bound. For the 85\% interval, we used $q_\text{lower} = 0.05$ and $q_\text{upper} = 0.9$. For the normal approximation and historical arrival methods, the inverse \gls{cdf} gives us the required quantiles. The results are displayed in \cref{tab:model_results_rmse} and \cref{fig:model_results_rmse_time,fig:model_results_rmse_stopn,fig:model_results_rmse_timeofday}, which are described in \cref{sec:prediction_model_comp_stats}.


Additionally, we also want to compare the predictive power of the various forecast methods, namely \emph{the probability of arriving before the bus}, and hence not missing it, as well as the expected waiting time given a passenger arrives at the stop by a certain time. In table \cref{tab:model_results_pr_miss} and \cref{fig:model_results_pr_gtfs,fig:model_results_pr_time,fig:model_results_pr_stop,fig:model_results_pr_timeofday}, we use the point estimate (mean or median, depending on the forecast method), as well as the 5\% quantile, and for each calculate the probability that the bus arrives after the specified time. We also compute the expected waiting time for a passenger arriving at a certain time \emph{and the bus arrives after the specified time} (i.e., the passenger catches the bus). \Cref{sec:prediction_model_comp_probs} discusses these results.






\begin{knitrout}
\definecolor{shadecolor}{rgb}{0.969, 0.969, 0.969}\color{fgcolor}\begin{table}

\caption{\label{tab:model_results_rmse}Predictive model comparison of RMSE, in seconds, PPI coverage, and mean PPI width, in minutes, with (0.025, 0.975) quantiles.}
\centering
\begin{tabular}[t]{lrrrl}
\toprule
  & RMSE (s) & $\Pr{\Varr_m \in \text{PPI}}$ & \multicolumn{2}{c}{PPI width (m)} \\
\midrule
$\mathcal{F}_1$: Particle filter & 230 & 0.77 & 7 & (0.4, 19.8)\\
$\mathcal{F}_2$: Normal approximation & 485 & 0.91 & 30 & (0, 102.6)\\
$\mathcal{F}_3$: Historical delays & 243 & 0.84 & 11 & (2.4, 26.8)\\
$\mathcal{F}_4$: Schedule-delay & 236 &  &  & \\
\bottomrule
\end{tabular}
\end{table}


\end{knitrout}




\subsection{Predictive error and coverage}
\label{sec:prediction_model_comp_stats}

Results for a full day (5~am--midnight), shows little difference in the \gls{rmse} for methods $\FM_1$, $\FM_3$, and $\FM_4$, displayed in \cref{tab:model_results_rmse}; the normal approximation, $\FM_2$, is double the others. The particle filter's ($\FM_1$) \gls{rmse} is slightly lower than that of the schedule-delay method $\FM_4$, but not by much.

The \gls{ppi} coverage is the proportion of \gls{ppi}'s that contain the true arrival time, $\Pr{\Tarr_m \in (\hat\alpha_{\text{lower}},\hat\alpha_{\text{upper}})}$. In this measure, the normal approximation, $\FM_2$, has excessive coverage, while the historical method's coverage is close to the target coverage of 85\%. The particle filter has relatively poor coverage at under 80\%, indicating that not enough uncertainty is captured by the particles---\textcolor{red}{we may need to increase the number of particles}.

The mean width of the \gls{ppi} is the average difference between the upper and lower predictive quantiles. The interval width for $\FM_2$ is four times that of $\FM_1$, and the width of the 2.5\% and 97.5\% quantiles of the \glspl{ppi} is five times as large. The historical delays method give similar results to the particle filter, though slightly higher---this is most likely due to the interval not decreasing as the bus nears the stop, as is the case with all of the other methods.

From the simple, overall summary in \cref{tab:model_results_rmse}, it isn't very easy to comment on the relative performance of the methods, except that the normal approximation seems to be ill-suited to the task. However, this was not unexpected: throughout the previous chapters, we have commented on the non-Gaussian nature of the uncertainty around the state of transit vehicles. We now examine these summaries using several variables to explain variation: time until arrival, stop sequence, and time of day.


\subsubsection{Time until actual arrival}

Possibly the most obvious variable to explore is the \emph{time until actual arrival} at the stop: we would expect greater performance the nearer the vehicle is to arriving at a stop. We computed \gls{rmse} along with \gls{ppi} coverage and width in one-minute intervals, the results of which can be seen in \cref{fig:model_results_rmse_time}.


The \gls{rmse} (\cref{fig:model_results_rmse_time1}) for the four models shows that, up to 20~minutes before arrival, $\FM_1$ has the lowest prediction error. \Gls{rmse} for $\FM_4$ increases more rapidly at first, but this slows down, unlike $\FM_1$ which shows a rather linear relationship with time until arrival.  Note, however, that \gls{gtfs} defaults to a value of zero if no observations are available, so while the prediction error may be small, that may offen only be so because the bus arrived close to schedule. The historical model has fairly constant error, though slowly increasing: this is likely because early stops have a smaller average delay, and the time until arrival is never very large. That is, the second stop may take the bus two minutes to reach, so we would be unlikely to see an estimate for it 20~minutes before arrival since \glspl{eta} are---in the currently used application---only being estimated once the vehicle has begun the route and reported a trip update.


As for the \gls{ppi}, \cref{fig:model_results_rmse_time2,fig:model_results_rmse_time3}, the normal approximation has too greater coverage and quickly increasing width, while the historical method is again fairly constant in both its coverage (close to 85\%) and width, though again we see a slight increase in the latter the further out the bus gets, for the same reason as above. The particle filter again isn't capturing enough uncertainty, as its coverage is below the desired 85\%, though as a result the width is smaller than both of the other methods, particularly compared to $\FM_2$---significantly so once we get above 20~minutes  \textcolor{red}{... I'm not sure about this para}.


\begin{knitrout}
\definecolor{shadecolor}{rgb}{0.969, 0.969, 0.969}\color{fgcolor}\begin{figure}[!t]
\subfloat[\label{fig:model_results_rmse_time1}]{\includegraphics[width=\textwidth]{figure/model_results_rmse_time-1} }\newline
\subfloat[\label{fig:model_results_rmse_time2}]{\includegraphics[width=\textwidth]{figure/model_results_rmse_time-2} }\newline
\subfloat[\label{fig:model_results_rmse_time3}]{\includegraphics[width=\textwidth]{figure/model_results_rmse_time-3} }\caption[Model comparative statistics as a function of time until arrival]{Model comparative statistics as a function of time until arrival.}\label{fig:model_results_rmse_time}
\end{figure}


\end{knitrout}



\subsubsection{Stop sequence}

Similarly to time until arrival, we grouped results by stop sequence---one being the first stop along the route---and computed \gls{rmse}, \gls{ppi} coverage and \gls{ppi} width, displayed in \cref{fig:model_results_rmse_stopn}. The \gls{rmse} increases for all models up until stop 40, after which point only few routes have that many stops, so there is increased variability between consecutive stops. All models show approximately the same error, besides $\FM_2$ as before. Method $\FM_1$ has the smallest error for the first 30 stops, after which point $\FM_4$ performs slightly better, but again the stop-to-stop variability here is larger.



\begin{knitrout}
\definecolor{shadecolor}{rgb}{0.969, 0.969, 0.969}\color{fgcolor}\begin{figure}
\subfloat[\label{fig:model_results_rmse_stopn1}]{\includegraphics[width=\textwidth]{figure/model_results_rmse_stopn-1} }\newline
\subfloat[\label{fig:model_results_rmse_stopn2}]{\includegraphics[width=\textwidth]{figure/model_results_rmse_stopn-2} }\newline
\subfloat[\label{fig:model_results_rmse_stopn3}]{\includegraphics[width=\textwidth]{figure/model_results_rmse_stopn-3} }\caption[Model comparative statistics as a function of stop sequence]{Model comparative statistics as a function of stop sequence.}\label{fig:model_results_rmse_stopn}
\end{figure}


\end{knitrout}

The \gls{ppi} coverage and width show much the same trend as before. This is not unexpected, since there is a relationship between time until arrival and stop sequence: low sequence stops do not ever have a long time until arrival. Method $\FM_1$ has a shorter interval width, but again this comes with lower coverage. The quantiles of observed \gls{ppi} width in \cref{fig:model_results_rmse_stopn3} again show little significant different between $\FM_1$ and $\FM_3$, while $\FM_2$ is again overestimating uncertainty.



\subsubsection{Time of day}

We grouped observations into 15~minute intervals over the day and calculated the summary statistics for each forecasting method. The results, displayed in \cref{fig:model_results_rmse_timeofday}, differ quite significantly from those seen previously, as we see a strong peak-hour effect\footnote{In the morning there is a single peak at around 8~am, whereas in the evening there are two peaks: one for schools at about 3~pm, and another for workers at around 5~pm.}.

In \cref{fig:model_results_rmse_timeofday1}, the prediction error is smallest for the particle filter method at all times \emph{except} during the morning and evening peaks, with the afternoon school peak showing a more considerable increase than the worker peak. Method $\FM_4$ also shows a slight increase at peak times, but it is far less significant. Similarly, $\FM_3$ shows increases at peak times, as well as another in the afternoon.


The coverage of the \glspl{ppi} for both $\FM_1$ and $\FM_3$ are, during the day time off-peak period (9h30--14h30), both close to 85\% (\cref{fig:model_results_rmse_timeofday2}), and we see that most of the particle filter's poor coverage comes from outside of this period. It is worth noting here that, though discussed in \cref{sec:nw_hist_model}, the historical data-based forecasting model has not yet been implemented, so there may yet be improvements in the particle filter's performance if it was able to forecast peak speed effects.





\begin{knitrout}
\definecolor{shadecolor}{rgb}{0.969, 0.969, 0.969}\color{fgcolor}\begin{figure}
\subfloat[\label{fig:model_results_rmse_timeofday1}]{\includegraphics[width=\textwidth]{figure/model_results_rmse_timeofday-1} }\newline
\subfloat[\label{fig:model_results_rmse_timeofday2}]{\includegraphics[width=\textwidth]{figure/model_results_rmse_timeofday-2} }\newline
\subfloat[\label{fig:model_results_rmse_timeofday3}]{\includegraphics[width=\textwidth]{figure/model_results_rmse_timeofday-3} }\caption[Model comparative statistics as a function of time of day]{Model comparative statistics as a function of time of day.}\label{fig:model_results_rmse_timeofday}
\end{figure}


\end{knitrout}

As for the interval widths, these are reasonably constant over the day, with a small increase at peak times for $\FM_1$ and $\FM_3$. There is no noticable difference in the widths for $\FM_1$ and $\FM_3$; however, again $\FM_2$ overestimates the uncertainty, leading to prediction interval widths of up to 40~minutes \emph{on average}.



\subsection{Comparing arrival probabilities and wait times}
\label{sec:prediction_model_comp_probs}

\Gls{rmse} is a simple measure of predictive performance but takes no account of the application and the cost of an inaccurate prediction. In our case, there are two costs we need to consider:
\begin{enumerate}
\item wait time at the bus stop, and
\item missing the bus altogether.
\end{enumerate}
Clearly, the latter---in most cases---incurs a much greater cost, but depends entirely on time until the \emph{next} bus arrival: for high-frequency routes, this is not a problem; for low-frequency ones, however, it is\footnote{Some routes only run hourly!}. In this section, we do not differentiate between high and low-frequency routes.


The three statistics we compare across the methods are:
\begin{itemize}
\item $\mathbb{P}_m = \Pr{\Varr_m \geq \hat\Tarr_m}$, the probability that the vehicle arrives after the point estimate, $\hat\Tarr_m$;
\item $\mathbb{P}_\ell = \Pr{\Varr_m \geq \hat\alpha_{m,\text{lower}}}$, the probability that the vehicles arrives after the lower bound of the \gls{ppi}; and
\item $\mathbb{E}_\ell = \E{\Varr_m - \hat\alpha_{m,\text{lower}} | \Varr_m \geq \hat\alpha_{m,\text{lower}}}$, the expected waiting time for a passenger arriving at the lower predictive bound, given that the bus arrives after it.
\end{itemize}


For the overall results, shown in \cref{tab:model_results_pr_miss}, we see that, if a passenger at any time looks at the \gls{eta} of their bus \emph{once} and arrives at the stop by the indicated time, then by $\FM_1$ their probability of catching the bus is almost double that of using the currently deployed method, $\FM_4$. Using historical data averages, then (not surprisingly) $\mathbb{P}_m$ is about 50\%; $\FM_2$ is over 95\% which indicates that it underestimates arrival times.


As for the lower quantile ($\mathbb{P}_\ell$), the probabilities are approximately the same. Since these are 5\% quantiles, we would expect the bus to arrive after it about 95\% of the time. As for the expected wait time, given arrival by the lower quantile and that the bus arrives after it, the particle filter has the shortest, while (as with the previous results) $\FM_2$ is over four times as long.


Perhaps the more relevant comparison, however, is between $\FM_1$ and $\FM_4$; however, since there is no \gls{ppi} available (the schedule-delay is just a single point estimate), we need another way to compare them. \Cref{fig:model_results_pr_gtfs1} shows, as dashed lines, the values of $\mathbb{P}_\ell$ from \cref{tab:model_results_pr_miss}, overlaid with a curve of the probability of catching the bus (y-axis) given a passenger arrives $x$ minutes before the displayed arrival time. We see that one would need to arrive a little over 6~minutes before the stated \gls{eta} for the same probability of catching the bus as the particle filter. In \cref{fig:model_results_pr_gtfs2}, we show the expected wait time, $\mathbb{E}_\ell$, given the passenger arrives before the lower limit for the three other methods (dashed lines) and the curve for arriving $x$ minutes before the stated time using $\FM_4$. Given arrival 6~minutes before the stated arrival time, the expected waiting time is about 6~minutes, a couple of minutes longer than the particle filter. The inverse would be, for an equivalent expected waiting time for both $\FM_1$ and $\FM_4$, one would need to arrive four minutes before the $\FM_4$-forecasted arrival time (\cref{fig:model_results_pr_gtfs2}), which corresponds to an 80\% chance of arriving before the bus does (\cref{fig:model_results_pr_gtfs1}).


For the remainder of this section, we use 6~minutes before the specified \gls{eta} to obtain a lower bound for $\FM_4$, and compare the probabilities as a function of time until arrival, stop sequence, and time of day.



\begin{knitrout}
\definecolor{shadecolor}{rgb}{0.969, 0.969, 0.969}\color{fgcolor}\begin{table}

\caption{\label{tab:model_results_pr_miss}The probability of catching a bus given a passenger arrives by the mean/median ($\mathbb{P}_m$) and lower quantile ($\mathbb{P}_\ell$), along with the expected waiting time, in minutes, given arrival by the lower quantile, for each of the for forecast methods.}
\centering
\begin{tabular}[t]{lrrrl}
\toprule
  & $\mathbb{P}_m$ & $\mathbb{P}_\ell$ & $\mathbb{E}_\ell$ (m) & (95\% CI)\\
\midrule
$\mathcal{F}_1$: Particle filter & 0.64 & 0.93 & 4.45 & (0.2, 16.6)\\
$\mathcal{F}_2$: Normal approximation & 0.96 & 0.98 & 20.90 & (1.2, 63.7)\\
$\mathcal{F}_3$: Historical delays & 0.54 & 0.98 & 7.13 & (0.7, 20.8)\\
$\mathcal{F}_4$: Schedule-delay & 0.38 &  &  & \\
\bottomrule
\end{tabular}
\end{table}


\end{knitrout}

\begin{knitrout}
\definecolor{shadecolor}{rgb}{0.969, 0.969, 0.969}\color{fgcolor}\begin{figure}
\subfloat[\label{fig:model_results_pr_gtfs1}]{\includegraphics[width=\textwidth]{figure/model_results_pr_gtfs-1} }\newline
\subfloat[\label{fig:model_results_pr_gtfs2}]{\includegraphics[width=\textwidth]{figure/model_results_pr_gtfs-2} }\caption[GTFS equivalent]{GTFS equivalent}\label{fig:model_results_pr_gtfs}
\end{figure}


\end{knitrout}


\subsubsection{Time until arrival}

\Cref{fig:model_results_pr_time} shows $\mathbb{P}_m$, $\mathbb{P}_\ell$, and $\mathbb{E}_\ell$ computed in one-minute intervals for each of the four models. We see that the probability that the vehicle arrives after the estimate is reasonably constant, with a slight decrease as the bus nears the stop; however, the four methods are distinct from each other, yielding the same conclusion as before \textcolor{red}{(be more specific)}.

As for $\mathbb{P}_\ell$, displayed in \cref{fig:model_results_pr_time2}, we see a different trend: $\FM_1$ and $\FM_3$ increase the further out the vehicle is, while in $\FM_4$ the probability decreases. Clearly, this is because we have chosen a fixed lower bound. When the vehicle is less than 15 minutes out, arriving 6~minute before the schedule-delay estimate results in a higher probability of catching the bus than the particle filter because the width is greater.


Finally, we look at the expected waiting time, given a passenger arrives at the lower bound \emph{and} the bus arrives sometime after, which is displayed in \cref{fig:model_results_pr_time}. The particle filter has a consistently shorter waiting time, though by about 30~minutes out the three methods excluding the normal approximation are approximately the same. We can see that the expected waiting time for $\FM_4$ is more or less independent of time until arrival.



\begin{knitrout}
\definecolor{shadecolor}{rgb}{0.969, 0.969, 0.969}\color{fgcolor}\begin{figure}
\subfloat[One\label{fig:model_results_pr_time1}]{\includegraphics[width=\textwidth]{figure/model_results_pr_time-1} }\newline
\subfloat[Two\label{fig:model_results_pr_time2}]{\includegraphics[width=\textwidth]{figure/model_results_pr_time-2} }\newline
\subfloat[Three\label{fig:model_results_pr_time3}]{\includegraphics[width=\textwidth]{figure/model_results_pr_time-3} }\caption[Summary values by time until arrival]{Summary values by time until arrival.}\label{fig:model_results_pr_time}
\end{figure}


\end{knitrout}


\subsubsection{Stop sequence}

Next, we computed the same values by stop sequence, as is shown in \cref{fig:model_results_pr_stop}. For most stops, the particle filter has a higher probability of the bus arriving after the point estimate ($\mathbb{P}_m$), though for early stops this is not the case. However, this is simply because the particle filter \emph{does not make a prediction} until the vehicle has been observed\footnote{Yet! Future work will work to use historical data in this situation}, whereas the \gls{gtfs} approach assumes the delay is zero until the vehicle is observed. This assumption can be reasonably accurate, particularly for the first few stops when there has not been enough time for the bus to get behind or ahead of schedule.

The trend for $\mathbb{P}_\ell$ is similar to what we saw with time until arrival. This time, however, the particle filter has much lower probabilities for the first few stops, again due to the reasons stated above. Even so, $\FM_4$ is consistently higher than $\FM_1$ until stop 40. The expected waiting time again shows better performance with the particle filter method until about stop 40, after which point the methods are more or less the same, given not many routes have over 40~stops. There is a trade-off here between waiting time and catch probability.


\begin{knitrout}
\definecolor{shadecolor}{rgb}{0.969, 0.969, 0.969}\color{fgcolor}\begin{figure}
\subfloat[One\label{fig:model_results_pr_stop1}]{\includegraphics[width=\textwidth]{figure/model_results_pr_stop-1} }\newline
\subfloat[Two\label{fig:model_results_pr_stop2}]{\includegraphics[width=\textwidth]{figure/model_results_pr_stop-2} }\newline
\subfloat[Three\label{fig:model_results_pr_stop3}]{\includegraphics[width=\textwidth]{figure/model_results_pr_stop-3} }\caption[Summary values by stop sequence]{Summary values by stop sequence.}\label{fig:model_results_pr_stop}
\end{figure}


\end{knitrout}


\subsubsection{Time of day}

Calculating the probabilities and expected waiting time in 15~minute intervals for each of the models is shown in \cref{fig:model_results_pr_timeofday}, in which the peak hour effects are visible, associated with an increased probability of arriving before the bus does in the particle filter method for both the median and lower bound. The schedule-delay method is negatively affected by peak hour in the lower bound estimate. The historical data estimate shows little effect due to peak hour but performs better in the day time (versus early morning and late evening).


For the expected waiting time, the particle filter is overall the lowest, even during peak where it shows a significant increase, seen in \cref{fig:model_results_pr_timeofday3}. The historical and schedule-delay methods both have similar expected waiting times of about 6--7 minutes.


\begin{knitrout}
\definecolor{shadecolor}{rgb}{0.969, 0.969, 0.969}\color{fgcolor}\begin{figure}
\subfloat[One\label{fig:model_results_pr_timeofday1}]{\includegraphics[width=\textwidth]{figure/model_results_pr_timeofday-1} }\newline
\subfloat[Two\label{fig:model_results_pr_timeofday2}]{\includegraphics[width=\textwidth]{figure/model_results_pr_timeofday-2} }\newline
\subfloat[Three\label{fig:model_results_pr_timeofday3}]{\includegraphics[width=\textwidth]{figure/model_results_pr_timeofday-3} }\caption[Summary values by time of day]{Summary values by time of day.}\label{fig:model_results_pr_timeofday}
\end{figure}


\end{knitrout}



\subsection{Result summary}
\label{sec:prediction_model_comp_summary}

We have now seen that the particle filter suggests some improvement is possible over the schedule-delay method used by \AT. It tends to underestimate the arrival time, but even so, the expected waiting time is, in most situations, less than that of all of the other methods. Conversely, the normal approximation performed very poorly in this case, likely since summing uncertainties, and assuming independence, quickly results in an overestimation of uncertainty, and thus wide prediction intervals (not to mention high prediction error).

As for the other methods which do not use real-time network information, the historical data-based method performed as well as or better than the schedule-delay method, with a few situations where it did not perform so well. The average and variance of arrival delays were based on two weeks' worth of data, which is---at most---10 observations per trip. Processing more weeks of data could bring slight improvements, but without taking into account real-time data, it is unlikely to yield significant improvements. Post-processing is an intensive procedure, so future development could include real-time computation of the mean and variance of arrival time for each stop along each trip.

\section{Real-time performance}
\label{sec:prediction_performance}

The primary constraint on real-time applications is time, especially when the predictions will be out of date once the next vehicle observations are available 30~seconds after the first. We recorded the timings of each component of our application during the simulation described in \cref{sec:prediction_model_comparison}. \Cref{tab:prediction_timing} presents the mean timing results for each component using both \emph{wall clock} (the time passed as recorded by a clock on the wall), as well as the \emph{CPU clock} (the processing time). The latter represents the overall computational complexity of the problem. Since computations are spread out over multiple cores (in this simulation we used 3)  the actual wall clock is up to 33\% faster in the vehicle update, network update, and \gls{eta} prediction steps (we discussed multithreading in \cref{sec:rt-implementation}). In some steps (such as while loading data), the wall clock time is longer than the CPU time: this is because the step involved downloading data from the server, during which time the processor is mostly inactive while it waits for the file.



\begin{knitrout}\small
\definecolor{shadecolor}{rgb}{0.969, 0.969, 0.969}\color{fgcolor}\begin{table}

\caption{\label{tab:prediction_timing}Time taken, in milliseconds, during various parts of the program, running on a single core.}
\centering
\fontsize{8}{10}\selectfont
\begin{tabular}[t]{lrlrl}
\toprule
 & Wall clock & (SE) & CPU time & (SE)\\
\midrule
(L) Load data & 28.1 & (0.22) & 16.7 & (0.19)\\
(U) Update vehicle information & 3.47 & (0.04) & 3.32 & (0.034)\\
(V) Vehicle state update & 3930 & (37) & 10400 & (96)\\
(N) Network state update & 0.874 & (0.12) & 1.6 & (0.026)\\
(P) Predict ETAs & 1650 & (26) & 3490 & (39)\\
\addlinespace
(W) Write ETAs to protobuf feed & 327 & (1.5) & 341 & (1.1)\\
\midrule
(T) Total iteration time & 5940 & (55) & 14200 & (130)\\
\bottomrule
\end{tabular}
\end{table}


\end{knitrout}


From the results in \cref{tab:prediction_timing}, we see that the average iteration time is about 6~seconds, which is well below our original target of 30~seconds. The CPU time is about 14~seconds, so, on average, we see a 50\% reduction attributed to multithreading on three cores. The most computationally intensive step is the vehicle state update, which involves transitioning, reweighting, and occasionally resampling 5000~particles for each vehicle, which can exceed 1000 at peak time. Next is the arrival time prediction step, which also involves estimation with a subset of particles (in this simulation, we used 200). The last significant step is writing the \glspl{eta} to file, which additionally (in this case) involved computing the median arrival time for each vehicle and stop combination (\cref{app:particle-summaries}). The remaining steps each had average times of less than 0.1~seconds.



\begin{knitrout}\small
\definecolor{shadecolor}{rgb}{0.969, 0.969, 0.969}\color{fgcolor}\begin{figure}

{\centering \includegraphics[width=\textwidth]{figure/prediction_timing_time-1} 

}

\caption[Number of vehicles and trip updates at different times ofthe day, along with the timing results over time for various stages of the program]{Number of vehicles and trip updates at different times ofthe day, along with the timing results over time for various stages of the program: (L) Load data, (O) Update vehicles information, (V) Vehicle state update, (N) Network state update, (P) Predict ETAs, (W) Write ETAs to protobuf feed, (T) Total iteration time.}\label{fig:prediction_timing_time}
\end{figure}


\end{knitrout}



Since the number of vehicles changes throughout the day, as displayed in the top half of \cref{fig:prediction_timing_time}, we also display the timings over time, as shown in the lower half of the figure. We see the two peak periods: morning peak, which includes both school and workers at the same time, and the broader evening peak since schools finish at about 3~pm while most workers finish about 5~pm. When the number of vehicles exceeds 1100, the total iteration time is around 9~seconds and remains at around 6~seconds during daytime off-peak. Again, this is the wall clock iteration time, so deployment on a server with more cores would result in even faster iterations.

