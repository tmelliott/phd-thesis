\section{Predicting arrival time}
\label{sec:prediction_arrival_time}




\begin{knitrout}
\definecolor{shadecolor}{rgb}{0.969, 0.969, 0.969}\color{fgcolor}\begin{figure}

{\centering \includegraphics[width=\linewidth]{figure/layover_observance-1} 

}

\caption[Vehicle delays at layover stops truncated to 10~minutes early and 20~minutes late]{Vehicle delays at layover stops truncated to 10~minutes early and 20~minutes late. Negative values indicate non-adherance by drivers, which makes up 40\% of cases.}\label{fig:layover_observance}
\end{figure}


\end{knitrout}

We now have estimates of the state of the trip (and any associated vehicle) and the road network. These are combined to forecast the time taken for the vehicle to reach all remaining stops along the route. Dwell times can also affect arrival time uncertainty, particularly at major stops. Additionally, the bus may have a layover at a stop---these are common at major points of interest, such as at a mall, or stations where many routes connect so passengers can transfer between services. We detect layovers in the GTFS \emph{stop times} table by finding departure times which are later than the associated arrival time (usually one second, as shown in figure X). Drivers should wait at these stops until the scheduled departure time before continuing the trip; however, in 40\% of cases we looked at over two weeks, the driver left before the scheduled departure time, as is demonstrated in \cref{fig:layover_observance}.

There are various ways we could forecast arrival times, each with its drawbacks and advantages. In this section, we present four forecasting methods. The first uses the particle filter estimate of vehicle state and the network state to obtain arrival time distributions. The second uses a normal approximation of the vehicle's state, along with the network state and dwell-time distributions. The third uses only historical delay data, and the last is the method currently used by \gls{at}, which uses the schedule and current real-time delay.

\textcolor{red}{Move this para earlier?}
For the first two methods, the current network state is subset to include only the segments for the particular route, such that
\begin{equation}
\label{eq:route_nw_state}
\RouteNWstate =
\begin{bmatrix}
\RouteNWstateseg_{1} & \RouteNWstatecor_{1,2} & \cdots \\
\RouteNWstatecor_{2,1} & \RouteNWstateseg_{2} & \cdots \\
\vdots & \ddots & \ddots
\end{bmatrix}
\end{equation}
where $\RouteNWstateseg_{i}, i = 1, \ldots, \Nseg$ is the average vehicle speed along the $i$th segment, and $\RouteNWstatecor_{i,j}$ is the covariance between segments $i$ and $j$. Note that in the current framework, we do not calculate segment covariances, but the above setup demonstrates that the model can include covariances, if available.

For bus stops dwell times, we collected two weeks of data (as was used in \cref{sec:nw_par_est}) and calculated the mean and variance of dwell time for all stops along each trip. We can determine if a bus \emph{did} stop (there are an arrival and departure time), but not that a bus \emph{did not} stop. If we see only an arrival or a departure time (or neither), it is impossible to differentiate between the bus truly not stoping, or deficiencies with the data. The bus may not have reported both observations, or, more likely, our polling interval (30 seconds) did not see both of them, since the real-time feed only includes the most recent observation. Therefore, we cannot calculate stopping probabilities from the data, so the same value as was used in \cref{cha:vehicle_model} is used for $\Prstop_j$.


\subsection{Particle filter (\Fpf{})}
\label{sec:prediction_arrival_time_pf}

Each active trip is associated with a single vehicle, itself associated with a set of  $\Np$~particles approximating the vehicle's state. Perhaps the most straightforward method---at least conceptually---of predicting arrival times is to let each particle progress to the end of the route and record its arrival times. Computationally, this is easy to implement but very intensive, significantly increasing iteration time for the application. However, as we no longer need to worry about updating (via importance resampling), we can use a smaller subset of particles, $\Np^\star \leq \Np$, which can be adjusted depending on how many stops there are along the remainder of the route.

\begin{itemize}
\item better argument for using smaller $N$
\item rewrite paragraph below more succinctly (perhaps as a list)
\item for the dwell time stuff, refer back to Ch 3
\end{itemize}


To implement the particle filter forecast method, which we denote \Fpf{}, we take a (weighted) subsample of $\Np^\star$ particles at time $\Vtime_k$, $\tilde\Vstate_k$ (the time of the most recent vehicle observation). For each particle, we calculate the arrival time at each upcoming stop. First, if the particle is at a stop,  the ``remaining travel time'' for the current segment is needed, which we calculate by taking the particle's current speed, adding some noise, and extrapolating to the end of the segment. Having completed the current (partial) segment $j$, the particle's travel time to stop $j$, $\Linkt\vi_j$, is used to estimate the arrival time,
\begin{equation}
\label{eq:particle_arrival_time}
\Tarr\vi_j = \Vtime + \Linkt\vi_j.
\end{equation}
Next, we iteratively add dwell time and travel time for stops and road segments, respectively, until the particle reaches the end of the route; alternatively, we could stop after some number of stops, as we discuss in \cref{sec:prediction_performance}. The dwell time is assumed Gaussian with mean and variance estimated from historical data, $\mu_{\dwell_j}$ and $\sigma_{\dwell_j}$, respectively:
\begin{equation}
\label{eq:prediction_dwell_time}
\begin{split}
\Istop\vi_j &\sim \Bern{\Prstop_j} \\
\pserve\vi_j &\sim \TNormal{\mu_{\dwell_j}}{\sigma_{\dwell_j}}{0}{\infty} \\
\pdwell\vi_j &= \Istop\vi_j \pserve\vi_j
\end{split}
\end{equation}


\begin{itemize}
\item better clarification of terms here
\end{itemize}

When estimating travel time, the particle's speed along each segment is drawn from the network state,
\begin{equation}
\label{sec:particle_travel_time_pred}
\Linkt\vi_j \sim
\Normal{\hat\RouteNWstateseg_j}{
  (P_j + \Delta_j\NWnoise)^2)\wedge\mu_{P_j}
},
\end{equation}
which allows uncertainty to increase for more temporally distant forecasts, with a maximum set by the historical (prior) uncertainty.


Having obtained arrival times for all $\Np^\star$ particles, the predictive distribution for stop $j$---as in \cref{cha:vehicle_model}---is simply
\begin{equation}
\label{eq:particle_predictive_dist}
p(\Tarr_j | \Tripr_k, \RouteNWstate_\Tripr) \approx
\sum_{i=1}^{\Np^\star} \Pwt \dirac_{\Tarr_j\vi}\left(\Tarr_j\right)
= \frac{1}{\Np^\star}\sum_{i=1}^{\Np^\star} \dirac_{\Tarr_j\vi}\left(\Tarr_j\right)
\end{equation}
since all weights are equal. From the particle approximation of the distribution of arrival times we obtain point estimates or quantiles---for example, the mean, median, or a 90\% credible region. As discussed in \cref{app:particle-summaries}, computing quantiles (including the median) is a computationally demanding task for large numbers of particles due to the necessity to sort the particles in order from earliest to latest arrival times---at each stop.

\subsection{Normal approximation (\Fnorm{})}
\label{sec:prediction_arrival_time_normal}

Due to the computational demand of the particle filter, significant speed improvements can be obtained if we were to use a normal approximation instead. The network state is a multivariate normal random variable already; hence, the issue lies with stop dwell times having a point mass at zero, which leads to a mixture predictive distribution. For each stop the vehicle passes, there are twice as many components, so after $m$ stops, there would be $2^m$ components. Fortunately, these mostly converge after a few stops as shown in \cref{fig:normal_approx}.

\begin{knitrout}\small
\definecolor{shadecolor}{rgb}{0.969, 0.969, 0.969}\color{fgcolor}\begin{figure}

{\centering \subfloat[One intermediate stop\label{fig:normal_approx1}]{\includegraphics[width=.8\textwidth]{figure/normal_approx-1} }\newline
\subfloat[Two intermediate stops\label{fig:normal_approx2}]{\includegraphics[width=.8\textwidth]{figure/normal_approx-2} }\newline
\subfloat[Three intermediate stops\label{fig:normal_approx3}]{\includegraphics[width=.8\textwidth]{figure/normal_approx-3} }\newline
\subfloat[Eight intermediate stops\label{fig:normal_approx4}]{\includegraphics[width=.8\textwidth]{figure/normal_approx-4} }

}

\caption[Normal approximation for a series of stops ahead]{Normal approximation for a series of stops ahead. The true distribution is shown by the histogram with the underlying components (dashed curves). The red vertical lines represent the sample quantiles, while the dashed blue lines represent the approximated quantiles using the optimisation algorithm. Finally, the green curve and corresponding vertical lines represent the single normal approximation.}\label{fig:normal_approx}
\end{figure}


\end{knitrout}

We can use a mixture of normal distributions to approximate the arrival time distribition \citep{Wang_2012} by expressing the mean and uncertainty as vectors $\tilde\mu$ and $\tilde\sigma^2$, respectively, along with a third vector $\tilde\pi$ denoting the $\tilde N$ mixture weights\footnote{We are using the tilde over parameters, e.g., $\tilde x$, to help distinguish them from others used throughout the thesis}, such that
\begin{equation}
\label{eq:ch5:mixture_weight_spec}
\tilde\pi_i > 0, i = 1, \ldots, \tilde N
\text{ and } \sum_{i=1}^{\tilde N} \tilde\pi_i = 1.
\end{equation}
The arrival time at stop $j + n$ is given by
\begin{equation}
\label{eq:arrival_time_normal_approx}
\Tarr_{j+n} | \tilde\mu, \tilde\sigma^2, \tilde\pi, \RouteNWstate =
\sum_{\ell=j}^{j+n-1} \RouteNWstateseg_\ell +
\sum_{i=1}^{\tilde N} \tilde\pi_i z_i,\quad
z_i \sim \Normal{\tilde\mu_i}{\tilde\sigma^2_i}.
\end{equation}


Each component $i$ has an indicator of whether or not it stopped at stop $m$, $I_{im} = \{0,1\}$, giving the total dwell time as
\begin{equation}
\label{eq:mixture_dwell_times}
\begin{split}
\tilde\mu_i &= \sum_{m=j}^{j+n} I_{im} \dwell_m \\
\tilde\sigma_i^2 &= \sum_{m=j}^{j+n} I_{im} \dwellvar_m
\end{split}
\end{equation}
which assumes dwell times at individual stops are independent of each other. The vector of indicators is a branching tree for each stop; all current components get duplicated, and one is assigned $I_{i,m+1} = 1$ and the other 0.

\begin{itemize}
\item this needs to be rewritten, preferably more succinctly
\end{itemize}


Finally, mixture weights are obtained through the stopping probability at each stop, $\pi_j$:
\begin{equation}
\label{eq:ch5:mixture_weights}
\begin{split}
\tilde\pi_i &= \prod_{m=j}^{j+n} \tilde p_{im} \\
\tilde p_{im} &=
\begin{cases}
\pi_m & \text{if } I_{im} = 1 \\
1 - \pi_m & \text{otherwise.}
\end{cases}
\end{split}
\end{equation}


The mixture approximation works well for a few stops ahead, but after some time the mixture weights become small and the components combine, as shown in \cref{fig:normal_approx}. To prevent $\tilde N$ from becoming too large, the full distribution is simplified into a single component with mean and variance
\begin{equation}
\label{eq:mixture_mean}
\begin{split}
\E{\Tarr_m | \tilde\pi, \tilde\mu, \tilde\sigma^2, \RouteNWstate} &=
\E{\sum_{\ell=j}^{j+n-1} \RouteNWstateseg_\ell +
  \sum_{i=1}^{\tilde N} \tilde\pi_i z_i} \\
&= \sum_{\ell=j}^{j+n-1} \E{\RouteNWstateseg_\ell} +
  \sum_{i=1}^{\tilde N} \tilde\pi_i \E{z_i} \\
&= \sum_{\ell=j}^{j+n-1} \hat\RouteNWstateseg_\ell +
  \sum_{i=1}^{\tilde N} \tilde\pi_i \tilde\mu_i
\end{split}
\end{equation}
and
\begin{equation}
\label{eq:mixture_variance}
\begin{split}
\Var{\Tarr_m | \tilde\pi, \tilde\mu, \tilde\sigma^2, \RouteNWstate} &=
\Var{\sum_{\ell=j}^{j+n-1} \RouteNWstateseg_\ell +
  \sum_{i=1}^{\tilde N} \tilde\pi_i z_i} \\
&= \sum_{\ell=j}^{j+n-1} \Var{\RouteNWstateseg_\ell} +
  \sum_{i=1}^{\tilde N} \tilde\pi_i^2 \Var{z_i} \\
&= \sum_{\ell=j}^{j+n-1} \hat\RouteNWstatesegvar_\ell +
  \sum_{i=1}^{\tilde N} \tilde\pi_i \tilde\sigma_i^2
\end{split}
\end{equation}
respectively, assuming segment travel time and dwell time are independent---assuming otherwise makes this model impossible to work with; indeed, this model versus the particle filter (which makes no such assumption) is effectively testing the viability of this assumption.


For quantiles $q_\alpha$ we use an optimisation algorithm to solve
\begin{equation}
\label{eq:mixture_quadratic}
\left[
  p\left(\Tarr_m | \tilde\pi, \tilde\mu, \tilde\sigma^2, \RouteNWstate\right) - q_\alpha
\right]^2 = 0
\end{equation}
which is straightforward using Brent's Algorithm \citep{Brent_1971}, implemented in the Boost \textsf{C++} library.


\begin{itemize}
\item reword this \ldots
\end{itemize}

However, after eight stops, the number of components is $2^8 = 256$, after which point it would be more efficient to use the particle filter. We use a simplification criteria such that, when passed, a single new component is close enough in approximation to the set of components that it can replace them. We have chosen to compare quantiles, and if the maximum absolute difference is less than one minute\footnote{When providing \glspl{eta} to commuters, they are rounded to the nearest minute, anyway.}, we replace the set of components with a single new one with mean and variance as specified in \cref{eq:mixture_mean,eq:mixture_variance}. At subsequent stops, the process repeats, and we iteratively check if a single distribution is an adequate approximation.


We also need to prevent the mixture from becoming too large: if $\tilde N > 8$, we combine components with $\tilde\pi_i < \frac{1}{2}\max_i(\pi_i)$, using appropriately modified versions of \cref{eq:mixture_mean,eq:mixture_variance}. In practice, after about five stops the distribution is approximately normal, and the error is minimal.


%%%%%%%%%%%%%%%%%%%%%%%%%%%%%%%%%%%%%%%%%%%%%%%%%%%%%%%%% Historical data
\subsection{Historical arrival delays}
\label{eq:prediction_arrival_historical}

Another way to make predictions is to use historical instead of real-time data. We collected two weeks of data and recorded arrival times by stop and trip to obtain a distribution which can then be used to make predictions. The arrival time at stop $j$ along trip $\Tripr$ has mean $\hat\alpha_{rj}$ and variance $\hat\zeta_{rj}^2$, so the predicted arrival time is simply
\begin{equation}
\label{eq:arrival_pred_historical}
\Tarr_{\Tripr j} \sim \Normal{\hat\alpha_{\Tripr j}}{\hat\zeta_{\Tripr j}^2}.
\end{equation}


%%%%%%%%%%%%%%%%%%%%%%%%%%%%%%%%%%%%%%%%%%%%%%%%%%%%%%%%% Schedule + delay
\subsection{Schedule delays}
\label{eq:prediction_arrival_sched_delay}

The currently deployed prediction method uses the scheduled arrival time at stop $j$ along route $r$, $S_{rj}$, along with the arrival or departure time at the most recently visited stop, $T_{rm}$, giving a current delay, in seconds, of
\begin{equation}
\label{eq:sched_cur_delay}
\bar\delta_{r} = T_{rm} - S_{rm}.
\end{equation}
Then, the predicted arrival time is simply
\begin{equation}
\label{eq:sched_pred_arr}
\hat\Tarr_{rj} = S_{rj} + \bar\delta_r.
\end{equation}
Note, however, that if a given trip has no associated observations, the default delay is $\bar\delta_r = 0$, regardless of whether or not there is a vehicle servicing it, the vehicle is running late, or the trip has been cancelled altogether.
