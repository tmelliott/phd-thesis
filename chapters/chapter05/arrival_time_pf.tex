\subsection{Particle filter}
\label{sec:prediction_arrival_time_pf}

Each active trip is associated with a single vehicle, itself associated with a set of  $\Np$~particles approximating the vehicle's state. Perhaps the most straightforward method---at least conceptually---of predicting arrival times is to let each particle progress to the end of the route and record its arrival times. Computationally, this is easy to implement but very intensive, significantly increasing iteration time for our whole algorithm. However, as we no longer need to worry about updating (via importance resampling), we can use a smaller subset of particles, $\Np^\star \leq \Np$, which can be adjusted depending on how many stops there are along the remainder of the route.


To implement the particle filter forecast method, $\mathcal{F}_1$, we start with a set of particles at time $\Vtime_k$, $\tilde\Vstate_k$, which was the time of the most recent vehicle observation. For each particle, we calculate the arrival time at each upcoming stop. First, if the particle is at a stop,  the ``remaining travel time'' for the current segment is needed, which we calculate by taking the particle's current speed, adding some noise, and extrapolating to the end of the segment. Having completed the current (partial) segment $j$, the particle's travel time to stop $j$, $\Linkt\vi_j$, is used to estimate the arrival time,
\begin{equation}
\label{eq:particle_arrival_time}
\Tarr\vi_j = \Vtime + \Linkt\vi_j.
\end{equation}
Next, we iteratively add dwell time and travel time for stops and road segments, respectively, until the particle reaches the end of the route. The dwell time is assumed Gaussian with mean and variance estimated from historical data, $\mu_{\dwell_j}$ and $\sigma_{\dwell_j}$, respectively:
\begin{equation}
\label{eq:prediction_dwell_time}
\begin{split}
\Istop\vi_j &\sim \Bern{\Prstop_j} \\
\pserve\vi_j &\sim \TNormal{\mu_{\dwell_j}}{\sigma_{\dwell_j}}{0}{\infty} \\
\pdwell\vi_j &= \Istop\vi_j \pserve\vi_j
\end{split}
\end{equation}


When estimating travel time, the particle's segment speed is drawn from the network state,
\begin{equation}
\label{sec:particle_travel_time_pred}
\Linkt\vi_j \sim
\Normal{\hat\RouteNWstateseg_j}{
  (P_j + \Delta_j\NWnoise)^2)\wedge\mu_{P_j}
},
\end{equation}
which allows uncertainty to increase for more temporally distant forecasts, with a maximum set by the historical (prior) uncertainty.


Having obtained arrival times for all $\Np^\star$ particles, the predictive distribution for stop $j$---as in \cref{cha:vehicle_model}---is simply
\begin{equation}
\label{eq:particle_predictive_dist}
p(\Tarr_j | \Tripr_k, \RouteNWstate_\Tripr) \approx
\sum_{i=1}^{\Np^\star} \dirac_{\Tarr_j\vi}\left(\Tarr_j\right)
\end{equation}
from which we can obtain point estimates or quantiles---for example, the mean, median, or a 90\% credible region. Note that we no longer use particle weights, as these were used to obtain the subsample of size $\Np^\star$.
