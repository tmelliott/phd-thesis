
\chapter{Arrival time prediction for journey planning}
\label{cha:etas}
Arrival time prediction is most often used to provide commuters with a countdown while they wait for their bus, and has been shown to reduce experienced wait time \citep{TCRP_2003}. However, research has also shown that passengers who use \gls{rti} have shortened actual wait times \citep{Lu_2017}, demonstrating that, if reliable, passengers can use \gls{rti} to better plan their commutes. In this chapter, we shift our focus from the probabilistic view of arrival time estimation in \cref{cha:prediction} to a practical one.


Accurately predicting the arrival time of a transit vehicle is a decidedly difficult problem. There are several known sources of variability---traffic lights and intermediate bus stops, for example---but also countless others which we cannot model. \citet{Mazloumi_2011} proposed a method of finding prediction intervals of arrival time using neural networks. More recently, \citet{Fernandes_2018} explored methods for displaying uncertainty, which included using quantile dot plots, to provide travellers with a means of making informed decisions. We examine the use of a single point estimate in an attempt to balance reliability and accuracy and compare this with the use of an interval estimate to convey uncertainty.


Another desirable aspect of \gls{rti} is journey planning: given an origin and a destination, and optionally one or more constraints, what is the optimal route (or routes for journeys requiring multiple \emph{legs}). \Citet{Berczi_2017} developed an approach to journey planning that uses a \emph{probabilistic model} of arrival times to decide between competing options. Since the initial selection of candidate options is a complicated topic (see \citet{Hame_2013a,Hame_2013b,Zheng_2016} for methods of doing so) we do not consider this problem here. Instead, we use the results from \cref{cha:prediction} to decide between pre-selected options.


Both the arrival time estimates (point and interval) and journey planning decisions use the arrival time distributions estimated in chapter 5. First, however, we need to express the distribution of arrival times in a form that is easy to work with and distribute. The first consideration is that \glspl{eta} are typically displayed in minutes as integers, so any estimates first need to be rounded. Take, for example, an \gls{eta} of 105~seconds, or 1.75~minutes, which we may round up to 2~minutes, or down to 1~minute. We use Bayesian posterior predictive probabilities to estimate the median arrival time $a_{0.5}$ such that $\Pr{A \geq a_{0.5}} = 0.5$. If we round 1.75 in the above example to $\hat a_{0.5} = 2$~minutes, then the equality is incorrect: the probability that the actual arrival time $A$ is later than the estimated arrival time $\hat a_{0.5}$ is, in fact, \emph{less} than 50\%. However, if we round \emph{down} to $\hat a_{0.5} = 1$~minute, then, although the equality remains invalid, the probability that the bus arrives after $\hat a_{0.5}$ is \emph{greater than} 50\%.


To compute the \gls{cdf} of integer-valued arrival times, we first convert each particle's arrival time $\Tarr\vi$ to minutes and round down, which is the equivalent of taking its modulus with 60:
\begin{equation}
\label{eq:particle_eta_modulus}
\breve\Tarr\vi = \Tarr\vi \mod 60.
\end{equation}
Next, we tally the number of particles with each arrival time $a$ and calculate the probability of the bus arriving in the interval $[a a+1)$,
\begin{equation}
\label{eq:pf_pdf_arrivaltime}
\Pr{A \in [a, a+1)} =
\frac{1}{N^\star} \sum_{i=1}^{N^\star} I_{\breve\Tarr\vi = a},
\end{equation}
where $I_{a=b}$, the indicator, is 1 if $a$ equals $b$, and zero otherwise. Computationally, this is significantly easier than computing quantiles, since the latter involves sorting the particles (see \cref{app:particle-summaries}), whereas tallying integers can be performed with a single, unsorted pass over the particle vector. This provides significant performance enhancements, remembering that quantiles require sorting of particles at every stop.


\Cref{eq:pf_pdf_arrivaltime} lets us express the \gls{cdf} as
\begin{equation}
\label{eq:pf_cdf_arrivaltime}
\Pr{A < a} = \sum_{x=0}^{x=a-1} \Pr{A \in [x, x+1)}.
\end{equation}
\Cref{fig:eta_cdf} displays the \gls{cdf} for a single stop. The remainder of this chapter explores the reliability and usefulness of various summary statistics to convey this distribution to commuters.

\begin{knitrout}\small
\definecolor{shadecolor}{rgb}{0.969, 0.969, 0.969}\color{fgcolor}\begin{figure}

{\centering \includegraphics[width=.6\textwidth]{figure/eta_cdf-1} 

}

\caption[CDF of arrival time]{CDF of arrival time.}\label{fig:eta_cdf}
\end{figure}


\end{knitrout}



\section{Estimates of arrival time}
\label{sec:eta_estimates}

As previously discussed, there is considerable uncertainty in the posterior predictive distribution of arrival time. This uncertainty makes point estimates challenging to provide since there is a strong negative correlation between \emph{accuracy}---how far the estimates are from the actual value---and \emph{reliability}---how \emph{useful} the estimate is. Take for example the median, which has, on average, the highest accuracy, but 50\% of the time the bus arrives earlier than the estimate, which could lead to a passenger missing their bus. To overcome this, we also explore using \emph{prediction intervals} to balance accuracy and reliability.

The method for obtaining estimates from the \glspl{cdf} from \cref{sec:etas_cdfs} uses quantiles, such that the $q$-quantile, $q\in(0,1)$, is found by
\begin{equation}
\label{eq:eta_calc_quantile}
\max\{a : \Pr{A < a} \geq q\}
\end{equation}
\Cref{fig:eta_calc_quantile} shows a single \gls{cdf} with a horizontal line at $q$, and the largest ETA with its quantile below this line is the desired value; in this case, $\hat A_{0.5} = 11$.


\begin{knitrout}
\definecolor{shadecolor}{rgb}{0.969, 0.969, 0.969}\color{fgcolor}\begin{figure}

{\centering \includegraphics[width=.8\textwidth]{figure/eta_calc_quantile-1} 

}

\caption[Quantile estimation from a CDF]{Quantile estimation from a CDF. ETA values with quantiles below the desired threshold are coloured red; the maximum of these is the quantile value.}\label{fig:eta_calc_quantile}
\end{figure}


\end{knitrout}


\subsection{Point estimate}
\label{sec:etas-point}

In a perfect world, it would be possible to predict precisely how long it would take a bus to arrive at a stop, and display this single number to passengers. Alas, as we saw in \cref{cha:prediction}, this is not a perfect world, and arrival time prediction inherently comes with significant uncertainty. Yet we still need to decide on the best single value to use as a point estimate of arrival time, since much of the infrastructure currently available only allows this. We also want to examine if it is possible to come up with a single statistic that performs well---on average---and, more importantly, better than the currently deployed method.





\begin{knitrout}\small
\definecolor{shadecolor}{rgb}{0.969, 0.969, 0.969}\color{fgcolor}\begin{figure}

{\centering \includegraphics[width=\textwidth]{figure/eta_overall_results-1} 

}

\caption[ETA stuff]{ETA stuff}\label{fig:eta_overall_results}
\end{figure}


\end{knitrout}


We calculated, for all stops, trips, and times, a range of arrival times quantiles ($q \in \{0.05, 0.25, 0.5, 0.6, 0.6, 0.75, 0.9\}$), and for each computed the \gls{rmse}, the probability of catching the bus given you arrive at the stated \gls{eta}, $\Pcatch$, and the expected waiting time given you catch the bus, $\Ewait$. The results are displayed in \cref{fig:eta_overall_results}, along with the same values computed used the \gls{gtfs} arrival time estimates. For the particle filter, the lowest \gls{rmse} value is achieved when using the 60\% quantile, which obtains a 66\% probability of the bus arriving after the \gls{eta} and an expected waiting time of 2.7~minutes. The coverage of the quantiles are being overestimated---we would expect the 60\% quantile to have an approximately 40\% success rate---which indicates that the current implementation of the particle filter is underestimating arrival times.


\begin{knitrout}\small
\definecolor{shadecolor}{rgb}{0.969, 0.969, 0.969}\color{fgcolor}\begin{figure}

{\centering \includegraphics[width=\textwidth]{figure/eta_headway_results-1} 

}

\caption[More ETA stuff]{More ETA stuff}\label{fig:eta_headway_results}
\end{figure}


\end{knitrout}

An important consideration is the cost of missing the bus. For each trip, we computed the scheduled time between the current trip and the subsequent one (which is termed \emph{headway}) to compute the expected waiting time if the bus arrives \emph{before} the predicted time. That is, if a bus is predicted to arrive in 5~minutes, but actually arrives in 3, and the time until the next bus (servicing the same trip) is 10~minutes, then the expected waiting time will be $10-5=5$~minutes (assuming the passenger arrives in 5~minutes). This \emph{headway} is an essential component to prediction cost.

The total expected wait time can be conditioned on whether or not the bus was caught,
\begin{equation}
\label{eq:eta_wait_conditional}
\begin{split}
\E{\text{wait}} &=
  \Pr{\text{catch}} \E{\text{wait}|\text{catch}} +
  (1 - \Pr{\text{catch}}) \E{\text{wait}|\text{miss}} \\
  &= \Pr{A \geq a} \E{A - a | A \geq a} +
  \Pr{A < a}\E{\text{headway} - a + A | A < a}
\end{split}
\end{equation}
where $a$ is the estimated arrival time, and $A$ is the actual. For simplicity, we assume headway is maintained (which it is not, \cite{})---that is, if a bus has 20~minute frequency and you miss the bus by 5~minutes, your expected waiting time is 15~minutes.

\Cref{fig:eta_headway_results} shows the values of $\Pcatch$, $\Ecatch$, $\Emiss$, and $\Ewait$ by headway (rounded down, in minutes). Capture probability is not so affected by headway, though it decreases slightly for longer headways. In contrast, expected wait times are very much affected. Most unexpectedly, $\Ecatch$ decreases with headway, which could be caused by a higher proportion of short headway at peak times when traffic is more congested, leading to more uncertainty (so the quantiles are more dispersed), or the buses take longer than expected. For $\Emiss$, the expected wait time is not unexpectedly strongly correlated with headway. Overall, the difference in $\Ewait$ between predictors is negligible for headways less than 10~minutes, after which time differences appear. Low frequency routes should prefer a low quantile to reduce the chance of missing the bus.


We did not examine time-until-arrival, which we saw in the previous chapter explained a lot of the variation in predictor performance. We would expect a similar result here, so if one wanted to develop a predictor based on both headway and time until arrival, that would be straightforward enough. However, it is clear that the choice of predictor is tightly linked with the cost of a bad prediction: is waiting at the bus stop too long more costly than missing it altogether?

\subsection{Interval estimate}
\label{sec:etas-interval}

\begin{itemize}
\item emphasise that ETAs are symmetric on the probability scale, but not (necessarily) on the arrival time scale (due to lower bound)
\item KF approach gives normal predictions, giving symmetric interval on arrival time scale also (which is probably wrong)
\end{itemize}

Deciding on a ``best'' single-value estimate of arrival time is exceedingly difficult given the amount of uncertainty involved. An alternative approach would be to provide \emph{two} estimates---a lower and upper bound---that is to say, a \emph{prediction interval}. To do so, we need to consult the following criteria:
the probability of the bus arriving outside the interval versus the width of the interval; and
the probability of the bus arriving before the lower bound versus the expected waiting time (given arrival at the lower bound).
The first point corresponds to how useful---from a commuter's point of view---the prediction is: a narrow interval is favourable, but we must balance that with covering a reasonable range of plausible values. The second point again relates to \emph{cost}, which we can most easily express as expected waiting time; there are, of course, other costs (such as being late for work), which we examine more closely in \cref{sec:etas-journey-planning}.







\begin{knitrout}\small
\definecolor{shadecolor}{rgb}{0.969, 0.969, 0.969}\color{fgcolor}\begin{figure}

{\centering \includegraphics[width=\textwidth]{figure/eta_cis-1} 

}

\caption[ETA CIs]{ETA CIs}\label{fig:eta_cis}
\end{figure}


\end{knitrout}


The most straightforward intervals to compute are \emph{symmetric}\footnote{They are not perfectly symmetric, however, due to integer rounding}. That is, for a \mbox{$100(1-\alpha)$\%} interval we use $A_{\alpha/2}$ and $A_{1-\alpha/2} + 1$ (the plus one on the upper bound is so the upper bound is effectively rounded up). We computed intervals for $\alpha \in \{0.1, 0.05, 0.1, 0.2\}$, and for each evaluated the observed coverage, the probability that the bus arrived before the lower bound, and the average width (in minutes). The graphs in \cref{fig:eta_cis} show that for smaller values of $\alpha$ (larger coverage) the coverage probability and average width increase, which is to be expected. Similarly, the probability that the bus arrives before the lower bound (resulting in missing the bus) increases with $\alpha$.

An alternative to using symmetric intervals is to use asymmetric ones: in this scenario, the lower and upper bounds could be estimated using different criteria. For example, the lower bound might be chosen to minimise the chance of missing the bus, while the upper bound is chosen to balance excessive width while still covering as greater plausible range as possible. Again, we might also wish to vary the criteria depending on the proximity of the bus, the stop sequence, or headway. \textcolor{red}{[to do, I suppose]}


\section{Journey planning}
\label{sec:etas-journey-planning}

So far we have been concerned with the arrival time of a bus at a single stop with no consideration of the passenger's commute as a whole. The simplest journey consists of a single route choice, so the only decision is which trip to catch; a slightly more complex journey may offer two alternative routes (\cref{sec:journey_simple}) which the passenger need choose one. Finally, there may be no single route which goes from the passenger's start location to their desired destination, in which case a transfer between two (or more) different routes is necessary (\cref{sec:journey_transfer}).

The remainder of this chapter explores the use of arrival time \glspl{cdf} to estimate probabilities and expectations associated with various journey options. These can then be used by passengers to make decisions based on their journey-specific constraints: for example, are they going to work, meeting a friend for coffee, or having a job interview? Additionally, we will compare (where possible) our results with those made using the currently deployed GTFS system.


\begin{knitrout}\small
\definecolor{shadecolor}{rgb}{0.969, 0.969, 0.969}\color{fgcolor}\begin{figure}

{\centering \includegraphics[width=\textwidth]{figure/eta_journey_arrival_prep-1} 

}

\caption[Route options]{Route options}\label{fig:eta_journey_arrival_prep}
\end{figure}


\end{knitrout}

\subsection{Choosing between two alternative routes}
\label{sec:journey_simple}



In this scenario, a passenger lives within walking distance of two major roads, along which various routes travel into the central city, as displayed in \cref{fig:eta_journey_arrival_prep}. The passenger must decide which route option to take before leaving home for the city. Let's say they have an appointment in town at 9~am, so at 8:10~am, they look at the real-time app on their phone and must decide whether to walk to route option A or B. Factors that will influence their decision may include:
\begin{itemize}
\item How long will I have to wait until the next bus?
\item What is the probability I will arrive in time for the next bus? And how long until the next one?
\item What is the probability that I will arrive at my destination on-time? And how early will I be?
\item How long will I be on the bus?
\end{itemize}
We can answer these questions using the \glspl{cdf} of arrival time at the start and end stops and taking into account walking times.


\begin{knitrout}\small
\definecolor{shadecolor}{rgb}{0.969, 0.969, 0.969}\color{fgcolor}\begin{figure}

{\centering \includegraphics[width=\textwidth]{figure/eta_journey_arrival-1} 

}

\caption[ETA predictions for two route options]{ETA predictions for two route options. The coloured curves represent the CDF of arrival time for individual trips made at 7am. The vertical black lines indicate the estimated walking time (according to Google Maps) from the Start location to each stop.}\label{fig:eta_journey_arrival}
\end{figure}


\end{knitrout}


\Cref{fig:eta_journey_arrival} displays the \glspl{cdf} of all active trips\footnote{Our application currently only estimates arrival times for active trips.} along the two route options at 8:10~am when the passenger is about to leave home (solid line), and includes dashed lines representing the walking time to each of the alternative stops. At the lower end of each \gls{cdf} is a label indicating the trip's start time, which is the simplest method of identifying trips, and the curves are coloured by the route number. For option A, the passenger is likely to miss the 7:48 and 7:50 trips but has a 62\% chance of catching the 7:58. Additionally, two more active trips are arriving later, which the passenger could catch if they miss the 7:58. As for option B, the 7:45 trip is about to arrive at the stop, so can be ignored; if the passenger chooses this option, they have a 23\% and 99\% chance of catching the 7:56 and 8:05 trips, respectively. Since no further trips have begun, we cannot (yet) forecast how long until the next trip if they miss the 8:05 trip.



\begin{knitrout}\small
\definecolor{shadecolor}{rgb}{0.969, 0.969, 0.969}\color{fgcolor}\begin{figure}

{\centering \includegraphics[width=\textwidth]{figure/eta_journey_arriveby-1} 

}

\caption[ETA predictions for two route options]{ETA predictions for two route options. The coloured curves represent the CDF of arrival time for individual trips made at 7am. The vertical black lines indicate the estimated walking time (according to Google Maps) from the Start location to each stop.}\label{fig:eta_journey_arriveby}
\end{figure}


\end{knitrout}

Based on the arrival results alone, it would likely make sense to walk to option A (a shorter walk) which has a decent chance of a short wait time; however, we also need to account for the passenger's appointment at 9~am. \Cref{fig:eta_journey_arriveby} provides \glspl{cdf} of the arrival time of the trips at the final stop, as well as vertical lines representing the appointment time (solid) and walking time (dotted). This time, we want the curve to be on the left-hand side of the dotted line, which is the case for most of the trips. For option A, the 7:58 has an (almost) 100\% chance of arriving on time, while the 8:00 and 8:08 have 98\% and 96\% chances, respectively. Option B tells a similar story, with (almost) 100\% for the 7:55 and 99\% for the 8:05 trips.


\begin{knitrout}\small
\definecolor{shadecolor}{rgb}{0.969, 0.969, 0.969}\color{fgcolor}\begin{table}

\caption{\label{tab:eta_journey_results}Journey planning.}
\centering
\fontsize{8}{10}\selectfont
\begin{tabular}[t]{lllrrllll}
\toprule
\multicolumn{1}{c}{} & \multicolumn{1}{c}{} & \multicolumn{1}{c}{} & \multicolumn{2}{c}{Particle filter} & \multicolumn{2}{c}{GTFS} & \multicolumn{2}{c}{Outcome} \\
\cmidrule(l{3pt}r{3pt}){4-5} \cmidrule(l{3pt}r{3pt}){6-7} \cmidrule(l{3pt}r{3pt}){8-9}
Option & Route & Trip & $P_\text{catch}$ & $P_\text{arrive}$ & Catch & Arrive & Catch & Arrive\\
\midrule
A & 25L & 13:00 & 0.02 & 0.99 & N & Y & N & Y\\
 & 25B & 13:05 & 0.05 & 0.99 & N & Y & N & Y\\
 & 25L & 13:10 & 0.98 & 0.94 & Y & Y & Y & Y\\
 & 25B & 13:15 & 1.00 & 0.81 & Y & N & Y & Y\\
\midrule
B & 27W & 13:05 & 0.00 & 0.99 & N & Y & N & Y\\
 & 27H & 13:15 & 0.90 & 0.86 & Y & Y & Y & Y\\
\bottomrule
\end{tabular}
\end{table}


\end{knitrout}


The values, along with the binary GTFS predictions\footnote{``yes, you will make it'' or ``no, you will not''} and the observed result\footnote{This is historical data, after all} are displayed in \cref{tab:eta_journey_results}. For our predictions, the capture probabilities are all valid, while the arrival probabilities are overly optimistic: the 8:08 on option A and 8:05 on option B have probabilities over 95\%, but the bus does not arrive in time. The GTFS-predictor correctly identified the tardiness of the 8:08 trip, but was not so lucky with the 8:05, which could indicate that this bus was somewhat later than usual.

It is worth noting at this point that, as mentioned in the previous chapter, the current state of our application leads to \emph{underestimates} of arrival time, so it would be wise to be cautious with decisions made based on the above probabilities. Future work improving the underlying speed model and restricting speeds more strongly could improve performance.




\begin{knitrout}\small
\definecolor{shadecolor}{rgb}{0.969, 0.969, 0.969}\color{fgcolor}\begin{figure}

{\centering \includegraphics[width=\textwidth]{figure/eta_journey_results_avg-1} 

}

\caption[Results of performing the same journey planning prediction with different starting times (from 6am to 6pm), using the same start and end locations]{Results of performing the same journey planning prediction with different starting times (from 6am to 6pm), using the same start and end locations. Observations are whether or not the passenger would have arrived at the stop before the bus, jittered to better see each observation. The smoothed curve represents the mean, and indicates that on average our method is good at determining if the bus can be caught or not. Additionally, predictive probabilities of 0 and 1 have been removed.}\label{fig:eta_journey_results_avg}
\end{figure}

\begin{table}

\caption{\label{tab:eta_journey_results_avg}GTFS prediction results over all journeys, with the displayed values representing the proportion of outcomes in each cell (rows are conditioned by the GTFS prediction of Yes of No. }
\centering
\fontsize{8}{10}\selectfont
\begin{tabular}[t]{llllll}
\toprule
\multicolumn{1}{c}{} & \multicolumn{5}{c}{Observed outcome} \\
\cmidrule(l{3pt}r{3pt}){2-6}
\multicolumn{1}{c}{ } & \multicolumn{2}{c}{Catch bus} & \multicolumn{1}{c}{} & \multicolumn{2}{c}{Arrive on time} \\
\cmidrule(l{3pt}r{3pt}){2-3} \cmidrule(l{3pt}r{3pt}){5-6}
GTS Prediction & No & Yes &  & No & Yes\\
\midrule
No & 0.97 & 0.03 &  & 0.57 & 0.43\\
Yes & 0.11 & 0.89 &  & 0.01 & 0.99\\
\bottomrule
\end{tabular}
\end{table}


\end{knitrout}

Of course, the results in \cref{fig:eta_journey_arrival,fig:eta_journey_arriveby} and \cref{tab:eta_journey_results} are based on \emph{one single forecast} made at 8:10~am. To evaluate the performance of our prediction method, we repeated the process described above in 5~minute intervals between 6~am and 6~pm. For the appointment time, we used 15~minute intervals allowing for 30--45 minutes for the journey. That is, leaving between 7:15 and 7:30 had a targetted arrival time of 8~am; 7:30--7:45 targetted 8:15; and so on. In each case, we computed the probabilities for each of catching the bus and arriving on time. In \cref{fig:eta_journey_results_avg}, we have graphed the distribution of predicted probabilities for each of the two outcomes, for both catching the bus and arriving on time (A and B, respectively).

Further, \cref{tab:eta_journey_results_avg} presents a two-way contingency table for the binary GTFS predictions, with the outcome in rows. In 89\% of cases, the GTFS method correctly predicted that the passenger would catch the bus, and in 97\% of cases did it correctly identify that a passenger would miss the given trip. Conversely for the arrivals, in 99\% of cases the GTFS method correctly predicted that the bus would arrive on time; however, in only 57\% of cases did it correctly identify that the bus would \emph{not} arrive on time.


\subsection{Planning a multi-stage journey}
\label{sec:journey_transfer}

A more complex scenario is one where the passenger must transfer from one route onto another. Transfer journeys are common for travellers commuting from further afield, where there are often several \emph{feeder routes} which connect at a hub which usually has more frequent trips to another hub. It may also happen that your local route does not go to your destination, as is demonstrated in \cref{fig:eta_journey_arrival_prep}. In this scenario, the passenger must first catch a bus along route group A\footnote{We use route groups since there are several different routes which make the same journey, as can be seen with route group B in the south-west of \cref{fig:eta_journey_arrival_prep}.} to the stop marked ``Transfer'', at which point they disembark and wait for the next bus along route group B to get to their final destination (marked ``End'').



\begin{knitrout}\small
\definecolor{shadecolor}{rgb}{0.969, 0.969, 0.969}\color{fgcolor}\begin{figure}

{\centering \includegraphics[width=\textwidth]{figure/eta_journey_transfer_prep-1} 

}

\caption[Transfer options]{Transfer options}\label{fig:eta_journey_transfer_prep}
\end{figure}


\end{knitrout}



\begin{knitrout}\small
\definecolor{shadecolor}{rgb}{0.969, 0.969, 0.969}\color{fgcolor}\begin{figure}

{\centering \includegraphics[width=\textwidth]{figure/eta_journey_transfer_graph-1} 

}

\caption[Arrival times]{Arrival times}\label{fig:eta_journey_transfer_graph}
\end{figure}


\end{knitrout}


Similarly to the previous example, we take one single forecast of all trip arrivals at 1~pm to make decisions. For simplicity, we leave out walking time, but as in the previous section this could easily be included, if desired. \Cref{fig:eta_journey_transfer_graph} shows, for all active trips at 1~pm, their arrival time \glspl{cdf} at the start and transfer stops (for the first leg) and the transfer and end stops (for the second). We could of course add additional constraints, such as arrival by a specific time, but in this instance we are solely intereste in whether or not each trip in group A will arrive \emph{before} the trips in group B.


To calculate the probability that a given trip along route A will arrive at the transfer stop before a given trip along route B, we use the following inequality:
\begin{equation}
\label{eq:eta_total_prob}
\begin{split}
\Pcatch =
\Pr{A < B} &= \sum_{x=1}^{\infty} \Pr{A < B\,|\,B = x}\Pr{B=x} \\
  &= \sum_{x=1}^\infty
    \Pr{A < x} \Pr{B=x} \\
  &= \sum_{x=1}^\infty
    \Pr{A < x} \left[
      \Pr{B < x + 1} - \Pr{B < x}
    \right]
\end{split}
\end{equation}
which we can easily compute given the \glspl{cdf} of the arrival time of each trip obtained from \cref{eq:pf_cdf_arrivaltime}. In table \cref{tab:eta_journey_transfer_res} we display the results, along with the binary predictions using GTFS and the observed outcomes. Our results appear reasonable at predicting transfer probabilities; in contrast, the GTFS makes one poor choice between A1 and B3, which it predicts \emph{will} connect, but infact they do not (our prediction was for a 32\% chance). This is where a prediction probability shows its power, in that a passenger could, where applicable, base their decision on how important it is they make the transfer. While this is a fairly non-exciting example (it's a short wait for the next bus), there are other situations where the headway between trips is 30--60~minutes, in which case missing a transfer by a few minutes would be frustrating. However, until our application has been updated to make predictions for upcoming (and not just active) trips, we could not obtain any examples of this.


\begin{knitrout}\small
\definecolor{shadecolor}{rgb}{0.969, 0.969, 0.969}\color{fgcolor}\begin{table}

\caption{\label{tab:eta_journey_transfer_res}Transfer probabilities}
\centering
\fontsize{8}{10}\selectfont
\begin{tabular}[t]{llrll}
\toprule
Leg 1 & Leg 2 & $\mathbb{P}_\text{transfer} & GTFS & Outcome\\
\midrule
1A & 2A & 0.04 & N & N\\
 & 2B & 0.71 & Y & Y\\
 & 2C & 0.32 & Y & N\\
 & 2D & 1.00 & Y & Y\\
 & 2E & 0.99 & Y & Y\\
\midrule
1B & 2A & 0.00 & N & N\\
 & 2B & 0.20 & N & N\\
 & 2C & 0.05 & N & N\\
 & 2D & 0.96 & Y & Y\\
 & 2E & 0.93 & Y & Y\\
\bottomrule
\end{tabular}
\end{table}


\end{knitrout}


As before, we proceed to repeat the procedure for multiple times (this time between 6~am and 8~pm), and compute the probabilities of available transfers and whether or not they connected. \Cref{fig:eta_journey_transfer_many} shows the predicted transfer probabilities grouped by outcome, and coloured by the GTFS prediction. This is complimented with contingency \cref{tab:eta_journey_transfer_many,tab:eta_journey_transfer_many2} for the GTFS method and our own, respectively. For the GTFS method, in 16\% of cases a predicted successfull connection failed, versus only 8\% using our method and a rule of $\Ptransfer > 0.5$. The false negative rate was similar for both methods, although slightly higher for our own (7\% versus 3\% for GTFS).


\begin{knitrout}\small
\definecolor{shadecolor}{rgb}{0.969, 0.969, 0.969}\color{fgcolor}\begin{figure}

{\centering \includegraphics[width=\textwidth]{figure/eta_journey_transfer_many-1} 

}

\caption[Results of performing the same transfer journey planning prediction with different starting times (from 6am to 8pm)]{Results of performing the same transfer journey planning prediction with different starting times (from 6am to 8pm). Observations are whether or not the first bus would arrive at the transfer stop before the second bus, jittered to better see each observation. The smoothed curve represents the mean, and indicates that on average our method is good at determining if the bus can be caught or not. Additionally, predictive probabilities of 0 and 1 have been removed since, after checking, none or these were invalid and only served to distort the figure.}\label{fig:eta_journey_transfer_many}
\end{figure}


\end{knitrout}

\begin{knitrout}\small
\definecolor{shadecolor}{rgb}{0.969, 0.969, 0.969}\color{fgcolor}\begin{table}

\caption{\label{tab:eta_journey_transfer_many2}Results of running the transfer prediction problem over the course of the day. Rows represent the predicted outcome (No, the transfer will be made, or Yes, it will) and the numbers represent the proportion of predictions with each observed outcome. The GTFS rule is binary, while the particle filter results are based on transfer probability of at least 0.5, 0.8, or 0.95 (as indicated).}
\centering
\fontsize{8}{10}\selectfont
\begin{tabular}[t]{llllllllllll}
\toprule
\multicolumn{1}{c}{} & \multicolumn{11}{c}{Outcome} \\
\cmidrule(l{3pt}r{3pt}){2-12}
\multicolumn{1}{c}{} & \multicolumn{2}{c}{GTFS} & \multicolumn{1}{c}{} & \multicolumn{2}{c}{PF (P > 0.5)} & \multicolumn{1}{c}{} & \multicolumn{2}{c}{PF (P > 0.8)} & \multicolumn{1}{c}{} & \multicolumn{2}{c}{PF (P > 0.95)} \\
\cmidrule(l{3pt}r{3pt}){2-3} \cmidrule(l{3pt}r{3pt}){5-6} \cmidrule(l{3pt}r{3pt}){8-9} \cmidrule(l{3pt}r{3pt}){11-12}
Prediction & No & Yes &  & No & Yes &  & No & Yes &  & No & Yes\\
\midrule
No & 0.97 & 0.03 &  & 0.93 & 0.07 &  & 0.84 & 0.16 &  & 0.7 & 0.3\\
Yes & 0.16 & 0.84 &  & 0.08 & 0.92 &  & 0.03 & 0.97 &  & 0 & 1\\
\bottomrule
\end{tabular}
\end{table}


\end{knitrout}



In this section we have seen how having access to the full distribution of arrival times allows us to compute the probabilities of events, rather than make singular binary predictions of their outcome. This can enable more sophisticated decision making by travellers depending on their own needs. Of course, all of the examples shown were manually estimated, so some work is needed before it can be automatic, but the availability of the \glspl{cdf} for each trip and stop makes this easy. It has the additional advantage that, were any improvements to the vehicle or network models made, these would automatically be integrated into the arrival prediction component and immediately be used to adjust the probabilities. From here, it is up to travellers themselves, or other intermediate agencies (which could be a transport planning app, for example) to use the available information to make decisions.

\begin{itemize}
\item references on dynamic routing (it's a hard problem, etc etc)
\end{itemize}

