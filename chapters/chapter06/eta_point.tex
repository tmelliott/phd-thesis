\subsection{Point estimate}
\label{sec:etas-point}

In a perfect world, it would be possible to predict precisely how long it will take a bus to arrive at a stop, and display this single number to passengers. Alas, as we have seen, arrival time prediction is prone to significant levels of uncertainty. However, since much of the infrastructure currently only allows for point estimates, we present some here. We want to examine if it is possible to come up with a single statistic that performs well---on average---and, more importantly, is more reliable than the currently deployed method (which we refer to as the schedule-delay method).





\begin{knitrout}\small
\definecolor{shadecolor}{rgb}{0.969, 0.969, 0.969}\color{fgcolor}\begin{figure}

{\centering \includegraphics[width=.7\textwidth]{figure/eta_overall_results-1} 

}

\caption[Comparison of summary statistics for various quantiles of the predictive distribution, and the currently deployed GTFS method]{Comparison of summary statistics for various quantiles of the predictive distribution, and the currently deployed GTFS method. Results are displayed for all day average, and off-peak (between 9h30 and 14h30).}\label{fig:eta_overall_results}
\end{figure}


\end{knitrout}


To compare several choices of point estimate, we computed---for every stop $i$ along every trip $j$ at each time $k$---a selection of quantiles
\[
  q \in \{0.05, 0.25, 0.5, 0.6, 0.6, 0.75, 0.9\}.
\]
For each, we calculated \gls{mae} and \gls{mape} to assess accuracy, along with the observed proportion of estimates $\hat\Teta_{qijk}$ that were \emph{earlier} than the bus' true arrival $A_{ijk}$,
\begin{equation}
\label{eq:pr_bus_caught}
P_\text{caught} = \frac{N_\text{caught}}{N_\text{eta}}
= \frac{\sum_{i,j,k} I_{A_{ijk} < \hat\Teta_{qijk}}}{\sum_{i,j,k} 1}
\end{equation}
where $A_{ijk}$ is a real number, $N_\text{caught}$ is the number of predictions earlier than actual arrival, and $N_\text{eta}$ is the total number of predictions. For those estimates that were earlier than the actual arrival, we computed the average waiting time
\begin{equation}
\label{eq:avg_wait_time}
\bar W_\text{caught} =
\frac{1}{N_\text{caught}} \sum_{i,j,k} A_{ijk} - \Teta_{qijk}.
\end{equation}
The results are displayed in \cref{fig:eta_overall_results}, along with the same values computed using the schedule-delay arrival time estimates. For the particle filter, the lowest \gls{mae} value is achieved by PF60 with $q = 0.6$, of which 67\% of preditions were earlier than the true arrival and had an average waiting time of 2.7~minutes. From $P_\text{catch}$, the quantiles are underestimating arrival time, which is to be expected given our rule for computing them (\cref{eq:eta_calc_quantile}). We would expect the 50\% quantile to have a 50\% success rate, but it has instead a 75\% rate; referring back to \cref{fig:eta_calc_quantile} we see that, due to the discreet nature of the \gls{eta} \gls{cdf}, the prediction is $\hat\Teta_{0.5} = 11$ which is, in fact, the 25\% quantile [TOM\footnote{Better words for these few lines}].



An important consideration is the cost of missing the bus, which we define as the scenario in which the predicted arrival time is later than the bus' true arrival. For each trip, we computed the scheduled time between the current trip and the subsequent one, referred to as \emph{headway}, to compute the expected waiting time if a passenger misses the bus. That is, if a bus is predicted to arrive in 5~minutes, but actually arrives in 3, and the time between buses servicing the same trip is $H = 10$~minutes, then the expected waiting time for a passenger arriving at the predicted arrival time (in 5~minutes) will be $10+3-5=8$~minutes.


The total expected wait time can be conditioned on whether or not the bus was caught,
\begin{equation}
\label{eq:eta_wait_conditional}
\begin{split}
\E{\text{wait}} &=
  \Pr{\text{catch}} \E{\text{wait}\cond{}\text{catch}} +
  (1 - \Pr{\text{catch}}) \E{\text{wait}\cond{}\text{miss}} \\
  &= \Pr{A \geq \hat\Teta} \E{A - \hat\Teta \cond{} A \geq \hat\Teta} +
  \Pr{A < \hat\Teta}\E{H + A - \hat\Teta \cond{} A < \hat\Teta}
\end{split}
\end{equation}
where $\hat\Teta$ is the estimated arrival time, and $A$ is the actual. For simplicity, we assume headway $H$ is maintained---that is, if a bus has 20~minute headway and you miss it by 5~minutes, the next bus is expected to arrive in 15~minutes. This is not true in most situations, however, but predicting headway is a difficult problem that has been the focus of its own research \citep{Hans_2015,cn}.


To estimate $\Pr{\text{catch}}$ and $\E{\text{wait}\cond{}\text{catch}}$ we used the values estimated by \cref{eq:pr_bus_caught,eq:avg_wait_time}, respectively. This allowed us to estimate the total expected wait times, as displayed in \cref{fig:eta_headway_results} over a range of trip headways. Capture success rates were mostly unaffected by headway, while average wait times, unexpectedly, shorted for less frequent trips. This could be attributed to there being more trips with short headway at peak times when there is more uncertainty in arrival times, and buses tend to be later than predicted. The relationship between headway and wait time if the bus is missed is as expected, with little noticeable different between predictors.


The overall wait time shows little different between predictors for trips with a headway less than 10~minutes, but the methods quickly disperse as headway increases beyond this. However, we can see that it is possible to maintain the expected wait time below 10~minutes by choosing an appropriate quantiles, which indicates that for low frequency routes the most reliable estimate would be the 5\% or 25\% quantile, which minimises the probability of missing the bus. As a comparison, the same values were calculated based on the schedule-delay method, and are shown in \cref{fig:eta_headway_results} as dashed black lines. We see that, in most cases, it is the equivalent to using $q=0.75$, and has a 50\% chance of catching the bus, so the total expected waiting time is about 50\% of the headway.

\begin{knitrout}\small
\definecolor{shadecolor}{rgb}{0.969, 0.969, 0.969}\color{fgcolor}\begin{figure}

{\centering \includegraphics[width=\textwidth]{figure/eta_headway_results-1} 

}

\caption[Capture probabilities and expected wait times by trip headway (time until the next trip) by using varying quantiles as point estimates of arrival time, and assuming the passenger arrives at the specified time]{Capture probabilities and expected wait times by trip headway (time until the next trip) by using varying quantiles as point estimates of arrival time, and assuming the passenger arrives at the specified time. The dashed black line represents the same values using the schedule-delay prediction.}\label{fig:eta_headway_results}
\end{figure}


\end{knitrout}
