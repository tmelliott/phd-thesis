\section{Particulars of the \rt{} implementation}
\label{sec:nw_implementation}

After stepping through the real-time model, estimation of its parameters, and improvement of forecats using historical data, the last remaining piece of the network model is to implement it in real-time within our iterative program, implemented in C++ within our \verb+transitr+ package. The first step is, of course, to load the historical data and model both the real-time parameters using JAGS. These are then inserted into the same database in which we store all of the GTFS and network data.


The network is implemented as independent objects of class \verb+Segment+, which are initialised when the program is first started with their model parameter values loaded---if available---from the database. The network state itself is implemented using the `Eigen' C++ library, allowing us to store the state vector and uncertainty matrix as an \verb+Eigen::Vector+ or \verb+Eigen::Matrix+, respectively. The library includes all common (and uncommon) vector and matrix operations, for example multiplication. Even though the current state is one-dimentional, we may in future wish to extend this to include, say, the rate of change of speed, which could better model peak periods. This would then involve a length-two state vector and $2\times2$ uncertainty matrix. By using `Eigen', this change would involve minimal code changes.


The following sections discuss the initialisation, data-extraction, update, and prediction steps as implemented within our package, as well as some final details on the real-time performance.


\subsection{Initialisation}
\label{sec:nw_implementation_init}

The initialisation phase of the network is quite important, since in many cases it will be the only piece of information available for speed and arrival time predictions, at least until some vehicles have traversed it and provided observations. Therefore, to initialise the segment, we check if there is any historical data available for it: if there is, we simply set $\NWstate_{\ellc}$ to be the historical mean speed at time $t_c$ (as we estimated in \cref{sec:nw_hist_model}). To initialise the uncertainty, we use $\NWstatevar_{\ell 0} = \frac{30}{3.6^2}$~\gls{mps} (which is approximately 30~km/h).

Another part of initialisation is setting the constraint for the \emph{maximum speed}. Since we do not know the speed limit along most road segments, the best we can do is to use the maximum speed limit, 100~km/h, which is approximately 30~m/s. For segments with enough observations, we can use the maximum observed speed to determine the posted speed limit along the road.


\subsection{Data collection}
\label{sec:nw_implementation_data}

As discussed in \cref{sec:vehicle_speeds}, traffic speed along road segments are estimated each time a vehicle completes travel along a road segment. Within the vehicle update step, there is a \emph{current segment} index, $r$, which is initialised to zero. After each mutation, the program determines the minimum segment which each particle is currently on, call this $r_{\text{min}}$. If this is greater than the previous value of $r$, then the speed for all segments from $r$ to $r_{\text{min}}-1$ are computed using \cref{eq:pf_speed_dist}.
\textcolor{blue}{[[move to chapter 3?]]}


After estimating the mean and uncertainty of vehicle speeds along a segment, the data is \emph{pushed} onto the segment object using a \verb+push_data()+ method, which is thread-safe (\cref{sec:rt-implementation}), allowing multiple vehicles to simultaneously complete a segment and add observations to it when the program is running on multiple cores.



\subsection{State update}
\label{sec:nw_implementation_update}

The state update step is quite simple, particularly since we have independent segments and a one-dimensional state. First, segments are updated after the vehicle states have finished processing, and any new observations have been collected. Then a parallel iteration over segments is performed to check if there are any new observations: if there are, then the update step (\cref{sec:nw_realtime}) is performed. Thanks to the \verb+Eigen+ library, the implementation is straightforward.

Once the state has been updated, the data vector is cleared and the segment's timestamp set to the current time.


\subsection{State forecasts}
\label{sec:nw_implementation_forecast}

While not part of the network state model, the forecasting is an essential component of the real-time application. Each segment has a \verb+forecast+ function which returns the forecasted speed, given the current timestamp\footnote{this is the time the segment was updated, not the wall clock time} and segment traffic speed. The forecasts are available in 10~minute intervals, so a forecast for 5~minutes uses the current travel time state, while a forecast for 25~minutes uses the 20~minute forecast, and so on, capped at 60~minutes. As for the uncertainty, this comes from the historical data too, but rather than forecasting we simply use the variance of vehicle speed at the given time of day.

Currently the forecast method used i the na\:ive constant-speed method; however, since the next arrival time prediction component accesses forecasts through a \verb+Segment::forecast()+ method, once the historical-based kNN forecasts have been implemented they will automatically be integrated into the forecast model with no change needed to the prediction component.
