\section{Particulars of the \rt{} implementation}
\label{sec:nw_implementation}

After stepping through the real-time model, estimation of its parameters, and improvement of forecasts using historical data, the last remaining piece of the network model is its real-time implementation within \pkg{transitr} package. The first step is, of course, to load the historical data and model the real-time parameters using \prog{JAGS}. These are then inserted into the same database in which we store the \gls{gtfs} and network data.


The network is implemented as independent objects of class \class{Segment}, which are initialised when the program is first started with their model parameter values loaded---if available---from the database. The network state itself is implemented using the \pkg{Eigen} \Cpp{} library, allowing us to store the state vector and uncertainty matrix as an \verb+Eigen::Vector+ or \verb+Eigen::Matrix+, respectively. This library includes all the necessary vector and matrix operations, for example, multiplication. Even though the current state is one-dimensional, we may in future wish to extend this to include, say, the rate of change of speed, which could better model peak periods. This would involve a length-two state vector and $2\times2$ uncertainty matrix. By using \pkg{Eigen}, this change would require minimal code changes. The following sections discuss the initialisation, data-extraction, update, and prediction steps as implemented within \pkg{transitr}.


\subsection{Initialisation}
\label{sec:nw_implementation_init}

The initialisation phase of the network is quite important since in many cases it will be the only piece of information available for speed and arrival time predictions, at least until a vehicle traverses it and provides a real-time observation. Therefore, to initialise a \class{Segment}, we check if there is any historical data available for it: if there is, we set $\NWstate_{\ell,0}$ equal to the historical mean speed of all observations over time. To initialise the uncertainty, we use $\NWstatevar_{\ell 0} = \frac{30}{3.6^2}$~\gls{mps} (which is approximately 30~km/h).

Another part of initialisation is setting the constraint for the \emph{maximum speed}. Since we do not know the speed limit along most road segments, the best we can do is to use the maximum speed limit, 100~km/h, which is approximately 30~m/s. For segments with enough observations, we can use the maximum observed speed to estimate the posted speed limit along the road.



\subsection{State update}
\label{sec:nw_implementation_update}

The state update step is simple due to the segments being independent with one-dimensional states. First, \class{Segment}s are updated after processing the \class{Vehicle}s to collect any new observations (\cref{sec:vehicle_speeds}). Then we perform a parallel iteration over \class{Segment}s: if there are new observations, the update step (\cref{sec:nw_realtime}) is executed. The \pkg{Eigen} library makes this implementation straightforward. Once the state has been updated, the data vector is cleared and the \class{Segment}'s timestamp set to the current time.


\subsection{State forecasts}
\label{sec:nw_implementation_forecast}

While not part of the network state model, forecasting is an essential component of our real-time application. Each \class{Segment} has a \verb+forecast()+ method which returns the forecasted speed given the current timestamp\footnote{This is the time the \class{Segment} was updated, not the wall clock time.} and segment traffic speed. The forecasts are available in 10~minute intervals, so a forecast for 5~minutes uses the current travel time state, while a forecast for 25~minutes uses the 20-minute forecast, and so on, capped at 60~minutes. As for the uncertainty, this comes from the historical data too, but rather than forecasting, we use the historical variance of vehicle speeds at the desired time.

Currently, the forecast method uses the na\:ive constant-speed predictor. Since the arrival time prediction component (\cref{cha:prediction}) accesses forecasts through the \verb+Segment::forecast()+ method, an implementation of the historical-based forecasts\footnote{Or indeed any other desirable forecast method.} will automatically be integrated into the forecast model with no change needed to the prediction component.
