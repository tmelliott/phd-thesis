
\section{Estimating network parameters}
\label{sec:nw_par_est}


To use the Kalman in any meaningful way, we first need to estimate the system noise, $\NWnoise_\ell$, and between-vehicle variance, $\NWvar_{\ellc}$. To estimate values for these parameters, we fit the same model from \cref{sec:nw_model} to the historical data shown in \cref{fig:tt_figure} using \gls{mcmc} sampling methods as implemented by JAGS \citep{JAGS}.


For the first step in the modelling process, we fit the following hierarchical model:
\begin{align}
\label{eq:nw_model_simple}
\Vttobs_{\ellc}^m &\sim \Normal{\Vtt_{\ellc}^m}{(\Vtterr_{\ellc}^m)^2} &
\NWstate_{\ell,0} &\sim \Uniform{0}{\MaxSpeed_\ell} \nonumber \\
\Vtt_{\ellc}^m &\sim \TNormal{\NWstate_{\ellc}}{\NWvar_{\ell}^2}{0}{\MaxSpeed_\ell} &
\NWvar_\ell &\sim \GammaD{0.01}{0.01} \\
\NWstate_{\ellc} &\sim \TNormal{\NWstate_{\ell-1,c}}{(\NWtdiff_c \NWnoise_{\ell})^2}{0}{\MaxSpeed_\ell} &
\NWnoise &\sim \GammaD{0.01}{0.01} \nonumber
\end{align}
For the observation error, as in the previous section, we assumed this to be constant for all observations, $\Vtterr_{\ellc}^m = 0.8$~\gls{mps} (which is about 10~km/h). The model implementation was performed from R using the \verb+rjags+ package \citep{rjags}. The model output was processed using the \verb+tidybayes+ package \citep{tidybayes} and graphed using \verb+ggplot2+ \citep{ggplot2}. The \verb+coda+ package was used to assess the model's convergence results \citep{coda}.


\subsection{Simulated data}
\label{nw_par_est_sim}

\begin{knitrout}\small
\definecolor{shadecolor}{rgb}{0.969, 0.969, 0.969}\color{fgcolor}\begin{figure}

{\centering \includegraphics[width=0.8\textwidth]{figure/nw_model_sim_results-1} 

}

\caption[Traceplots of model parameters for simulated data, with each of the four chains coloured separately]{Traceplots of model parameters for simulated data, with each of the four chains coloured separately. Dashed grey lines show the 95\% posterior credible interval, and the solid lines represent the true values.}\label{fig:nw_model_sim_results}
\end{figure}


\end{knitrout}


To assess the model, we start by fitting it to the simulated data (\cref{fig:nw_sim_data}), for which we know the values of $\NWnoise$ and $\NWvar$. Traceplots of the model parameters are shown in \cref{fig:nw_model_sim_results}, with the true values overlaid with a dashed line, show that the chains mixed well. \Cref{tab:nw_model_sim_smry} gives a summary of the simulation values and their posterior estimates, along with convergence statistics. The 95\% credible intervals for the two parameters contain the true values, so there is nothing to suggest the model is inadequate for modelling the simulated data and estimating the network parameters.


\begin{table}

\caption{\label{tab:nw_model_sim_smry}Simulation MCMC results. The multivariate $\hat R$, including for all of the $\NWstate$ parameters, was 1.04, indicating that convergence had been achieved after 15,000 iterations.}
\centering
\begin{tabular}[b]{llllllr}
\toprule
Parameter & True Value & Mean & 2.5\% & 50\% & 97.5\% & $\hat R$\\
\midrule
$\NWnoise$ & 0.002 & 0.0017 & 0.0014 & 0.0017 & 0.0021 & 1.001\\
$\NWvar$ & 0.8 & 0.75 & 0.64 & 0.75 & 0.86 & 1.001\\
\bottomrule
\end{tabular}
\end{table}





\subsection{Real data}
\label{nw_par_est_real}



Having shown that the JAGS model is a valid way of estimating the network parameters $\NWnoise$ and $\NWvar$, the real travel time data from \cref{fig:tt_figure} was modelled. Traceplots of the network parameters, as shown in \cref{fig:nw_model_n1_view}, show again that the chains mixed well, which is reaffirmed by the convergence results in \cref{tab:nw_model_fit_smry}. Not shown here are the summaries of the $\NWstate$'s (there are 187 of them), however the Gelman diagnostic $\hat R$ was less than 1.03 for all of them, indicating they reached convergence.

\begin{knitrout}\small
\definecolor{shadecolor}{rgb}{0.969, 0.969, 0.969}\color{fgcolor}\begin{figure}

{\centering \includegraphics[width=0.8\linewidth]{figure/nw_model_n1_view-1} 

}

\caption[Model output for the parameters $\NWnoise$ and $\NWvar$, with 95\% posterior credible region denoted by dashed grey lines]{Model output for the parameters $\NWnoise$ and $\NWvar$, with 95\% posterior credible region denoted by dashed grey lines.}\label{fig:nw_model_n1_view}
\end{figure}


\end{knitrout}

\begin{table}

\caption{\label{tab:nw_model_fit_smry}MCMC results for the average segment speeds. The multivariate $\hat R$, including for all of the $\NWstate$ parameters, was 1.07, indicating that convergence had been achieved after 15,000 iterations.}
\centering
\begin{tabular}[b]{lllllr}
\toprule
Parameter & Mean & 2.5\% & 50\% & 97.5\% & $\hat R$\\
\midrule
$\NWnoise$ & 0.0014 & 0.00096 & 0.0014 & 0.0022 & 1.005\\
$\NWvar$ & 1.79 & 1.63 & 1.78 & 1.96 & 1.001\\
\bottomrule
\end{tabular}
\end{table}



To assess the adequacy of these parameters, we fit the \kf{} to the same data\footnote{We are more thorough in the next section, which includes data from two weeks for testing and training.} to see how well the mean and predictive distributions fit the data. The estimate $\hat\NWstate_{\ellc|\ellc}$, along with their associated uncertainty, are shown in \cref{fig:nw_model_n1_kf1}, where we see that the mean approximately follows the center of the data. The predictive vehicle speed distribution, which accounts for both $\NWstatevar_{\ellc}$ and vehicle variation, $\NWvar_{\ellc}$, is shown in \cref{fig:nw_model_n1_kf2}, which shows most of the points contained within the 95\% posterior predictive region.


\begin{knitrout}\small
\definecolor{shadecolor}{rgb}{0.969, 0.969, 0.969}\color{fgcolor}\begin{figure}

{\centering \subfloat[Estimated mean speed, showing state estimates along with the 95\% credible region.\label{fig:nw_model_n1_kf1}]{\includegraphics[width=.8\textwidth]{figure/nw_model_n1_kf-1} }\\
\subfloat[Predictive distribution of vehicle speeds, which includes between-vehicle variability.\label{fig:nw_model_n1_kf2}]{\includegraphics[width=.8\textwidth]{figure/nw_model_n1_kf-2} }\\

}

\caption[Fitted KF to observed speed data]{Fitted KF to observed speed data.}\label{fig:nw_model_n1_kf}
\end{figure}


\end{knitrout}



Finally, the timing comparison, displayed in \cref{tab:nw_model_n1_timecomp}. The hierarchical model fit using JAGS took about 2700 times as long as the \kf{} implementation. So, even if we reduced the number of chains, the number of iterations, and tried to speed up the JAGS model, the \kf{} would still be significantly faster.


\begin{table}

\caption{\label{tab:nw_model_n1_timecomp}Timing results (seconds).}
\centering
\begin{tabular}[b]{lrrr}
\toprule
  & User & System & Total\\
\midrule
JAGS & 123.264 & 0.004 & 123.273\\
Kalman filter & 0.044 & 0.000 & 0.046\\
\bottomrule
\end{tabular}
\end{table}




\subsection{Hierarchical model over multiple road segments}
\label{sec:nw_par_est_multiple}



To assess how effective this method is for estimating $\NWvar$ and $\NWnoise$, we selected five other road segments around Auckland and repeated the above process. However, instead of fitting the same model independently over segments,  we use a hierarchical Bayesian model on the parameters, since it seems reasonable that, while they wont be the same over all segments, there will be an underlying population distribution. This will allow us to obtain estimates of the parameter values without explicitly needing to model every road (of which there are 8,151).


The following hierarchical model was used:
\begin{align*}
\label{eq:tt_hist_hier}
\Vttobs_{\ellc}^m &\sim \Normal{\Vtt_{\ellc}^m}{\left(\Vtterr_{\ellc}^m\right)^2} &
\NWstate_{\ell0} &\sim \Normal{0}{10^2} \\
\Vtt_{\ellc}^m &\sim \TNormal{\NWstate_{\ellc}}{\NWvar_{\ell}^2}{0}{\MaxSpeed_\ell} &
\mu_\psi &\sim \Normal{0}{10^2} \\
\NWstate_{\ellc} &\sim \TNormal{\NWstate_{\ellc-1}}{(\NWtdiff_c \NWnoise_{\ell})^2}{0}{\MaxSpeed_\ell} &
\sigma_\psi &\sim \GammaD{0.001}{0.001} \\
\log\left(\NWvar_{\ell}\right) &\sim \Normal{\mu_\psi}{\sigma_\psi^2} &
q &\sim \GammaD{0.001}{0.001}
\end{align*}
We initially attempted to fit a hierarchical segment noise parameter $\NWnoise_{\ell}$ with hyperparameters, but the values were all approximately equal for all segments and convergence was very slow (100,000's of iterations). Instead we have opted for a single common system noise parameter across all segments. The speed data for six segments on two consecutive Tuesdays is shown in \cref{fig:nw_model_n2_segplots}, with the locations of the roads displayed in \cref{fig:nw_seg_maps}.

\begin{knitrout}\small
\definecolor{shadecolor}{rgb}{0.969, 0.969, 0.969}\color{fgcolor}\begin{figure}

{\centering \includegraphics[width=.8\textwidth]{figure/nw_model_n2_segplots-1} 

}

\caption[Speeds along six roads throughout Auckland on two consecutive Tuesdays]{Speeds along six roads throughout Auckland on two consecutive Tuesdays.}\label{fig:nw_model_n2_segplots}
\end{figure}


\end{knitrout}

\begin{knitrout}\small
\definecolor{shadecolor}{rgb}{0.969, 0.969, 0.969}\color{fgcolor}\begin{figure}

{\centering \includegraphics[width=\textwidth]{figure/nw_seg_maps-1} 

}

\caption[Segment locations, drawn using the ggmap package \citep{ggmap}]{Segment locations, drawn using the ggmap package \citep{ggmap}. Segments begin at the solid point and terminate at the open point.}\label{fig:nw_seg_maps}
\end{figure}


\end{knitrout}




\begin{knitrout}\small
\definecolor{shadecolor}{rgb}{0.969, 0.969, 0.969}\color{fgcolor}\begin{figure}

{\centering \includegraphics[width=\textwidth]{figure/nw_model_n2_diag-1} 

}

\caption[Traceplots of top-level network parameters]{Traceplots of top-level network parameters.}\label{fig:nw_model_n2_diag}
\end{figure}


\end{knitrout}


\begin{table}

\caption{\label{tab:nw_model_n2_smry}MCMC results for the top-level parameters for the hierarchical segment model. The multivariate $\hat R$, including for all of the $\NWstate$ parameters, was 2.03.}
\centering
\begin{tabular}[b]{lllllr}
\toprule
Parameter & Mean & 2.5\% & 50\% & 97.5\% & $\hat R$\\
\midrule
$\mu_\phi$ & 0.383 & -0.191 & 0.389 & 0.926 & 1.00\\
$\sigma_\psi$ & 0.608 & 0.315 & 0.549 & 1.24 & 1.00\\
$q$ & 0.0014 & 0.0012 & 0.0014 & 0.0017 & 1.02\\
\bottomrule
\end{tabular}
\end{table}




The model was fit using JAGS to the first day of data for the six segments, while the second was reserved for testing the validity of the estimated parameters. Each of the four chains were run with a 100,000~iteration burn-in phase, followed by 50,000 iterations with a thinning interval of 50. Traceplots of the top-level parameters are displayed in \cref{fig:nw_model_n2_diag}, which demonstrate good mixing of the chains. \Cref{tab:nw_model_n2_smry} shows the posterior mean and quantiles for these parameters, along with their Gelman convergence diagnostic.


Traceplots for the segment-specific variance parameter are shown in \cref{fig:nw_model_n2_diag_2}, and also demonstrate good mixing. The Gelman convergence diagnostic is again very close to unity, indicating that running the chain for longer would not decrease the posterior variance by much.



\begin{knitrout}\small
\definecolor{shadecolor}{rgb}{0.969, 0.969, 0.969}\color{fgcolor}\begin{figure}

{\centering \includegraphics[width=\textwidth]{figure/nw_model_n2_diag_2-1} 

}

\caption[Traceplots of network parameters for each segment]{Traceplots of network parameters for each segment.}\label{fig:nw_model_n2_diag_2}
\end{figure}


\end{knitrout}


The posterior mean of $\NWvar$ for each segment was used along with the population parameters to fit the \kf{} as before. \Cref{fig:nw_model_n2_kf} shows the original data overlayed with a posterior sample of $\NWstate$'s, along with the next week's data with results from the \kf{}, including the 95\% credible region for the mean travel time, along with the 95\% posterior predictive region for individual vehicle travel times.







\begin{knitrout}\small
\definecolor{shadecolor}{rgb}{0.969, 0.969, 0.969}\color{fgcolor}\begin{figure}
\includegraphics[width=\textwidth]{figure/nw_model_n2_kf-1} \caption[Results for the hierarchical approach to modelling road speed]{Results for the hierarchical approach to modelling road speed. The training data is shown on the left, with a sample of posterior fits. On the right is the test data with the results of the Kalman filter estimation, showing speed estimate, its uncertainty (red), and the posterior predictive region for vehicle speeds (grey).}\label{fig:nw_model_n2_kf}
\end{figure}


\end{knitrout}

The results show that the $\NWstate$ values estimated with JAGS nicely fit the data, as do the \kf{} estimates. The posterior predictive region covers most of the points, and is not excessively large, in all of the segments.
