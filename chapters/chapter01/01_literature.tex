


\section{Review of bus prediction}
\label{sec:literature}


The driving force behind advances in transit modelling and arrival time prediction
has been the evolution of the technologies used to track transit vehicles in \rt{}.
As we will see, much of the literature focuses on a specific type of data,
or is constrained by the available technologies at the time.
The major factor is predictive models is the \rt{} faesibility,
which becomes more of a problem as fleet sizes increase.


Transit vehicles have been fitted with \gls{avl} devices for many decades now \citep{TCRP_1997},
but usually has been limited in its use for several reasons.
First and foremost is the deployment of devices, are were often expensive.
There's also the issue of retrieving, processing, and publishing \gls{rti} for commuters to use,
a non-issue today with mobile devices becoming a necessity in developed cities.


Collection and storage of data is also considered,
particularly when its use in predictive models is desired,
for example in regression of neural network models.


There are three main categories of data collection discussed here.
These are vehicle location, time points, and passenger counts.
\emph{Vehicle location} data refers to observations of a vehicle's position,
either in reference to the route
(for example using an odometer or fixed signpost recievers to observe distance travelled),
or the Earth, as is the case with \gls{gps} devices,
which are now almost standard.
\emph{Time point} observations are made by reporting the time that a particular vehicle
arrived at or departed from a particular location,
such as a bus stop, intersection, or other (e.g., automatic toll readers, \citep{Yu_2011}).
Finally, observations of the number of passengers boarding and debarking are made using \glspl{apc},
which while unavailble in Auckland, are prominent in some of the major sources cited.
This data is often used in a historical context,
providing estimates of passenger demand, which has a large influence on dwell time at stops.


There are also other sources of information unique to some papers,
whether it be signal patterns, traffic flows, or other,
and papers have made use of its specialising for operational purposes.


The models themselves fall into three categories:
\rt{} state-based, historical data, and a combination of the two.
The latter is by far the most popular, as many papers have shown it to outperform the others.


Of the models used in the last 20~years, the \kf{} has been the most popular due to its simplicity
\citep{Wall_1999,Dailey_2001,Cathey_2003,Shalaby_2004,Yu_2010}.
The main concept is to infer the current position and speed of a vehicle along the route
and based on some travel time prediction function
estimate the \glspl{eta} at future stops
\citep{cn}.
We give a full description of the \kf{} in \cref{sec:kf}.


One key idea is that of travel time variability over time.
\citet{Cathey_2003} proposed a general prescription using a prediction function
which could use historical data, including time of day, weather, etc
to generate predictions.
Others using \gls{knn} and \gls{ann} to predict arrival times based off a variety of data
\citep{Jeong_2005,Yu_2006,Yu_2010,Yu_2011}.
Many used specific data types,
such as on-board observations \citep{cn},
cameras \citep{Xinghao_2013,Yu_2011},
and toll readers \citep{Yu_2011}.
{}

These models saw improvements, but still unable to respond to \rt{} events accurately.
The concept of using headway---the time since the previous trip---has been implemented,
and shown improvements \citep{cn}.
This is, in effect, estimating \rt{} traffic.


However often trip frequency is not high enough to capture \rt{} changes in congestion.
So \citet{Yu_2011} proposed using data from multiple routes to estimate current travel times.
This was done by using automatic toll readers uproute from a major stop.
This provided a marked improvement over alternative single-route strategies.
Since then, other methods of estimating traffic state (congestion, effectively)
have beeen proposed, as as using taxis \citep{Xinghao_2013},
and other stuff \ldots
Some more advanced methods using shockwaves proposed by \citet{Julio_2016}.


As far as the methods used to estimate \rt{} vehicle and road state,
many have been used bus the predominant one has been \emph{\gls{rbe}};
most other methods---namely \gls{ann} and \gls{svm}---were used
with time point and \gls{apc} data,
so vehicle tracking was not incorporated in the model
(no filtering of noisy position observations was required).


We will give an indepth overview of \gls{rbe} in \cref{sec:recursive-bayes},
but for now suffice to say that it is a commonly used method of filtering out
noisy observations of a Markov process, particularly usedful in vehicle tracking applications
\citep{Gordon_1993,Carpenter_1999,Gustafsson_2002}.


The \kf{} is the simplest of the \glspl{rbe},
and was commonly used by the likes of \citet{Wall_1999,Cathey_2003} and others.
However, it is a very simplisted model and has high potential for failure,
particularly when sampling intervals are long \citep{cn},
allowing a higher uncertainty in the vehicle trajectory.


In \citet{Hans_2015} (?? or earlier) a \pf{} was used to model transit vehicles.
The main advantage was described as being that it could cover a higher proportion
of likely trajectories, even with low sampling.
They also used particle trajectories for prediction (the median was used as a point estimate).
Their goal, however, was in operations control, not \glspl{eta}.


Modelling the (\rt{}) travel times along roads or, more commonly, between stops,
has been shown to be a vital predictor of arrival time
\citep{cn}.
Initial work just used the time of the previous trip along the same route,
combined with some \gls{svm} model to predict current vehicle travel time \citet{Yu_2006}.
Further work used other routes serving the same stop(s) \citet{Yu_2011}.
More recently, \citet{Cats_2016} (maybe 2015 paper too?) used travel times of previous buses
between stops to predict travel times,
and some fancy weighting formula.



Perhaps the most important question is now, \emph{is this all worth it}?
The technologies and models have been implemented for the sole purpose of
improving the reliability of \glspl{eta} displayed to commuters.
Early research found that a countdown reduced experienced wait times
\citep{TCRP_2003,TCRP_2003b},
although they did not mention accuracy of the countdown.
Other people have \ldots






\pagebreak

Over the last two decades,
there has been an enormous advance in both the technologies available and the models used
to track buses in \rt{} and simultaneously make arrival-time predictions for the upcoming stops.
Of particular interest are the recursive Bayesian filters,
namely the Kalman Filter, which has been used in transit models since the late 1990's,
and the \pf{}, which has only recently shown up in the transit literature,
but has several unique characteristics we wish to exploit in our work.
In addition to these models,
there has also been a lot of progress in the field using data-oriented and machine learning models
such as \gls{knn}, \gls{ann}, and \gls{svm}.
We will briefly describe these,
focusing on the transit-specific features that were incorporated into them.


The first use of recusive Bayesian models in arrival time estimation
was by \citet{Wall_1999},
in which measurements of \emph{distance-until-destination}
were used to estimate vehicle position,
and a \emph{travel time function} based on historical data
was used to estimate time until arrival.
The simple model, based solely on historical data and the current position,
was capable of estimating arrival time to within 5~minutes accuracy,
even with sparse data,
along two test routes.

\textcolor{red}{
    \singlespacing
    There's another paper by \citet{Dailey_2001}.
    \begin{itemize}
        \item predicting \gls{eta} up to 1~hour in advance
        \item \gls{avl} data: \{time, location\}, 1--3~min rate
        \item low path sampling, so complex model inappropriate
            (\emph{this is needed for ch~3!!!})
        \item with historical statistics
        \item generate \emph{time to arrival} function,
            using average speed between current pos and stop
        \item piecewise constant model over parts of route
        \item updated using \kf{} for optimal \glspl{eta}
        \item real arrival times linearly interpolated
        \item results show vast improvement (accuracy and reliability)
            compared to schedule
    \end{itemize}
}

In a later paper,
\citet{Cathey_2003} described a general framework for arrival time prediction
involving three components:
the tracker, the filter, and the predictor.
The \emph{tracker} consists of vehicle data combined with schedule information,
allowing linking ot trips and vehicles.
The \emph{filter} allows a sequence of \rt{} observations of vehicle position
to estimate location and speed along the route.
The \emph{predictor} is charged with generating arrival time estimates.


The tracking component used by \citet{Cathey_2003} has become more-or-less obsolete
with advances in the vehicle tracking technologies
(see \cref{sec:gtfs}).
For the position filter, they used a \kf{} (\cref{sec:kf}),
which assumes a Gaussian state distribution.
From this they were able to estimate the instantaneous speed of a vehicle,
as well as smooth noisy position observations.


For the predictor step, \citet{Cathey_2003} provided several examples.
In one, they used the scheduled travel times,
and therefore schedule-based speeds,
to estimate the travel time, explicitely accounting for layover stops.
In the second example,
a two-dimentional speed function over distance (along route) and time,
obtained from a probe vehicle (so, historical data),
was used instead.
They also used predictive error over time to determine
how early a passenger should arrive at a stop
before the estimated arrival time to have a 90\% probability of catching the bus,
and compared this to using the schedule to show that their proposed methodology
provided a significant gain in information.

\textcolor{red}{
    \singlespacing
    Things missed \ldots
    \begin{itemize}
        \item give block id (sequence of trips for a single vehicle),
            which is not available in \gls{gtfs}
    \end{itemize}
}



Another important component of transit modelling are the passengers themselves,
which, thanks to new \gls{apc} technologies,
\citet{Shalaby_2004} were able to incorporate into their predictive model.
In their test system,
they had available to them vehicle location,
the arrival and departure times at bus stops,
as well as the number of passengers boarding and alighting.
They then fit two separte \kf{} models to each of the
link running time (travel time between stops)
and the dwell time (time spent waiting at a stop while passengers
board and alight).
Their model used high demand locations,
so stop skipping was not included in their model.


Historical data (three previous days) was used to predict the
running time and dwell time at the next time point.
From the \gls{apc} data, they were also able to compute
passenger arrival rates at stops,
and therefore improve the dwell time predictions:
a bus running late will encounter more waiting passengers,
and the dwell time will be longer.
A \kf{} algorithm based on another used for a similar model
by \citet{Reinhoudt_1997},
was used which predicted travel time using historical data
along the next link,
and travel time along the previous link.


\citet{Shalaby_2004} compared their \kf{} results to other models
implemented on the same route,
and found that it outperformed historical (only),
regression, and neural network models in a variety of situations,
notably a ``special event'', and during lane closures,
which demonstrates the importance of using \rt{} information.
They did note that the predictive model needed to be developed
further to handle overlapping routes serving the same bus stops.


The importance of \rt{} information on traffic conditions
was mentioned by \citet{Jeong_2005},
who compared several models, including \gls{ann},
in a \rt{} application.
Again, they stress the importance of dwell time in predictions,
which was used along side travel time and
schedule adherence in their \gls{ann} model.
They compared two separate predictive approaches:
one in which the model was retrained as new data was recieved,
and another that used data from the previous model update to make predictions,
and found that there was no significant improvement by
retraining in \rt{}.



\citet{Yu_2006} implemented an \gls{svm}, a variant of \gls{ann},
to predict arrival time for a single route,
and showed improved predictions but large computational requirements.

\textcolor{red}{
    \singlespacing
    \begin{itemize}
        \item data consisted of arrival time at time points
        \item uses travel time of current/preceeding bus on links
            to estimate traffic conditions, develop prediction models
        \item their \gls{svm} model can integrate latest bus info
            and predict accurate arrival
        \item falls over with larger deployment
    \end{itemize}
}

Later, \citet{Yu_2010} propose hybrid \gls{svm}/\kf{}.
\textcolor{red}{
    \singlespacing
    \begin{itemize}
        \item SVM to predict baseline from historical data
        \item KF combines latest arrival information with
            SVM to predict arrival
        \item travel time data collected via on-board collection
        \item time points for \rt{} data
        \item \kf{} hybrid models perform better than ANN/SVM alone
        \item low accuracy in high demand/narrow roads,
            affecting travel times/dwell times
    \end{itemize}
}

\citet{Chang_2010}
\textcolor{red}{
    \singlespacing
    \begin{itemize}
        \item
    \end{itemize}
}





\subsection{Particulars of Bus Behaviour}
\label{sec:bus-behaviour}


Many of the models used in transit research were originally developed for
more generic tracking applications or traffic state models;
however, there are several features that are unique to transit vehicles,
such as bus lanes and bus stops \citep{Yu_2011},
so models have needed adjusting accordingly.
The two which are of particular importance are dwell time and headway,
which have become a requirement in modern bus prediction methods
\citep{Jeong_2005,Hans_2014,Hans_2015,Cats_2016}.


Dwell time is defined as \emph{the total time lost due to stopping at a bus stop}.
This is shown visually in \cref{fig:dwell-time},
which shows the hypothetical paths of two vehicles,
one of which stops while the other does not.
Therefore, there are two components to consider:
whether the bus stops, and if so, for how long.
The former can be expressed as a Bernoulli random variable, $p_j \sim \mathcal{B}(\pi_j)$,
where $\pi_j$, the average proportion of buses that stop,
is modeled according to the data available and the framework being used.


If a bus does stop, it must first decelerate,
and after servicing the bus stop it will take some time to accelerate back up to speed.
Additionally, there is time taken to open and close the doors.
These delays are usually combined into a common affect, $\gamma$,
as they apply to every bus and stop.
Once a bus has stopped, it will remain there while passengers board and disembark.
This can be modeled with an exponential distribution,
$\tau'_j \sim \mathcal{E}(\mu_\tau)$,
or else more complex queuing models can be constructed if the data allows
\citep{Hans_2015}.
The total dwell time experienced by the vehicle is $\tau_j = p_j(\gamma + \tau_j')$,
as shown in \cref{fig:dwell-time}.

% \begin{figure}[bt]
%   \centering
%   \includegraphics[width=0.8\textwidth]{dwell_time.png}
%   \caption{Dwell time, $\tau_j$, is the total time lost due to stopping at a bus stop (indicated by the dashed black line).}
%   \label{fig:dwell-time}
% \end{figure}

The other important feature is headway,
which is the \emph{time delay between consecutive buses},
and can affect both travel times and dwell times.
If two consecutive routes are close together,
then the travel times are expected to be similar;
however, since they are both servicing the same route,
we expect the number of passengers waiting at a stop---and therefore
the dwell time---to be inversely proportional to the headway
\citep{Hans_2014,Hans_2015}.




\subsection{Availability of AVL Data}
\label{sec:data-types}

While there has been a lot of research in transit applications,
many of them have been concerned with operations control,
for example reducing bus bunching behaviour \citep{Hans_2015},
or else they require more informative data than is available in most systems,
such as \glspl{apc} and actual arrival times.
For the latter, it is usually because the \gls{avl} recording is triggered by stops
\citep{Hans_2015},
is high frequency \citep{Chang_2010},
or in some cases the test data was collected manually
\citep{Yu_2010}.


In Auckland's transport system, the only publicly available data is provided
by \gls{gps}, which is updated approximately every 30~seconds,
so arrival times must be estimated as latency variables and,
in a \rt{} deployment, will remain unknown.
Another issue specifically with Auckland Transport is that,
over the next few years,
the route structure is being completely redesigned
and rolled out slowly across the city.
Therefore, focus will be on using \rt{} and recent data (same day, week) to predict arrival times,
rather than relying on historical data collected over months or years.


