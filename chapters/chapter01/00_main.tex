\glsresetall

\chapter{Introduction}
\label{cha:intro}


Over the last three decades, technological advances have changed the way people use public transport. Before \gls{rti}, travellers would arrive at a bus stop based exclusively on the scheduled arrival time and then wait until the bus finally appeared around the corner. There was no way of knowing if the bus was on-time, running late, or indeed if it was on its way at all. These days, however, travellers can check the location of a bus from their phone before even leaving home. Despite the accessibility of such \gls{rti}, unforeseeable traffic situations and various bus behaviours make
converting locations into arrival time estimates a difficult task.


Around the world, public transport providers have invested in systems to provide \gls{rti} to travellers. Research has found that the experienced wait time of travellers is shorter (versus their actual wait time) when an arrival time countdown is displayed at the stop \citep{TCRP_2003}. Further, \citet{Cats_2015} and \citet{Lu_2017} each reported that passengers who use \gls{rti} actually have shorter wait times, on average, than those who do not. Of course, if \glspl{eta} are unreliable, passengers instead get frustrated. Improving the \emph{reliability} of \gls{rti} is therefore crucial for improving \emph{ridership} (the number of passengers using public transport).


One limitation of \gls{rti} is the infrastructure required to keep track of vehicles and relay information to commuters \cite{TCRP_2003b}. As fleet sizes increase (over 1000 in Auckland), so do the demands of the \gls{atis}, which has led some providers, notably Auckland Transport, to use the simplest of arrival time prediction frameworks. This simplification consequently results in less reliable systems, as I will discuss in \cref{sec:auckland_etas}.


The history of \gls{rti} for public transport spans several decades, which we will now examine, focussing on the key concepts essential for reliable \glspl{eta}. Afterwards, the current state of arrival time prediction in Auckland, New Zealand is examined to establish why---despite decades of research---there is room for significant improvement. Finally, I present the aims of this research and summarize the process of estimating the arrival time of buses.


\section{A brief history of real-time information}

For a long time, public transport services such as buses and trains had only static timetables that passengers would use to plan their journey. There was no way for passengers to know if their bus was on-time, late, or early, nor could operators track how their services were running. In the 1960s, the first vehicle tracking systems were trialled in Germany and the United States of America \citep{TCRP_1997}. These early systems used \emph{beacons} (on signposts) placed along the route that the bus detects, allowing the bus to provide its location to the service provider in real-time. Similar vehicle tracking technologies continued to develop over the subsequent decades, but the focus remained on service monitoring and operational control.


The earliest uses of \gls{avl} technology for passenger information were \glspl{dms} at bus stops displaying a countdown to the next bus's arrival \citep{TCRP_2003}. One early example of this was Transport for London's \emph{Countdown} system \citep{Balogh_1993}, which used the beacon (or signpost) \gls{avl} technology in conjunction with vehicle odometers. The system was able to determine vehicle locations and predict arrival times at stops.


Other vehicle tracking methods have since emerged using a range of technologies, such as odometers, but the most notable is the widely used \gls{gps}. One significant advantage of the \gls{gps} is that it does not need any fixed infrastructure (such as signposts along the route), making it easy for transit providers to add new routes or reroute existing ones without breaking the \gls{avl} system. A disadvantage of the \gls{gps} is that, rather than receiving \emph{route-specific} positions (``Signpost 8'', for example) the observations are \emph{map coordinates} which require an additional step to match them to the route.\footnote{Route matching and the issues associated with it are discussed in depth in later chapters.}


\citet{Cathey_2003} proposed a general prescription for making real-time arrival time estimates from \gls{gps} vehicle location data consisting of three components. First, a \emph{tracker} matches observations to scheduled \emph{trips} (time-specific instances of a route), which involves mapping \gls{gps} observations to the \emph{route path} to calculate the bus's \emph{trip-distance-travelled} (the distance the bus has driven along the route). Second, a \emph{filter} estimates the underlying vehicle state (including speed) which is updated, in real-time, as new observations are received. \Citeauthor{Cathey_2003} used a Kalman filter for this step due to its superior real-time performance and previous use for modelling transit vehicle state \citep{Wall_1999,Dailey_2001}. In the final \emph{prediction} component of their prescription, \citeauthor{Cathey_2003} used the estimated vehicle state in conjunction with travel time forecasts (estimated from historical data) to predict \emph{time until arrival}.


Other \gls{avl} systems report the vehicle's arrival and departure times from \emph{time points} (which are usually a subset of specific stops), providing observations of travel time between them. This type of reporting led to a range of new methods of arrival time prediction. For example, \gls{ann} and \gls{svm} models were implemented to predict travel time \citep{Jeong_2005,Shalaby_2004,Yu_2011,Cats_2015,Cats_2016,Yin_2017}. Many of these were in-depth research projects with access to high-quality data, such as high-frequency polling (10~seconds of less) or video footage to calculate, for example, precise arrival and departure times. Despite their differences, all of their methods specify the importance of two components to arrival time prediction: \emph{travel time} between stops (or time points), and \emph{dwell time} (time spent servicing a stop).


The concept of travel time between time points forms the foundation of arrival time estimation, so the estimation or even forecasting of travel times has been the focus of much research. Returning to Transport for London's \emph{Countdown} system, \citet{Reinhoudt_1997} stored real-time travel time between time points and stored them in a database. They then used a \kf{} to update link travel times, improving the accuracy of arrival time prediction. Others, such as \citet{Wall_1999}, \citet{Dailey_2001}, and \citet{Cathey_2003}, used historical data to estimate \emph{time until arrival} given a vehicle's trip-distance-travelled and speed. Further improvements to prediction accuracy were obtained when real-time travel time along links were updated using a \kf{}, which was capable of reacting to changes in traffic \citep{Shalaby_2004}.


Other models used for arrival time prediction include \gls{ann} and \gls{svm}. \Citet{Yu_2006} demonstrated the feasibility and applicability of using an \gls{svm} to predict travel time based on the current segment travel time and the travel time of the most recent bus servicing the same route along the next segment. The limitation was, of course, that the model was trained for only one single route and only provided an arrival time for the next time point. Later, \citet{Yu_2010} proposed an improved hybrid model using an \gls{svm} to model baseline travel time and a \kf{} to incorporate real-time data. Their method was again limited to a single route and single-interval, but demonstrated the importance of using real-time information for arrival time prediction. The next significant result was from \citet{Yu_2011} when they demonstrated the use of data from multiple routes to predict arrival time. In their study, \citeauthor{Yu_2011} used the travel time of recent buses between an automatic toll reader and a specific bus stop. While not generalised, they demonstrated the advantage of combining data across routes.


\Citet{Yin_2017} presented a model to predict travel times between stops using travel times of previous buses from multiple routes, in which they chose two overlapping routes and manually identified common stops. They were able to predict travel time reasonably well, though both their \gls{svm} and \gls{ann} models failed to capture the peak period congestion. An alternative to machine learning models was proposed by \citet{Chen_2014}, who used a \emph{particle filter}\footnote{Their particle filter is unrelated to ours, but the general methodology is the same and described in \cref{sec:recursive-bayes}} to predict car travel time along roads from a database of historical data. Another unique approach by \citet{Julio_2016} used the concept of \emph{traffic shock waves} to predict travel times of buses along links based on the travel time along adjacent links. They used \gls{gps} observations to calculate vehicle trajectories over time, and developed an \gls{ann} to learn patterns: as congestion builds along a segment, so too does congestion along the segment before it, for example.


Most of the methods described above only provide predictions for travel time along the next link, so arrival times are not available for stops further down the route. \Citet{Chang_2010} developed a \gls{knn} algorithm trained on historical vehicle trajectories. They predicted travel time along multiple upcoming links along a single route both accurately and efficiently (their method could make 4000~predictions per minute).


Since travel time of previous buses had been shown to improve arrival time predictions, other sources of information were explored to further improve prediction accuracy. This could be important along roads with low-frequency trips or subject to fast-changing traffic conditions. \Citet{Xinghao_2013} and \citet{Ma_2019} incorporated real-time taxi data to model traffic state. While this showed improvements, it is clearly limited to locations with open access to taxi data.


Another source of uncertainty in arrival time comes from \emph{dwell time} at stops. That is the time spent by the bus at stops while passengers board and disembark. \Citet{Shalaby_2004} showed that dwell times can have a massive influence on arrival times. Along with a \kf{} for link travel times (each link consisted of 2--8~stops), \citeauthor{Shalaby_2004} also implemented a \kf{} on stop dwell times, allowing them to respond to real-time demand fluctuations which they measured using \glspl{apc}. Other work has also demonstrated the necessity of incorporating dwell times into arrival time predictions \citep{Jeong_2005,Cats_2015,Cats_2016}.


When it comes to modelling transit vehicles and predicting arrival times, there is a lot of uncertainty in vehicle trajectories, particularly when observations are sparse, and much of this uncertainty is non-Gaussian (particularly multi-modal). \Citet{Hans_2015} presented a particle filter for modelling transit vehicles. Their model incorporated bus stop dwell times and traffic lights, and their data came from \gls{gps} and \glspl{apc}. The primary advantage of the particle filter over other methods (such as the \kf{}) is that it is capable of sampling a wide range of plausible trajectories. This is particularly important when forecasting multiple stops ahead, when the level of uncertainty increases significantly.


Since uncertainty is an unavoidable component of arrival time prediction, some attempts at conveying this to travellers has been explored. \Citet{Mazloumi_2011} assessed the use of \emph{prediction intervals} obtained from an \gls{ann} model, focussing on the accuracy (coverage) of these intervals. A more unique approach was demonstrated by \citet{Fernandes_2018} that involved displaying \emph{uncertainty graphs}---including quantile dot plots---and found that their test subjects made better decisions when they had uncertainty information, as opposed to singular point estimates.


The final component of \gls{rti} that we are interested in is \emph{journey planning}, which involves the selection of an optimal route to get to a destination, often under one or more constraints. \Citet{Horn_2004} developed a real-time routing method that could incorporate \emph{walking} and \emph{waiting costs} (each of these should be minimised where possible), with a focus on \emph{demand-responsive} services. More recently, \citet{Hame_2013a,Hame_2013b} proposed a \emph{Markov decision process} which could maximise the probability of arriving to the destination on-time, accounting for walking legs and distributions for the arrival time of buses at stops. Travel time uncertainty was incorporated into the method used by \citet{Zheng_2016}, who also demonstrated the complexity of the problem, which could take $10^3$~seconds to find an optimal route; their solution was to store paths, which reduced the computation to less than a second. As an alternative to incorporating travel time uncertainty, \citet{Berczi_2017} propose a dynamic routing strategy that uses any probabilistic model of arrival and departure times, and improved the probability of arriving at the destination on time.



\section{Arrival time predictions in Auckland and abroad}

Introduce Auckland, New Zealand, as the test site.
\begin{itemize}
\item examples involve AT
\item some may apply to others, particularly those using GTFS-realtime
\end{itemize}

Some Auckland-specific issues
\begin{itemize}
\item schedule adherence is bad
\item transfers not guaranteed
\end{itemize}

GTFS-realtime (cite) has made distributing and access transit data easier but has also led to a (likely unintended) side-effect.
\begin{itemize}
\item trip updates report current delay when the bus arrives at a stop
\item this delay is then added to the scheduled arrival time at upcoming stops -> ETA
\item unreliable if the schedule is not valid
\item compounded by [those effects described above] where drivers may try to finish faster
\end{itemize}

Besides being inaccurate and unreliable, arrival times based on schedule and current delay are prone to sudden changes.
\begin{itemize}
\item frustrates travellers
\item bus gets later -> ETA jumps
\item bus gets earlier -> ETA slumps
\item bus late to first stop and suddenly arrives -> ETA JUMPS HUGE, or the trip disappears from the DMS if the bus doesn't get to the first stop before it's scheduled to get to a target stop
\end{itemize}

Systems around the world may use specific setups, such as using taxi data or otherwise location specific.
\begin{itemize}
\item taxi data not widely available
\item Celan + Lep located (manually?) stops and potential time barriers for their network
\item they require manual setup!
\end{itemize}

Conversely, GTFS and GTFS-realtime provide an opportunity for a standard, globally available framework.
\begin{itemize}
\item desirable to have a framework that can use just GTFS
\item provide some method of combining information across buses/routes to estimate traffic state
\item extensible! can be deployed (Easily) to any other region/country that uses GTFS
\end{itemize}

Some nice build up to the strong point (what we are trying to achieve, essentially):
\begin{itemize}
\item to the best of our knowledge no such generalised GTFS-based system estimating network traffic conditions to predict bus arrival time exists
\item nor do many incorporate uncertainty into predictions (i.e., prediction intervals, arrival time distributions)
\end{itemize}

\section{Research proposal}

Public transport is essential for growing cities where travellers need to get to and from work, school, or other activities dependably.
\begin{itemize}
\item the primary source of unreliability is ETAs - inaccurate and untrustworthy
\item might jump (up or down) due to congestion/free-flow/speeding driver
\item deters commuters who might prefer alternative (private) transport options
\end{itemize}

The primary focus of this thesis is the development of an arrival time prediction framework that includes real-time traffic information based solely on GTFS(-realtime).
\begin{itemize}
\item not just for Auckland
\item traffic state independent of routes to increase available information
\item will compare its reliability against the current method in Auckland
\item we consider not only the reliability of point estimates, but also focus on the distribution (and prediction intervals) - we cannot predict arrival time accurately, but we want to account for as much uncertainty as possible
\end{itemize}

We also consider journey planning, notably the use of arrival time distributions to make decisions.
\begin{itemize}
\item currently no real-time decision-making scheme in Auckland
\item some work (Berczi et al., 2017) can make use of probabilistic arrival time distributions to make dynamic route choices
\item not our intention to select candidate routes (that's hard) but instead to demonstrate the reliability/usefulness of our arrival time distribution to decide between options
\item answer for example: which bus should I catch? will I get to work on time? what is the probability I'll make the transfer/how long will I have to wait?
\end{itemize}

The first necessity before approaching this problem is to provide an overview of the relevant background information.
\begin{itemize}
\item GTFS
\item what real-time transit data looks ("and feels") like
\item developing a network form from GTFS to allow the combination of data across routes
\item a brief overview of the types of models used throughout (RBMs)
\end{itemize}

A lot of previous research uses the concept of \emph{vehicle state}, which is updated when new data is received \citep{Cathey_2003}.
\begin{itemize}
\item their (and many others') approach introduces unnecessary uncertainty (map matching)
\item proposal: developing a method of incorporating GPS observations directly into the likelihood
\item main goal: to estimate average vehicle speed along road segments in real-time (similar to the work of Celan and Lep (2017, 2018))
\end{itemize}

After estimating vehicle speeds from the buses traversing the road network, it is possible to model the traffic state itself.
\begin{itemize}
\item similar to other work we implement a KF on the network state
\item requires modelling of historical data to determine model parameters
\item also presents the opportunity to forecast road state - the current implementation of the arrival time application does not use this, but we give a possibility here to demonstrate
\end{itemize}

Lastly is arrival time estimation itself, which combines vehicle and road states to generate an arrival time distribution.
\begin{itemize}
\item we explore how to obtain it
\item and how well the distribution approximates the truth (the uncertainty is real!)
\item in Ch 6 we focus on journey planning: point estimates for a quick "when will the bus arrive"; prediction intervals for simple planning; and answering more comprehensive JP questions (journey length, arrive-on-time probability, transfer probability, etc.)
\item the focus is on the effectiveness (reliability) of our PF proposal, and not journey planning itself (which is difficult and there exists already methods of dynamic routing using probabilistic blah blah blah)
\end{itemize}

Since this is a real-time application, the computational constraints must also be assessed throughout the thesis. Therefore at the end of each chapter, we outline the programmatic considerations necessary and discuss the relative components of the R package `transitr` developed as part of this research.
\begin{itemize}
\item main goal: 30~seconds (or less) per iteration
\item includes fetching data from the server and writing data to a file available to users
\end{itemize}

Finally, in chapter 7, we review our methods and findings and comment on what could be done to improve the predictions. The role of RTI in public transport is only going to become more critical as cities---particularly Auckland---grow, and improving the reliability of the information is the first step to increasing ridership.






\pagebreak

Over the last three decades, technological advances have changed the way people use public transport. In times gone by, one might have used a paper timetable to decide when to leave home, and then be subjected to an indefinite wait until the bus finally appeared around the corner. These days, you just take out your phone to see the bus' precide location on a map. Despite this revolutionary change, to this day it remains a complicated business of converting a location into an \gls{eta}.


Around the world, public transport providers have invested in systems to provide \gls{rti} to their passengers in efforts to improve ridership, the number of commuters choosing to use public transport. Research has shown that \gls{rti} reduced the experienced wait time at bus stops \citep{TCRP_2003} and allows for improved decision making by passengers, leading to shorter waiting times \citep{Cats_2015,Lu_2017}. This, of course, assumes the \gls{rti} is reliable, wherein lies the issue.


One of the main issues with \gls{rti} is that it requires considerable infrastructure to keep track of and report the information \citep{TCRP_2003b}, particularly in recent times when fleet sizes exceed 1000~vehicles. This has inevitably led to a simplification of how arrival times are predicted, which consequently has led to less reliable information, particularly in Auckland, New Zealand.


\section{Arrival time estimation of transit vehicles}
\label{sec:literature}

For a long time, public transport services such as buses and trains had static timetables that passengers could use to plan their journey. In the 1960s, the first vehicle tracking systems, which used fixed signposts to detect the passing of a vehicle, were trialed in Germany and the USA \citep{TCRP_1997}. This allowed the transit provider to monitor the \rt{} performance of its services from a central location.


Over the next few decades, \gls{avl} technology was slowly deployed primarily as a monitoring tool for service providers. The first use of \gls{avl} in \gls{rti} was as a \gls{dms} at bus stops displaying the \gls{eta}, in minutes, until the bus arrives \citep{TCRP_2003}. The most prominent example at the time was Transport for London's \emph{Countdown} system, which used the signpost \gls{avl} system with an odometer to determine the vehicle's position and provide \glspl{eta}.



The most difficult challenge for the provision of accurate \gls{rti} is time itself, particularly in the 1990s when computers were much less powerful than they are today. In response to this, simple, fast algorithms were developed that used \rt{} location information to update the estimates of arrival times, the most prominent being the \kf{}. \citet{Reinhoudt_1997} implemented a \kf{} to model the \emph{link travel time}, replacing the weighted average used by Transport for London's Countdown system, and found that it led to more accurate predictions of arrival time. Later, \citet{Wall_1999} and \citet{Dailey_2001} implemented \kf{}s to track the state of a vehicle, allowing for uncertainty in its location and speed to be incorporated into the arrival time prediction, which was based on a \emph{travel time function} estimated from historical data.



The main problem with the signpost \gls{avl} system is that it needs to be rolled out for each individual route. It is therefore no surprise that many transit providers began to favour the distribution of \gls{gps} devices to track their fleets. The result of this was a necessity for a much more generalised prediction framework, since data was no longer route-specific and instead required more complex methods for both tracking and prediction. \citet{Cathey_2003} released a general prescription in which they described the three necessary components to arrival time prediction: tracking, filtering, and prediction.


Tracking is, in our context, the process of combining schedule information with \rt{} location information to acquire the necessary data for the rest of the model. For transit vehicle tracking using a \gls{gps}, the first step is to match the location to the route: geographic observations are processed to compute, for example, \emph{trip distance travelled}, which is the distance the vehicle has travelled along the route \citep{Cathey_2003,Gong_2013}. Other methods involved using arrival time observations, in which case the travel time between \emph{time points} (using significant stops along the route) is observed \citep{Shalaby_2004,Jeong_2005,Yu_2011}.


The filtering component is responsible for taking a sequence of vehicle observations (as determined by the \emph{tracker}) and estimating the \emph{state} of the vehicle, which often includes distance travelled along the route and speed \citep{Dailey_2001,Cathey_2003}. This is where the computational efficiency of the \kf{} plays its pivotal role, since the state can easily and quickly be estimated (details are given in \cref{sec:recursive-bayes}). However, despite being adopted early, more recent applications favour deterministic methods, such as arrival time reporting \citep{Yin_2017,Cats_2015,Cats_2016}. While easy to implement, and often accurate, this approach can reduce reliability in the presence of poor data (such as erroneous or missing arrival or departure times), as observed in the Auckland Transport system.


The final component is the predictor, which combines the vehicle location information with a variety of other information (which can be historical or \rt{}). The general concensus is that this `other information' needs to include \emph{link travel time}---that is, the time taken to travel between various points along the route---and \emph{stop dwell time}---which is the time lost while servicing a bus stop.



Of course, not only has the technology for tracking a vehicle improved over the years: so have computers, leading to more and more complex prediction schemes. \Glspl{svm} \citep{Yu_2006,Yu_2010} and \glspl{ann} \citep{Mazloumi_2012,cn} models trained with historical link travel time data were found to better predict the arrival times of upcoming buses based on the travel time of the preceeding bus, and \citet{Yu_2011} extended the idea to involve vehicles from multiple routes, showing significant improvments over other methods. However, the main limitation was that the routes were pre-specified, which is a common constraint of proposed methods \citep{Chang_2010,Celan_2017,Celan_2018}.


A lot of the innovation for modelling transit vehicles has been in operational research, rather than \gls{rti}. For example \citet{Hans_2015} developed a particle filter to model buses in \rt{} in an attempt to reduce bunching (when consecutive trips end up travelling one after the other rather than spaced out temporally). The particle filter is a computationally demanding generalisation of the \kf{}, so historically it was impossible to implement for \rt{} applications, but in recent years this has changed; indeed, many of the \rt{} models employed come with computational constraints and need to be assessed for their feasibility \citep{Chang_2010,Cats_2016}.




\subsection{Dwell time behaviour at bus stops}
\label{sec:intro_dwell}

The effect of dwell time has also been shown to greatly influence arrival time uncertainty \citep{Jeong_2005,Meng_2013,Shen_2013,Robinson_2013,Gong_2013,Wang_2016}, though in most of these the available data has been either manually collected or available through technologies such as \glspl{apc}, which are less common than \gls{gps}.


\subsection{Road travel times for arrival time prediction}
\label{sec:intro_travel_time}

Early work, such as by \citet{Reinhoudt_1997} and \citet{Wall_1999}, demonstrated the use of \emph{link travel times} to inform the arrival time of upcoming buses.

Any state-of-the-art system for arrival time prediction needs to use \rt{} vehicle location data along with \rt{} link travel times and dwell time information. Where possible, link travel times should be obtained from more than one route in order to expand the available data, or even, if possible, non-transit data. The most popular choice of outside information is taxi data, which is often available to researchers, and has been shown by the likes of \citet{Xinghao_2013} and \citet{Ma_2019} to further improve estimation of traffic state, and ultimately arrival times.


Of course, looking outside of transit altogether we see that traffic flow modelling has been the focus of much research, and some of these methods have been brought into the transit realm. \citet{Julio_2016} developed a shockwave model which used up- and down-stream changes in traffic flows to make forecasts of bus travel times. \citet{Salonen_2013} compared the travel times of buses and cars \ldots


\subsection{Arrival times and journey planning}
\label{sec:intro_etas}


Arrival time prediction of transit vehicles is at first a simple task, but a little inspection shows that underneath there is a lot of complexity, from the types of data through to the models used to make the predictions. These models need to account for a range of systems: vehicle motion, traffic flow behaviour, and stop service times. Perhaps the least explored component of arrival time is indeed the uncertainty associated with it: even in the best of circumstances, we are unlikely to be able to reuce it to zero. Traffic lights, pedestrian crossings, passenger demand at stops, interactions with other buses, weather, and myriad other causes will affect the final arrival time. \citet{Mazloumi_2011} presented some work that involved assessing the coverage probability of a \emph{prediction interval} to convey the remaining uncertainty, but we found no other examples of prediction intervals pertaining to arrival time prediction.


The final aspect of \gls{rti} of significance is \emph{journey planning}, which is the process of determining the best route from origin to destination given a set of conditions, for example departure or arrival time. Early work by \citet{Horn_2004} used simple algorithms to minimise walking and waiting, and were able to make decisions in \rt{}. However, the complexity of this task can get large, particularly when a journey consists of multiple legs (separate bus journeys). \citet{Hame_2013a} developed a Markov decision process to explore all of the possible options under the constraint of arriving on time, which took up to 30~second to solve for some specific journeys. More recently \citet{Zheng_2016} proposed a method that accounted for travel time uncertainty, which is often significant. Even more general is a method by \citet{Berczi_2017} in which any probabilitic model of arrival time could be used as an input to determine an optimal journey plan.




\section{Arrival time prediction in Auckland and abroad}
\label{sec:auckland_etas}

Our principal study area is Auckland, New Zealand, which has a public transport service operated by Auckland Transport. The examples given and the issues discussed in this thesis are primarily related to Auckland Transport, but are  likely applicable to other providers around the world. The primary features of Auckland that make it difficult to predict arrival time are heavy congestion along arterial roads, many of which have no or inadequate transit infrastructure, such as bus lanes. Additionally, the scheduled adherence of vehicles is often poor, as the majority of routes are timed only by their initial start time (that is, the time the bus departs from the first stop). What's more, poor communication between subcontractors (\AT{} subcontracts individual routes out to various bus companies) means that, where one might expect two buses to link temporally and provide a convenient transfer, they do not---often one arrives at a transfer station just as the connecting route departs, which can, in extreme cases, result in a wait of 30~minutes or more, which I have experienced more than once myself.


Perhaps the main issue with \AT{} is that, while they may be investing money in select infrastructural projects, the vast majority of routes will still be running along roads with no priority, and neither is there any foreseable change to schedule adherance (bus drivers actively keeping to the scheduled), since ``Punctuality is measured by the percentage of total scheduled services leaving their origin stop no more than one minute early or five minutes late'' \citep[13]{AT_report_2019}. While this might not necessarily be a problem outside of Auckland, it highlights the issues that simply using the scheduled travel time between stops is inadequate and unreliable.
\note{Rewrite last sentence/paragraph with more emphasis on problematic nature of current system.}


Perhaps the most interesting point about all of this is that, despite all of this work, many public transport providers still use very simple arrival time prediction mechanisms. The Auckland Transport system, which contains \rt{} vehicle location information, only uses the arrival or departure delay from the most recently visited stop to update the arrival times estimate for upcoming stops. As a result, the system is often unreliable, particularly when traffic is more congested than usual, and leads to much frustration of passengers. One possible reason for this is that many of the arrival time predictions schemes have been developed for a particular city, rather than for transit in general. At first, this may seem reasonable: they are, after all, different cities. However, with the global adoption of \gls{gtfs} (details in \cref{sec:gtfs}), one would expect this not to be the case: ``The standardised format means that innovative tools and products that utilize GTFS can easily be applied accross transit agencies'' \citep[26]{TCRP_2020}.



Many of the arrival time prediction deployments around the globe use location-specific information to model the real-time state of traffic, as we saw with taxi data being the primary example. Other methods use location-specific designs: \citet{Celan_2017,Celan_2018} located stops and other potential time barriers (for example traffic lights, roundabouts, or pedestrian crossings) along specific routes in two major Slovenian cities, allowing them to develop a \emph{network} of nodes (stops, intersection) with edges (roads) connecting them. They found that ignoring potential time barriers and only using stop locations resulted in poorer predictive performance. \note[The issue is, this information is seldom available, so such networks are not available in general.]{This last sentence is weak, Tom}


Another \note[down side]{one word? better word?} of location-specific setups is exactly that they are location-specific, often requiring manual setup, for example, identifying intersections or connecting to unique data sources (taxis). With \gls{gtfs} becoming widespread, it would be desirable to have a generalised framework which could, in its simplest form, take a standardised transit data set and generate a transit network that could be modelled in \rt{}, providing arrival times predictions account for the all important travel time information. Any such framework should also take into account stop dwell times, but be extensible so that once deployed, it could be modified to suit any specific location. \note[To the best of our knowledge, no such system exists that uses a \gls{gtfs}-based system to make probabilistic arrival time estimates; that is, a distribution of arrival times that accounts for uncertainty in the system.]{This doesn't tie in very nicely}




\section{Research proposal}
\label{eq:proposal}

Public transport is essential for growing cities where travellers need to get to and from work, school, or other activities in a reliable manner. One major source of unreliability, particularly in Auckland, is with the \rt{} arrival predictions, which are often inaccurate and not trustworthy: sometimes the bus does not even show up, while \note[at others]{try again\ldots} the \gls{eta} increases as the bus encounters heavy traffic, or rapidly decreases as the bus approaches faster than expected. This can deter commuters from choosing to use public transport and prefering private means.


\note{Rewrite this paragraph\ldots}
The first part of this research focuses on developing a generalised arrival time prediction framework (not just specifically for Auckland) that can combine data from multiple routes to model traffic congestion, in \rt{}, in an effort to reduce the uncertainty in arrival time estimation. One complete, we will be able to assess the effectiveness of this approach, and determine whether or not more reliable predictions are indeed possible. One approach to reliability improvement will be to introduce the concept of \emph{prediction intervals} in an effort to account for the remaining uncertainty.


A second, related issue is uncertainty in journey planning, particularly when transfers between services are required. As described earlier, it is not uncommon to miss a transfer by a few minutes. It is for this reason that we will explore not only point or interval prediction, but indeed estimating the full arrival time distribution, which one could then use to estimate the probabilities of certain events common in journey planning: will I arrive on time?\ or how long will I have to wait at the transfer point?



To approach this problem, we begin by providing an overview of the relevant information in \cref{cha:data}, most notably of the \gls{gtfs} standard. This is followed up by an overview of the data available to us in Auckland, which includes vehicle positions and trip updates (times of arrival at and departure from bus stops). As many have noted, using a \emph{network} representation of roads and bus stops improves estimation and prediction, particularly when information from different routes can be combined, so we also describe the process of constructing a network from \gls{gtfs} data. We also include a brief overview of \rt{} modelling with \glspl{rbm}, before finally discussing the \rt{} implementation of our application using \textsf{Rcpp}.


A lot of previous research has using the concept of \emph{vehicle state}, which is updated as new information is obtained from the \emph{tracker} \citep{Cathey_2003}. However, this introduces an unnecessary source of error associated with map-matching. To bypass this, we propose a \rt{} vehicle model in \cref{cha:vehicle_model} that uses the observed \gls{gps} data directly in the likelihood function, which allows us to overcome some of the issues that will be introduced in \cref{cha:data}. The primary goal of the vehicle model is to estimate vehicle speeds along roads throughout the network, as this has been shown to be one of the primary sources of arrival time uncertainty \citep{Shalaby_2004,Yu_2010,Yu_2011,Yin_2017}.


When it comes to the network itself, there have been many alternate approaches to modelling and predicting travel times. A common method has been to use historical data combined with \rt{} vehicle information---for example, the vehicles speed or travel time along previous roads. In \cref{cha:network_model}, we present a simple \rt{} model that uses the \rt{} speed information obtained from the vehicle model to update the state of the network. That is, we use \rt{} transit data to infer \rt{} traffic conditions throughout the network. We also provide an example of how, with further research, it could be improved to make more accurate forecasts of arrival time.


The last component of this research is to make arrival time predictions and assess their reliability, which we perform in two stages. The first stage (\cref{cha:prediction}) involves making arrival time predictions by combining both the vehicle and network states to obtain an arrival time distribution. We assess the statistical properties of the method in comparison to several alternatives. The second stage (\cref{cha:etas}) involves an exploration of \emph{journey planning}, using the arrival time distribution to make informed decisions. This involves comparing point estimates, prediction intervals, and a brief overview of the possibilities of using distributions for decision making: which bus should I catch to maximise my change of getting to work on time? In both chapters, we compare our method with the currently deployed method in Auckland, which uses solely \gls{gtfs} information to make the predictions.


Of course, this is a \rt{} application, so throughout all of it we must consider the computational constraints. While many methods currently deployed run in seconds, we set ourselves a goal of running a full interation in no more than 30~seconds, from start (requesting the latest data from the server) to finish (the predictions available for users). The complexities associated with this constraint will be discussed throughout.


Finally, in \cref{cha:discussion}, we review our methods and findings, and comment on what could be done to improve the predictions. The role of \gls{rti} in public transport is only going to become more critical as cities---particularly Auckland---grow, and improving the reliability of the information is the first step to increasing ridership.
