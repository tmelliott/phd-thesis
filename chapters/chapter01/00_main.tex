\glsresetall

\chapter{Introduction}
\label{cha:intro}

Over the last three decades, technological advances have changed the way people use public transport. In times gone by, one might have used a paper timetable to decide when to leave home, and then be subjected to an indefinite wait until the bus finally appeared around the corner. These days, you just take out your phone to see the bus' precide location on a map. Despite this revolutionary change, to this day it remains a complicated business of converting a location into an \gls{eta}.


Around the world, public transport providers have invested in systems to provide \gls{rti} to their passengers in efforts to improve ridership, the number of commuters choosing to use public transport. Research has shown that \gls{rti} reduced the experienced wait time at bus stops \citep{TCRP_2003} and allows for improved decision making by passengers, leading to shorter waiting times \citep{Cats_2015,Lu_2017}. This, of course, assumes the \gls{rti} is reliable, wherein lies the issue.


One of the main issues with \gls{rti} is that it requires considerable infrastruture to keep track of and report the information \citep{TCRP_2003b}, particularly in recent times when fleet sizes exceed 1000~vehicles. This has inevitably led to a simplification of how arrival times are predicted, which consequently has led to less reliable information, particularly in Auckland, New Zealand.


\section{The evolution of \acrlong{rti}}
\label{sec:literature}

For a long time \pt{} services such as buses and trains used static timetables that passengers could use to plan their journey. In the 1960s, the first vehicle tracking systems, which used fixed signposts to detect the passing of a vehicle, were trialed in Germany and the USA \citep{TCRP_1997}. This allowed the transit provider to monitor the \rt{} performance of its services from a central location.


Over the next few decades, \gls{avl} technology was slowly deployed primarily as a monitoring tool for service providers. The first use of \gls{avl} in \gls{rti} was as a \gls{dms} at bus stops displaying the \gls{eta}, in minutes, until the bus arrives \citep{TCRP_2003}. The most prominent example at the time was Transport for London's \emph{Countdown} system, which used the signpost \gls{avl} system with an odometer to determine the vehicle's position and provide \glspl{eta}.



The most difficult challenge for the provision of accurate \gls{rti} is time itself, particularly in the 1990s when computers were much less powerful than they are today. In response to this, simple, fast algorithms were developed that used \rt{} location information to update the estimates of arrival times, the most prominent being the \kf{}. \cite{Reinhoudt_1997} implemented a \kf{} to model the \emph{link travel time}, replacing the weighted average used by Transport for London's Countdown system, and found that it led to more accurate predictions of arrival time. Later, \cite{Wall_1999} and \cite{Dailey_2001} implemented \kf{}'s to track the state of a vehicle, allowing for uncertainty in its location and speed to be incorporated into the arrival time prediction, which was based on a \emph{travel time function} estimated using historical data.



The main problem with the signpost \gls{avl} system is that it needs to be rolled out for each individual route. It is therefore no surprise that many transit providers began to favour the distribution of \gls{gps} devices to track their fleets. The result of this was a necessity for a much more generalised prediction framework, since data was no longer route-specific and instead required more complex methods for both tracking and prediction. \Cite{Cathey_2003} released a general prescription in which they described the three necessary components to arrival time prediction: tracking, filtering, and prediction.


Tracking is, in our context, the process of combining schedule information with \rt{} location information to acquire the necessary data for the rest of the model. For transit vehicle tracking using a \gls{gps}, the first step is to match the location to the route: geographic observations are processed to obtain the necessary data, for example \emph{trip distance travelled} \citep{Cathey_2003,Gong_2013}. Other methods involved using arrival time observations, in which case the travel time between \emph{time points} (using significant stops along the route) is observed \citep{Shalaby_2004,Jeong_2005,Yu_2011}. In most cases this is adequate, but there are some situations that can induce unreliability in the final estimates, as we discuss later.


The filtering component is responsible for taking a sequence of vehicle observations (as determined by the \emph{tracker}) and estimating the \emph{state} of the vehicle, which often includes distance travelled along the route and speed \citep{Dailey_2001,Cathey_2003}. This is where the computational efficiency of the \kf{} plays its pivotal role, since the state can easily and quickly be estimated (we discuss this in detail in \cref{sec:recursive-bayes}). However, despite being adopted early, more recent applications favour deterministic methods, such as arrival time reporting \citep{Yin_2017,Cats_2015,Cats_2016}. While easy to implement, and often accurate, this approach can reduce reliability in the presence of poor data (such as erroneous or missing arrival or departure times), as observed in the Auckland Transport system.


The final component is the predictor, which combines the vehicle location information with a variety of other information (which can be historical or \rt{}). The general concensus is that this `other information' needs to include \emph{link travel time}---that is, the time taken to travel between various points along the route---and \emph{stop dwell time}---which is the time lost while servicing a bus stop. Early work, such as by \cite{Reinhoudt_1997} and \cite{Wall_1999}, demonstrated the use of \emph{link travel times} to inform the arrival time of upcoming buses. The effect of dwell time has also been shown to greatly influence arrival time uncertainty \citep{Jeong_2005,Meng_2013,Shen_2013,Robinson_2013,Gong_2013,Wang_2016}, though in most of these the available data has been either manually collected or available through technologies such as \glspl{apc}, which are less common than \gls{gps}.



Of course, not only has the technology for tracking a vehicle improved over the years: so have computers, leading to more and more complex prediction schemes. \Citet{Yu_2006} and \citet{Yu_2010} showed that a \gls{svm} trained with historical link travel time data could better predict the arrival times of upcoming buses based on the travel time of the preceeding bus, and \cite{Yu_2011} extended the idea to involve vehicles from multiple routes, showing significant improvments over other methods. However, the main limitation was that the routes were pre-specified, which is a common constraint of proposed methods \citep{Chang_2010,Celan_2017,Celan_2018}.


A lot of the innovation for modelling transit vehicles has been in operational research, rather than \gls{rti}. For example \citet{Hans_2015} developed a particle filter to model buses in \rt{} in an attempt to reduce bunching (when consecutive trips end up travelling one after the other rather than spaced out temporally).



\paragraph{Adoption of GTFS globally led to better access to transit data, but also to the loss of better models---instead of predicting arrival times using fancy models, just adjust schedule with current delay (low server demand)...}
This is less reference-based and more a personal observation ...
\cite{TCRP_2020}.


\paragraph{Focus of vehicle modelling shifted to operations management---how do we make buses run on time better? But this needs independent research and implementation in each region.}
\cite{Wessel_2016, Hans_2015}.


\paragraph{As part of this, some great new models of vehicle behaviour arose, namely the particle filter (until now, it was too computationally demanding for real-time use).}
\cite{Hans_2015b,Chen_2014}.


\paragraph{Most recently, the current state-of-the-art varies between cities, countries, as much of the research is location-specific: for example, X, Y, and Z.}


\paragraph{The concensus: you need to use real-time information to predict arrival times, using as much data as possible (taxis? other buses?). The hard part is getting that data, and as far as we could find, there is no general method for combining data from multiple routes to use in arrival time prediction.}
\cite{Ma_2019,Salonen_2013,Xinghao_2013}.


\paragraph{Other issues involve journey planning, particularly for multi-leg trips.}
\cite{Horn_2004,Hame_2013a,Hame_2013b,Zheng_2016,Berczi_2017}.



\pagebreak
Over the last three decades, technological advances have changed the way people use public transport.
No longer do commuters arrive at a bus stop, take a seat,
and wait as long as it takes for their bus to appear at the corner;
instead they can open an app on their phone and see, on a map,
just how far away their bus is.
Indeed, the question of \emph{how far away is my bus?}
has become one of \emph{how long until my bus arrives?},
which is difficult to judge solely from the bus' location.


So, just how is a position on a map turned into a
decreasing number that reaches zero just as the bus reaches the stop?
Usually, it's not.
Most public transport users are well aware of the inaccuracy of the current system,
with \glspl{eta} often reaching zero with no bus in sight.
Despite decades of research into predicting bus arrival (see section~\ref{sec:literature}),
the responsibility of arrival time prediction has moved from a probabilitic model
to a deterministic function based soley on observations and static schedule information.


To reliably predict a bus's arrival time,
we need to know two things:
\emph{where it is now}, and \emph{how long it will take to get here}.
As mentioned above, the first of these is readily available thanks to  GPS technology;
the latter is simply a function of tavel time,
which depends on traffic conditions: free-flowing, slow, or crawling?
Sadly, this is an unknown quantity;
while information of this type is available (for example in Google Maps),
it is not \emph{accessible} for use in any predictive context.
But surely there is a way.
Surely, with the amount of data available, there is a way of figuring out
just how congested the intermediate roads are,
and using this to improve the reliability of \glspl{eta}.


In this thesis, we present an approach to bus modelling that
uses one of the most widely available transit data formats
(GTFS, chapter~\ref{cha:data})
to not only track the \rt{} locations of vehicles
(chapter~\ref{cha:vehicle_model}),
but also to model the congestion along roads within the transport network
(chapter~\ref{cha:network_model}).
This allows traffic flow to be incorporated into arrival time predictions
(chapter~\ref{cha:prediction}),
providing commuters with a more reliable answer to their question,
\emph{just how long until my bus arrives?}
(chapter~\ref{cha:etas}).


\textbf{Old Stuff}


Ever since the first use of an \gls{avl} technology in 1964, in Hamburg, Germany,
it has become an integral part of most transit systems around the world
\citep{TCRP_1997,TCRP_2003}.
The first \gls{avl} systems used \emph{signpost} technology,
in which buses are fitted with a transponder that communicates with
sensors positioned along the route.
Other technologies include \emph{odometers},
which provide measurements of the distance traveled by a vehicle,
and more commonly in recent years the \gls{gps}.


In order to convert \gls{gps} coordinates into arrival time predictions,
several steps need to be taken.
The first of these is to estimate the \emph{actual state} of the bus,
for example its speed and how far into the route it has traveled---%
referred to as \emph{distance into trip}---%
which is often estimated directly from the \gls{gps} coordinates
(see \cref{sec:kalman-filter}),
and needs to account for \gls{gps} error.
The second step uses the estimated state
to predict how long the bus will take to travel from its current position
to a position farther along the route, usually a bus stop,
referred to as \emph{travel time};
given travel time and the current time,
we can compute \emph{arrival time} at a stop.


In this section, we give an overview of some of the models that have been used
to generate \rt{} arrival time predictions,
with particular focus on Kalman and \pf{}ing.
We will briefly discuss some computer learning models which---%
although we will not be using them---%
cover important ground with respect to the important features of bus models.
Following this, we will describe in detail some of the
behaviours mentioned in the literature,
and finally discuss the difficulties associated with deploying many of these methods,
with particular mention of the type of data available.



%%% the sections of this chapter
\input{chapters/chapter01/01_literature.tex}
\input{chapters/chapter01/intro_history.tex}
